\documentclass[a4paper]{article}
\usepackage[a4paper, margin=1in]{geometry}
\input{preamble}
\title{\Huge{Commutative Algebra}}
\author{\huge{Daniel Yu}}
\date{September 24,2024}

\pdfsuppresswarningpagegroup=1

\begin{document}
\maketitle
\newpage% or \cleardoublepage
% \pdfbookmark[<level>]{<title>}{<dest>}
\tableofcontents
\pagebreak

Just Randomly dropped in lmao

\section{Something}
\begin{note}{Examples of Spec(A) and mSpec(A)}
  \begin{enumerate}
    \item k field, $A = k[x]$, then  $Spec(A) = \{\left( 0 \right) \} \cup \{ \left( f \right)  | f \text{ irreducible} \} $,
      $Spec(A) \setminus \{\left( 0 \right) \} $. When k algebraically closed, the irreducible polynomials in $k[x]$ are of the
      form  $b(x-a)$,  $a,b \in k$ $\to$  $spec(k[x]) = k$.   
    \item What if $k$ not algebraically closed?  $k = \R$. We set nonlinear irreducible $f \in k[x]$. Goal: determine a geometric
      interpretation of $\left( f \right) \in Spec(k[x])$ where $deg(f) >1$ irreducible. Let $k[x,y]$, then  $ \left( 0  \right) 
      \subseteq (X) \subseteq (x,y)$ where all ideals in this chain are prime, but only $(x,y)$ maximal,  $Spec(k[x,y]) \neq m Spec(k[x,y])$.
      Hw: prove $\left( [x.y] \right)$ is maximal. Generally, if $(a.b) \in k^2$, the ideal generated by  $x-a$ and  $y-b$:
       \[
      M_{(a,b)} = \left( x-a,y-b \right)  
      .\] 
      is maximal in $k[x,y]$. We can make a similar identitfication with  at least some part of  $mSpec(k[x,y])$ with  $k^2$, the k-plane. 
    \item etc.
  \end{enumerate}
\end{note}  
Highkey lost rn

\begin{definition}
  Given a field $k$, then a field  $F$ containing  $k$ so that addition and multiplication agree, we call  $F$
  a field extension of  $k$ and  $k$ the base field and write:
   \[
     \frac{F}{k} = \text { f over k}
  .\] 
\end{definition}
\begin{note}
    Given a field $k$, there is no natural way to give the set  $k^{2}$ the structure of a field:
  \[
    (0,1) \cdot (1,0) = (0,0) \text{ nonzero zero divisors and not a field}
  .\] 
  This is not to say that there are not definitions that turn $k^2$ into a ring:
  Consider $\R^{2}$ and define:
  \[
    (a,b) \cdot (c,d) = (ac - bd, ad + bc) 
  .\] 
  \[
    (a + bi)(c + di) = (ac - bd) + i(ad + bc)
  .\] 
\end{note}
\begin{remark}{Holy!!!}\\  
  $\R^{2}$ with the multiplication defined above is a field extension of $\R$ i.e. commonly known
  as $\C$:
   \[
     \C \cong \frac{\R[x]}{x^{2} + 1} = \R[i]
  .\] 
  And this generalizes for other $k$, the underlying set!
\end{remark}

\section{Nilpotent elements and Nilradical}
\begin{definition}
  \begin{enumerate}
    \item $x \in A$ is nilpotent if  $x^n = 0$ for some  $n \in \N$. 
    \item the set of all nilpotent elements of $A$ is called the \textbf{nilradical} of $A$ and
      denoted:
      \[
        nilrad(A) = \{a \in A | a^n = 0, \text{some n > 0}\} 
      .\]
    \item $A$ is reduced if  nilrad(A) = $\left( 0 \right) $ : 
  \end{enumerate}
\end{definition}

\begin{prop}
  A integral domain $\to$ A reduced. 
\end{prop}
\begin{prop}
  Furthermore $\frac{A}{nilrad(A)}$ is always reduced.
\end{prop}
\begin{prop}
  $nilrad(A) = \cap_{P \in Spec(A)} P$ 
\end{prop}

\begin{definition}
  The jacobson radical of ring $A$ is the intersection of all maximal ideals of $A$.
   \[
  Jrad(A) = \cap_{M \in mSpec(A)} M 
  .\] 
\end{definition}

\begin{prop}
  $x \in Jrad(A) \iff 1 - xy \in A^x$ for all $y \in A$.

  \begin{proof}
    $\to$ Assume  $x \in jrad(A)$ and  $1-xy \not\in A^{x}$ some $y \in A$. Then we know that $1-xy$ must  
    be contained in some maximal ideal of $A$, denote it as $M \subseteq A$ maximal ideal (any non-unit has to be in some maximal ideal). Since  $x \in M$,
    then $xy \in M$ (since  $x$  is in an ideal). Then, $(1-xy) + xy = 1 \in M$, so  $M = A$ and hence is not 
    a maximal ideal, a contradiction, so $1-xy$ can't be contained in a maximal ideal of  $A$. \\
    
    $\leftarrow$ Suppose  $x \not\in M$ some maximal  $M \subseteq A$ then the ideal generated by  $x$ and
     $M$ must be all of  $A$ then $u + xy = 1$ for some  $u \in M$ and  $y \in A \iff 1-xy \in M$ and hence
     $1 - xy \not\in A^x$.
  \end{proof}
\end{prop}

\begin{definition}
  $I \subseteq A$ ideal, the \textbf{radical} of the ideal $I$ is the set 
   \[
     \sqrt{I} = rad(I) = \{x \in A | x^n = I, \text{ some n > 0}\}  
  .\] 
\end{definition}

\begin{remark} In some sense, radical is a generalization of nilradical
   \[
  nilrad(A) = rad((0)), (0) \subseteq A
  .\] 
\end{remark}

\begin{prop}
  $rad(I) = \pi^{-1} (nilrad(\frac{A}{I}))$ where $\phi: A \to \frac{A}{I}$ 
  
  \begin{proof}
    $x \in rad(I) \iff x^{n} \in I$ some $n \in N$, so  $\pi(x^{n}) = \pi(x)^{n} = (x+I)^{n} = x^{n} + I = 
    O_{\frac{A}{I}} = \pi(O_A)$.

    I'm lost
  \end{proof}
\end{prop}

\end{document}
