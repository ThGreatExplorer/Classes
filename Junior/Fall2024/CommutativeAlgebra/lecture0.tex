\documentclass[a4paper]{article}
\usepackage[a4paper, margin=1in]{geometry}
% Some basic packages
\usepackage[utf8]{inputenc}
\usepackage[T1]{fontenc}
\usepackage{textcomp}
\usepackage[dutch]{babel}
\usepackage{url}
\usepackage{graphicx}
\usepackage{float}
\usepackage{booktabs}
\usepackage{enumitem}

\pdfminorversion=7

% Don't indent paragraphs, leave some space between them
\usepackage{parskip}

% Hide page number when page is empty
\usepackage{emptypage}
\usepackage{subcaption}
\usepackage{multicol}
\usepackage{xcolor}

% Other font I sometimes use.
% \usepackage{cmbright}

% Math stuff
\usepackage{amsmath, amsfonts, mathtools, amsthm, amssymb}
% Fancy script capitals
\usepackage{mathrsfs}
\usepackage{cancel}
% Bold math
\usepackage{bm}
% Some shortcuts
\newcommand\N{\ensuremath{\mathbb{N}}}
\newcommand\R{\ensuremath{\mathbb{R}}}
\newcommand\Z{\ensuremath{\mathbb{Z}}}
\renewcommand\O{\ensuremath{\emptyset}}
\newcommand\Q{\ensuremath{\mathbb{Q}}}
\newcommand\C{\ensuremath{\mathbb{C}}}

% Easily typeset systems of equations (French package)
\usepackage{systeme}

% Put x \to \infty below \lim
\let\svlim\lim\def\lim{\svlim\limits}

%Make implies and impliedby shorter
\let\implies\Rightarrow
\let\impliedby\Leftarrow
\let\iff\Leftrightarrow
\let\epsilon\varepsilon

% Add \contra symbol to denote contradiction
\usepackage{stmaryrd} % for \lightning
\newcommand\contra{\scalebox{1.5}{$\lightning$}}

% \let\phi\varphi

% Command for short corrections
% Usage: 1+1=\correct{3}{2}

\definecolor{correct}{HTML}{009900}
\newcommand\correct[2]{\ensuremath{\:}{\color{red}{#1}}\ensuremath{\to }{\color{correct}{#2}}\ensuremath{\:}}
\newcommand\green[1]{{\color{correct}{#1}}}

% horizontal rule
\newcommand\hr{
    \noindent\rule[0.5ex]{\linewidth}{0.5pt}
}

% hide parts
\newcommand\hide[1]{}

% si unitx
\usepackage{siunitx}
\sisetup{locale = FR}

% Environments
\makeatother
% For box around Definition, Theorem, \ldots
\usepackage{mdframed}
\mdfsetup{skipabove=1em,skipbelow=0em}
\theoremstyle{definition}
\newmdtheoremenv[nobreak=true]{definitie}{Definitie}
\newmdtheoremenv[nobreak=true]{eigenschap}{Eigenschap}
\newmdtheoremenv[nobreak=true]{gevolg}{Gevolg}
\newmdtheoremenv[nobreak=true]{lemma}{Lemma}
\newmdtheoremenv[nobreak=true]{propositie}{Propositie}
\newmdtheoremenv[nobreak=true]{stelling}{Stelling}
\newmdtheoremenv[nobreak=true]{wet}{Wet}
\newmdtheoremenv[nobreak=true]{postulaat}{Postulaat}
\newmdtheoremenv{conclusie}{Conclusie}
\newmdtheoremenv{toemaatje}{Toemaatje}
\newmdtheoremenv{vermoeden}{Vermoeden}
\newtheorem*{herhaling}{Herhaling}
\newtheorem*{intermezzo}{Intermezzo}
\newtheorem*{notatie}{Notatie}
\newtheorem*{observatie}{Observatie}
\newtheorem*{oef}{Oefening}
\newtheorem*{opmerking}{Opmerking}
\newtheorem*{praktisch}{Praktisch}
\newtheorem*{probleem}{Probleem}
\newtheorem*{terminologie}{Terminologie}
\newtheorem*{toepassing}{Toepassing}
\newtheorem*{uovt}{UOVT}
\newtheorem*{vb}{Voorbeeld}
\newtheorem*{vraag}{Vraag}

\newmdtheoremenv[nobreak=true]{definition}{Definition}
\newtheorem*{eg}{Example}
\newtheorem*{notation}{Notation}
\newtheorem*{previouslyseen}{As previously seen}
\newtheorem*{remark}{Remark}
\newtheorem*{note}{Note}
\newtheorem*{problem}{Problem}
\newtheorem*{observe}{Observe}
\newtheorem*{property}{Property}
\newtheorem*{intuition}{Intuition}
\newmdtheoremenv[nobreak=true]{prop}{Proposition}
\newmdtheoremenv[nobreak=true]{theorem}{Theorem}
\newmdtheoremenv[nobreak=true]{corollary}{Corollary}

% End example and intermezzo environments with a small diamond (just like proof
% environments end with a small square)
\usepackage{etoolbox}
\AtEndEnvironment{vb}{\null\hfill$\diamond$}%
\AtEndEnvironment{intermezzo}{\null\hfill$\diamond$}%
% \AtEndEnvironment{opmerking}{\null\hfill$\diamond$}%

% Fix some spacing
% http://tex.stackexchange.com/questions/22119/how-can-i-change-the-spacing-before-theorems-with-amsthm
\makeatletter
\def\thm@space@setup{%
  \thm@preskip=\parskip \thm@postskip=0pt
}


% Exercise 
% Usage:
% \oefening{5}
% \suboefening{1}
% \suboefening{2}
% \suboefening{3}
% gives
% Oefening 5
%   Oefening 5.1
%   Oefening 5.2
%   Oefening 5.3
\newcommand{\oefening}[1]{%
    \def\@oefening{#1}%
    \subsection*{Oefening #1}
}

\newcommand{\suboefening}[1]{%
    \subsubsection*{Oefening \@oefening.#1}
}


% \lecture starts a new lecture (les in dutch)
%
% Usage:
% \lecture{1}{di 12 feb 2019 16:00}{Inleiding}
%
% This adds a section heading with the number / title of the lecture and a
% margin paragraph with the date.

% I use \dateparts here to hide the year (2019). This way, I can easily parse
% the date of each lecture unambiguously while still having a human-friendly
% short format printed to the pdf.

\usepackage{xifthen}
\def\testdateparts#1{\dateparts#1\relax}
\def\dateparts#1 #2 #3 #4 #5\relax{
    \marginpar{\small\textsf{\mbox{#1 #2 #3 #5}}}
}

\def\@lecture{}%
\newcommand{\lecture}[3]{
    \ifthenelse{\isempty{#3}}{%
        \def\@lecture{Lecture #1}%
    }{%
        \def\@lecture{Lecture #1: #3}%
    }%
    \subsection*{\@lecture}
    \marginpar{\small\textsf{\mbox{#2}}}
}



% These are the fancy headers
\usepackage{fancyhdr}
\pagestyle{fancy}

% LE: left even
% RO: right odd
% CE, CO: center even, center odd
% My name for when I print my lecture notes to use for an open book exam.
% \fancyhead[LE,RO]{Gilles Castel}

\fancyhead[RO,LE]{\@lecture} % Right odd,  Left even
\fancyhead[RE,LO]{}          % Right even, Left odd

\fancyfoot[RO,LE]{\thepage}  % Right odd,  Left even
\fancyfoot[RE,LO]{}          % Right even, Left odd
\fancyfoot[C]{\leftmark}     % Center

\makeatother




% Todonotes and inline notes in fancy boxes
\usepackage{todonotes}
\usepackage{tcolorbox}

% Make boxes breakable
\tcbuselibrary{breakable}

% Verbetering is correction in Dutch
% Usage: 
% \begin{verbetering}
%     Lorem ipsum dolor sit amet, consetetur sadipscing elitr, sed diam nonumy eirmod
%     tempor invidunt ut labore et dolore magna aliquyam erat, sed diam voluptua. At
%     vero eos et accusam et justo duo dolores et ea rebum. Stet clita kasd gubergren,
%     no sea takimata sanctus est Lorem ipsum dolor sit amet.
% \end{verbetering}
\newenvironment{verbetering}{\begin{tcolorbox}[
    arc=0mm,
    colback=white,
    colframe=green!60!black,
    title=Opmerking,
    fonttitle=\sffamily,
    breakable
]}{\end{tcolorbox}}

% Noot is note in Dutch. Same as 'verbetering' but color of box is different
\newenvironment{noot}[1]{\begin{tcolorbox}[
    arc=0mm,
    colback=white,
    colframe=white!60!black,
    title=#1,
    fonttitle=\sffamily,
    breakable
]}{\end{tcolorbox}}




% Figure support as explained in my blog post.
\usepackage{import}
\usepackage{xifthen}
\usepackage{pdfpages}
\usepackage{transparent}
\newcommand{\incfig}[1]{%
    \def\svgwidth{\columnwidth}
    \import{./figures/}{#1.pdf_tex}
}

% Fix some stuff
% %http://tex.stackexchange.com/questions/76273/multiple-pdfs-with-page-group-included-in-a-single-page-warning
\pdfsuppresswarningpagegroup=1

\title{\Huge{Commutative Algebra}}
\author{\huge{Daniel Yu}}
\date{September 24,2024}

\pdfsuppresswarningpagegroup=1

\begin{document}
\maketitle
\newpage% or \cleardoublepage
% \pdfbookmark[<level>]{<title>}{<dest>}
\tableofcontents
\pagebreak

Just Randomly dropped in lmao

\section{Something}
\begin{note}{Examples of Spec(A) and mSpec(A)}
  \begin{enumerate}
    \item k field, $A = k[x]$, then  $Spec(A) = \{\left( 0 \right) \} \cup \{ \left( f \right)  | f \text{ irreducible} \} $,
      $Spec(A) \setminus \{\left( 0 \right) \} $. When k algebraically closed, the irreducible polynomials in $k[x]$ are of the
      form  $b(x-a)$,  $a,b \in k$ $\to$  $spec(k[x]) = k$.   
    \item What if $k$ not algebraically closed?  $k = \R$. We set nonlinear irreducible $f \in k[x]$. Goal: determine a geometric
      interpretation of $\left( f \right) \in Spec(k[x])$ where $deg(f) >1$ irreducible. Let $k[x,y]$, then  $ \left( 0  \right) 
      \subseteq (X) \subseteq (x,y)$ where all ideals in this chain are prime, but only $(x,y)$ maximal,  $Spec(k[x,y]) \neq m Spec(k[x,y])$.
      Hw: prove $\left( [x.y] \right)$ is maximal. Generally, if $(a.b) \in k^2$, the ideal generated by  $x-a$ and  $y-b$:
       \[
      M_{(a,b)} = \left( x-a,y-b \right)  
      .\] 
      is maximal in $k[x,y]$. We can make a similar identitfication with  at least some part of  $mSpec(k[x,y])$ with  $k^2$, the k-plane. 
    \item etc.
  \end{enumerate}
\end{note}  
Highkey lost rn

\begin{definition}
  Given a field $k$, then a field  $F$ containing  $k$ so that addition and multiplication agree, we call  $F$
  a field extension of  $k$ and  $k$ the base field and write:
   \[
     \frac{F}{k} = \text { f over k}
  .\] 
\end{definition}
\begin{note}
    Given a field $k$, there is no natural way to give the set  $k^{2}$ the structure of a field:
  \[
    (0,1) \cdot (1,0) = (0,0) \text{ nonzero zero divisors and not a field}
  .\] 
  This is not to say that there are not definitions that turn $k^2$ into a ring:
  Consider $\R^{2}$ and define:
  \[
    (a,b) \cdot (c,d) = (ac - bd, ad + bc) 
  .\] 
  \[
    (a + bi)(c + di) = (ac - bd) + i(ad + bc)
  .\] 
\end{note}
\begin{remark}{Holy!!!}\\  
  $\R^{2}$ with the multiplication defined above is a field extension of $\R$ i.e. commonly known
  as $\C$:
   \[
     \C \cong \frac{\R[x]}{x^{2} + 1} = \R[i]
  .\] 
  And this generalizes for other $k$, the underlying set!
\end{remark}

\section{Nilpotent elements and Nilradical}
\begin{definition}
  \begin{enumerate}
    \item $x \in A$ is nilpotent if  $x^n = 0$ for some  $n \in \N$. 
    \item the set of all nilpotent elements of $A$ is called the \textbf{nilradical} of $A$ and
      denoted:
      \[
        nilrad(A) = \{a \in A | a^n = 0, \text{some n > 0}\} 
      .\]
    \item $A$ is reduced if  nilrad(A) = $\left( 0 \right) $ : 
  \end{enumerate}
\end{definition}

\begin{prop}
  A integral domain $\to$ A reduced. 
\end{prop}
\begin{prop}
  Furthermore $\frac{A}{nilrad(A)}$ is always reduced.
\end{prop}
\begin{prop}
  $nilrad(A) = \cap_{P \in Spec(A)} P$ 
\end{prop}

\begin{definition}
  The jacobson radical of ring $A$ is the intersection of all maximal ideals of $A$.
   \[
  Jrad(A) = \cap_{M \in mSpec(A)} M 
  .\] 
\end{definition}

\begin{prop}
  $x \in Jrad(A) \iff 1 - xy \in A^x$ for all $y \in A$.

  \begin{proof}
    $\to$ Assume  $x \in jrad(A)$ and  $1-xy \not\in A^{x}$ some $y \in A$. Then we know that $1-xy$ must  
    be contained in some maximal ideal of $A$, denote it as $M \subseteq A$ maximal ideal (any non-unit has to be in some maximal ideal). Since  $x \in M$,
    then $xy \in M$ (since  $x$  is in an ideal). Then, $(1-xy) + xy = 1 \in M$, so  $M = A$ and hence is not 
    a maximal ideal, a contradiction, so $1-xy$ can't be contained in a maximal ideal of  $A$. \\
    
    $\leftarrow$ Suppose  $x \not\in M$ some maximal  $M \subseteq A$ then the ideal generated by  $x$ and
     $M$ must be all of  $A$ then $u + xy = 1$ for some  $u \in M$ and  $y \in A \iff 1-xy \in M$ and hence
     $1 - xy \not\in A^x$.
  \end{proof}
\end{prop}

\begin{definition}
  $I \subseteq A$ ideal, the \textbf{radical} of the ideal $I$ is the set 
   \[
     \sqrt{I} = rad(I) = \{x \in A | x^n = I, \text{ some n > 0}\}  
  .\] 
\end{definition}

\begin{remark} In some sense, radical is a generalization of nilradical
   \[
  nilrad(A) = rad((0)), (0) \subseteq A
  .\] 
\end{remark}

\begin{prop}
  $rad(I) = \pi^{-1} (nilrad(\frac{A}{I}))$ where $\phi: A \to \frac{A}{I}$ 
  
  \begin{proof}
    $x \in rad(I) \iff x^{n} \in I$ some $n \in N$, so  $\pi(x^{n}) = \pi(x)^{n} = (x+I)^{n} = x^{n} + I = 
    O_{\frac{A}{I}} = \pi(O_A)$.

    I'm lost
  \end{proof}
\end{prop}

\end{document}
