\documentclass[a4paper]{article}
% Some basic packages
\usepackage[utf8]{inputenc}
\usepackage[T1]{fontenc}
\usepackage{textcomp}
\usepackage[dutch]{babel}
\usepackage{url}
\usepackage{graphicx}
\usepackage{float}
\usepackage{booktabs}
\usepackage{enumitem}

\pdfminorversion=7

% Don't indent paragraphs, leave some space between them
\usepackage{parskip}

% Hide page number when page is empty
\usepackage{emptypage}
\usepackage{subcaption}
\usepackage{multicol}
\usepackage{xcolor}

% Other font I sometimes use.
% \usepackage{cmbright}

% Math stuff
\usepackage{amsmath, amsfonts, mathtools, amsthm, amssymb}
% Fancy script capitals
\usepackage{mathrsfs}
\usepackage{cancel}
% Bold math
\usepackage{bm}
% Some shortcuts
\newcommand\N{\ensuremath{\mathbb{N}}}
\newcommand\R{\ensuremath{\mathbb{R}}}
\newcommand\Z{\ensuremath{\mathbb{Z}}}
\renewcommand\O{\ensuremath{\emptyset}}
\newcommand\Q{\ensuremath{\mathbb{Q}}}
\newcommand\C{\ensuremath{\mathbb{C}}}

% Easily typeset systems of equations (French package)
\usepackage{systeme}

% Put x \to \infty below \lim
\let\svlim\lim\def\lim{\svlim\limits}

%Make implies and impliedby shorter
\let\implies\Rightarrow
\let\impliedby\Leftarrow
\let\iff\Leftrightarrow
\let\epsilon\varepsilon

% Add \contra symbol to denote contradiction
\usepackage{stmaryrd} % for \lightning
\newcommand\contra{\scalebox{1.5}{$\lightning$}}

% \let\phi\varphi

% Command for short corrections
% Usage: 1+1=\correct{3}{2}

\definecolor{correct}{HTML}{009900}
\newcommand\correct[2]{\ensuremath{\:}{\color{red}{#1}}\ensuremath{\to }{\color{correct}{#2}}\ensuremath{\:}}
\newcommand\green[1]{{\color{correct}{#1}}}

% horizontal rule
\newcommand\hr{
    \noindent\rule[0.5ex]{\linewidth}{0.5pt}
}

% hide parts
\newcommand\hide[1]{}

% si unitx
\usepackage{siunitx}
\sisetup{locale = FR}

% Environments
\makeatother
% For box around Definition, Theorem, \ldots
\usepackage{mdframed}
\mdfsetup{skipabove=1em,skipbelow=0em}
\theoremstyle{definition}
\newmdtheoremenv[nobreak=true]{definitie}{Definitie}
\newmdtheoremenv[nobreak=true]{eigenschap}{Eigenschap}
\newmdtheoremenv[nobreak=true]{gevolg}{Gevolg}
\newmdtheoremenv[nobreak=true]{lemma}{Lemma}
\newmdtheoremenv[nobreak=true]{propositie}{Propositie}
\newmdtheoremenv[nobreak=true]{stelling}{Stelling}
\newmdtheoremenv[nobreak=true]{wet}{Wet}
\newmdtheoremenv[nobreak=true]{postulaat}{Postulaat}
\newmdtheoremenv{conclusie}{Conclusie}
\newmdtheoremenv{toemaatje}{Toemaatje}
\newmdtheoremenv{vermoeden}{Vermoeden}
\newtheorem*{herhaling}{Herhaling}
\newtheorem*{intermezzo}{Intermezzo}
\newtheorem*{notatie}{Notatie}
\newtheorem*{observatie}{Observatie}
\newtheorem*{oef}{Oefening}
\newtheorem*{opmerking}{Opmerking}
\newtheorem*{praktisch}{Praktisch}
\newtheorem*{probleem}{Probleem}
\newtheorem*{terminologie}{Terminologie}
\newtheorem*{toepassing}{Toepassing}
\newtheorem*{uovt}{UOVT}
\newtheorem*{vb}{Voorbeeld}
\newtheorem*{vraag}{Vraag}

\newmdtheoremenv[nobreak=true]{definition}{Definition}
\newtheorem*{eg}{Example}
\newtheorem*{notation}{Notation}
\newtheorem*{previouslyseen}{As previously seen}
\newtheorem*{remark}{Remark}
\newtheorem*{note}{Note}
\newtheorem*{problem}{Problem}
\newtheorem*{observe}{Observe}
\newtheorem*{property}{Property}
\newtheorem*{intuition}{Intuition}
\newmdtheoremenv[nobreak=true]{prop}{Proposition}
\newmdtheoremenv[nobreak=true]{theorem}{Theorem}
\newmdtheoremenv[nobreak=true]{corollary}{Corollary}

% End example and intermezzo environments with a small diamond (just like proof
% environments end with a small square)
\usepackage{etoolbox}
\AtEndEnvironment{vb}{\null\hfill$\diamond$}%
\AtEndEnvironment{intermezzo}{\null\hfill$\diamond$}%
% \AtEndEnvironment{opmerking}{\null\hfill$\diamond$}%

% Fix some spacing
% http://tex.stackexchange.com/questions/22119/how-can-i-change-the-spacing-before-theorems-with-amsthm
\makeatletter
\def\thm@space@setup{%
  \thm@preskip=\parskip \thm@postskip=0pt
}


% Exercise 
% Usage:
% \oefening{5}
% \suboefening{1}
% \suboefening{2}
% \suboefening{3}
% gives
% Oefening 5
%   Oefening 5.1
%   Oefening 5.2
%   Oefening 5.3
\newcommand{\oefening}[1]{%
    \def\@oefening{#1}%
    \subsection*{Oefening #1}
}

\newcommand{\suboefening}[1]{%
    \subsubsection*{Oefening \@oefening.#1}
}


% \lecture starts a new lecture (les in dutch)
%
% Usage:
% \lecture{1}{di 12 feb 2019 16:00}{Inleiding}
%
% This adds a section heading with the number / title of the lecture and a
% margin paragraph with the date.

% I use \dateparts here to hide the year (2019). This way, I can easily parse
% the date of each lecture unambiguously while still having a human-friendly
% short format printed to the pdf.

\usepackage{xifthen}
\def\testdateparts#1{\dateparts#1\relax}
\def\dateparts#1 #2 #3 #4 #5\relax{
    \marginpar{\small\textsf{\mbox{#1 #2 #3 #5}}}
}

\def\@lecture{}%
\newcommand{\lecture}[3]{
    \ifthenelse{\isempty{#3}}{%
        \def\@lecture{Lecture #1}%
    }{%
        \def\@lecture{Lecture #1: #3}%
    }%
    \subsection*{\@lecture}
    \marginpar{\small\textsf{\mbox{#2}}}
}



% These are the fancy headers
\usepackage{fancyhdr}
\pagestyle{fancy}

% LE: left even
% RO: right odd
% CE, CO: center even, center odd
% My name for when I print my lecture notes to use for an open book exam.
% \fancyhead[LE,RO]{Gilles Castel}

\fancyhead[RO,LE]{\@lecture} % Right odd,  Left even
\fancyhead[RE,LO]{}          % Right even, Left odd

\fancyfoot[RO,LE]{\thepage}  % Right odd,  Left even
\fancyfoot[RE,LO]{}          % Right even, Left odd
\fancyfoot[C]{\leftmark}     % Center

\makeatother




% Todonotes and inline notes in fancy boxes
\usepackage{todonotes}
\usepackage{tcolorbox}

% Make boxes breakable
\tcbuselibrary{breakable}

% Verbetering is correction in Dutch
% Usage: 
% \begin{verbetering}
%     Lorem ipsum dolor sit amet, consetetur sadipscing elitr, sed diam nonumy eirmod
%     tempor invidunt ut labore et dolore magna aliquyam erat, sed diam voluptua. At
%     vero eos et accusam et justo duo dolores et ea rebum. Stet clita kasd gubergren,
%     no sea takimata sanctus est Lorem ipsum dolor sit amet.
% \end{verbetering}
\newenvironment{verbetering}{\begin{tcolorbox}[
    arc=0mm,
    colback=white,
    colframe=green!60!black,
    title=Opmerking,
    fonttitle=\sffamily,
    breakable
]}{\end{tcolorbox}}

% Noot is note in Dutch. Same as 'verbetering' but color of box is different
\newenvironment{noot}[1]{\begin{tcolorbox}[
    arc=0mm,
    colback=white,
    colframe=white!60!black,
    title=#1,
    fonttitle=\sffamily,
    breakable
]}{\end{tcolorbox}}




% Figure support as explained in my blog post.
\usepackage{import}
\usepackage{xifthen}
\usepackage{pdfpages}
\usepackage{transparent}
\newcommand{\incfig}[1]{%
    \def\svgwidth{\columnwidth}
    \import{./figures/}{#1.pdf_tex}
}

% Fix some stuff
% %http://tex.stackexchange.com/questions/76273/multiple-pdfs-with-page-group-included-in-a-single-page-warning
\pdfsuppresswarningpagegroup=1

\author{\huge{Daniel Yu}}
\date{September 9, 2024}

\pdfsuppresswarningpagegroup=1
\begin{document}

\title{\Huge{Machine Learning and Statistical Learning}\\ Lecture 1}
\maketitle
\newpage% or \cleardoublepage
% \pdfbookmark[<level>]{<title>}{<dest>}
\tableofcontents
\pagebreak
  \section{Review of Machine Learning Basics}

  \textbf{Statistical Learning}
  \begin{itemize}
    \item Input (aka feature/predictor): $\vec{x} = (\vec{x_1}, \ldots, \vec{x_d}) = \begin{pmatrix} x_1\\ \vdots\\ x_d \end{pmatrix} \in X = \mathbb{R}^d$
    \item Output (label): $y \in C \subseteq \mathbb{R}$
  \end{itemize}

  Note: here y is an outcome we wish to predict from $\vec{x}$.
  
  \begin{remark}
    Measuring a pair $(\vec{x}, y)$ is a sample of pair of random variables \textit{R.V.} $(\vec{X}, Y)$ with underlying
    distributions.
  \end{remark}

  We want to learn relation between $\vec{x}$ and $y$ from training set $D = \{(\vec{x}^{i}, y^{i}), i = 1, \ldots, n \}$

  Statistical Learning $\approx$ Machine Learning only that Statistical learning is less focused on algorithms than ML.

  \begin{definition}
    A learning algorithm is a function that takes training set $D = \{(\vec{x}^{i}, y^{i}), i = 1, \ldots, n \}$ as input
    and has as output a prediction $y = \hat{f}(\vec{x})$ that for every $\vec{x}$ predicts y value.
  \end{definition}

  \begin{definition}
    A probability model is a joint probability distribution $P$ of $\vec{x} \in X$ and $y \in C$ on the pair $(\vec{X}, Y) \in X \times C = \mathbb{R}^d \times C$ 
    of a random vector $\vec{X} \subseteq \mathbb{R}^d$ and random variable $Y \in C$.
    
  \end{definition}

  \begin{definition}
    A function model is a single function $f: X \to C$ or a class $F$ of such functions where one function is assumed to give a good 
    prediction $y \in C$ from $\vec{x} \in X$.
  \end{definition}
  
  \begin{enumerate}
    \item Supervised Learning (regression and classification)
    \item Unsupervised Learning
  \end{enumerate}

  \begin{remark}
    One example of Supervised Learning is in generative models: $(y, \vec{x}) \to P(y | \vec{x}) P(\vec{x})$ \textit{note that this is 
    just the chain rule in probability where P(x,y) = P(y|x) P(x)} where $P(y | \vec{x})$ is the 
    conditional probability that a set of inputs $\vec{x}$ produces $y$ and $P(\vec{x})$ represents the marginal probability of the
    input features $\vec{x}$ are in a dataset (i.e. how likely different feature combinations are as each $x_i$ is a feature). \\

    However, in discriminant models, the above does not apply as they focus solely on learning $P(y | \vec{x})$ the conditional
    probability. \\
  
    One example of Unsupervised Learning is probablistic modeling: $(\vec{x}) \to P(\vec(x))$, so we are simply estimating the probability 
    distribution of underlying data, using each datapoint $\vec{x}$ to improve our estimation of the probability distribution $P(\vec{x})$.
  \end{remark}
  
  \subsection{Supervised Learning} 
  \begin{enumerate}
    \item regression - $y \in R \text{ is continuous, quantitative}$ 
    \item classification - $y \in {1, \ldots , K} \text{ is discrete, qualitative}$
  \end{enumerate}
  In a statistical learning (theoretical standpoint) they are the same but in a ML algorithmic standpoint they are different!
  
  \begin{remark}
    A probability model has fixed joint prob distribution: $(\vec{X}, Y) \sim P(\vec{x}, y)$. For some measurable set $A \subset \mathbb{R} \times C$,
    the probability that $(\vec{X}, Y) \in A$ is:
    \[
      P(A) = P((\vec{X}, Y) \in A)
    .\] 
  \end{remark}

  TODO: Missing stuff -- parametric vs non-parametric

  \subsection{Regression Model Assumptions}
  \[
   y^{(i)} = h(\vec{x}^{(i)}) + \epsilon_i
  .\] 
  Here $h(\vec{x})$ depends on some finite or infinite collection of parameters $\vec{\theta}$
  
  \textit{We assume that errors are random variables of the form $\epsilon_i \sim \text{ Normal }(0, \sigma^2)$}. 
  \begin{remark}
    This is a strong assumption!
  \end{remark}


\begin{proof}
    \begin{align}
        E(Y | \vec{X} = \vec{x}) &= E(h(\vec{X}) + \epsilon | \vec{X} = \vec{x}) \text{ since $h(\vec{X})$ and $\epsilon$ are independent} \\
                                 &= E(h(\vec{X}) | \vec{X} = \vec{x}) + E(\epsilon | \vec{X} = \vec{x}) \\
                                 &= h(\vec{x}) + 0 \\
                                 &= h(\vec{x})
    \end{align}
\end{proof}

\begin{remark}
  Proof above is only true for the assumption that $\epsilon$ is random white noise.
  
\end{remark}

\subsection{Expected Prediction Error}
  Assume we have an algorithm for estimating $h(\vec{x}^0)$ for test point $\vec{x}^0$. The estimator is:
  \[
    \vec{y}^0 = \hat{h}(\vec{x}^0) \approx h(\vec{x}^0)
  .\] 
  
  \begin{definition}
    Our loss function is then defined as:
    \[
      L(Y, h(\vec{X}))
    .\] 
  \end{definition}
  
  \textbf{Loss Function Types}
  \begin{itemize}
    \item $L(Y, h(\vec{X})) = (Y - h(\vec{X}))^2$ square loss error
    \item $L(Y, h(\vec{X})) = |Y - h(\vec{X})|$ absolute loss error
  \end{itemize}
  
  \begin{definition}
  Expected Prediction Error (Expected Value of Error)
    \[
      EPE(h) = E_{Y, \vec{X}} [L(Y, h(\vec{X}))] = E_{\vec{X}} E_{Y | \vec{X}} [L(Y, h(\vec{X}) | \vec{X}]
    .\]
    We want to minimize pointwise:
  \[
    \hat{h}(\vec{x}) = argmin_c E_{Y | X} [L(Y, c) | \vec{X} \to \vec{x} ] 
  .\]
  where $c = h(\vec{X})$
  \end{definition}

  \begin{remark}
    In Theory we want prediction error for all future values but in practice we only have to the test error.
  \end{remark}

  \subsubsection{Squared Error Loss}
  Let  $L(Y, h(\vec{X})) = (Y - h(\vec{X}))^2$ square loss error
  \[
    \hat{h}(\vec{x}) = argmin_c E_{Y | \vec{X}} [(Y - c)^2 | \vec{X} = \vec{x}]
  .\] 
  \[
    = argmin_c E_{Y | \vec{X}} [Y^2 - 2cY + c^2 | \vec{X} = \vec{x}] 
  .\]
  Note:
  \[
    E[Y^2 - 2cY + c^2 | \vec{X} = \vec{x}] = E[Y]^2 - 2cE[Y] + c^2  
  .\] 
  \[
  \frac{d}{dc}[E[Y]^2 - 2cE[Y] + c^2] = -2E[Y] + 2c = 0
  .\] 
  \[
    c = E[Y] = E[Y | \vec{X} = \vec{x}]
  .\]
  Logically this makes sense because the minimizer of the squared loss function when $\hat{h(x)}=E[Y|\vec{X} = \vec{x}]$ 
  is the mean (expected value) of Y given X = x which captures the squared loss function's central
  tendency as it squares distance. 

  \subsubsection{Absolute Error}
  Let $L(Y, h(\vec{X})) = |Y - h(\vec{X})|$ absolute loss error:
  \[
    \hat{h}(\vec{x}) = argmin_c E_{Y | \vec{X}} [ | Y - c| | \vec{X} = \vec{x}] 
  .\] 
  Expand the $|Y - c|$ piecewise:
  \[
    E[|Y - c|] = \int_{-\infty}^c (c-y) p(y) dy + \int_c^{\infty} (y-c) p(y) dy
  .\] 
  Take the derivative to minimize the expected value:
  \[
    \frac{d}{dc} ( E[|Y - c|]) = \frac{d}{dc} [\int_{-\infty}^c (c-y) p(y) dy + \int_c^{\infty} (y-c) p(y) dy] = 0
  .\]
  \[
    = \int_{-\infty}^c p(y) dy - \int_c^{\infty} p(y) dy = 0 
  .\] 
  \[
  \int_{-\infty}^c p(y) dy = \int_c^{\infty} p(y) dy
  .\] 
\textbf{This implies that to minimize the Expected Value, c is the median of the distribution of Y, 
because the cumulative probability mass to the left of c must be equal to the cumulative 
probability mass to the right of c.}
\[
  \hat{h}(\vec{x}) = median(Y | \vec{X} = \vec{x})
.\] 
\begin{remark}
  Absolute Value is not differentiable (although it is continuous) so greedy descent methods can't be used.
\end{remark}

\subsection{Classification}
Consider the binary case where $Y = 0,1$ for class 1 and class 2 respectively:
\begin{align*}
  \hat{h}(\vec{x}) &= E[Y|\vec{X} = \vec{x}] \\ 
                   &= P(\text{class 1} | \vec{X} = \vec{x}) \cdot 1 + P(\text{class} 2 | \vec{X} = \vec{x}) \cdot 0 \\
                   &= P(Y = 1 | \vec{X} = \vec{x}) \cdot 1 + P(Y = 0 | \vec{X} = \vec{x}) \cdot 0 \\
                   &= P(Y = 1 | \vec{X} = \vec{x}) \in \R
.\end{align*}

In the multiclass case: $Y = 1, \ldots, K$:
\begin{align*}
  \hat{\vec{x}} &= \arg\min_{f(x)} E_{Y|\vec{X}} [L(Y, h(X))|\vec{X} = \vec{x}] \\
                &= \arg\min_k E_{Y | \vec{X}}[L(Y, k) | \vec{X} = \vec{x}] \\
                &= \arg\min_k \sum_{y=1}^K L(y,k) P(Y = y | \vec{X} = \vec{x}) 
.\end{align*}

# TODO classifiers stuff

  \section{Bias-Variance Tradeoff}
  
  Assuming, iid $\epsilon_i$ as noise:
  \begin{definition}
    The Expected Prediction (in practice test) Error (EPE) for a fixed test point $\vec{x}^0$ is the average over both 
      $y^0$ and the entire training set $D$. 

      \[
        EPE(\vec{x}^0) = E_{y^0, D} [L(y^0, \hat{h}(\vec{x}^0))]
      .\] 
      We want in theory \textbf{full expected prediction error}, meaning we average over all training sets:
      $D = \{(\vec{x}^i, \vec{y}^i, i = 1,\ldots,n \}$ where each $D_i$ is one possible split of train/test data and
      over all possible values $Y=y^0$ at $\vec{x}^0$.
  \end{definition}

  \begin{remark}
    D is a random variable whose distribution is all possible train/test splits. 
    $y^0$ is a random variable representing the true value of the outcome at $\vec{x}^0$ but inherents the random "noise" 
    in the data generation process, leading to different $\hat{y}^0$ being observed.
  \end{remark}

  \subsection{Mathematical Decomposition of Expected Prediction Error at fixed $\vec{x}^{(0)}$}
  \begin{note}
    This is for the case where the loss function is the \textbf{Mean Squared Error} loss function.
  \end{note}
  \begin{align*}
    EPE(\vec{x}^0) &= E_{y^0, D} [(y^0 - \hat{y}^0)^2] \\
                   &= E_{y^0} E_D [(y^0 - \hat{y}^0)^2] 
  \end{align*}
  
  \[
  =E_{y^0} E_D \{[ \textcolor{teal}{\left( y^0 - E_{y^0} [y^0] \right)} +  (E_{y^0} [y^0] - E_D[\hat{y}^0]) + \textcolor{red}{(E_D[\hat{y}^0] - \hat{y}^0)}]^2 \}
  .\] 
    \[
      = E_{y^0} \textcolor{teal}{[y^0 - E_{y^0} [y^0]]^2} + [E_{y^0} [y^0] - E_D[\hat{y}^0]]^2 + \textcolor{red}{
      E_D[E_D[\hat{y}^0] - \hat{y}^0]^2} 
  .\] 
  where $y_0 = h(\vec{x}^0) + \epsilon$ and $\hat{y}^0 = \hat{h}\left( \vec{x} \right)$ 
  \begin{note}
    Note all the cross terms dissappear from the expression about which is why we can eliminate them:
    \begin{align*}
      & E_{y^0}E_D [ (Y^0 - E_{y^0} [y^0]) \left( E_D [\hat{y}^0] - \hat{y}^0 \right) ] \text{  the first expression does not depend on D} \\
      &= E_{y^0} [\left( y^0 - E_{y^0} [y^0] \right) E_D \left( E_D [\hat{y}^0] - \hat{y}^0 \right)] \\
      &= 0
    .\end{align*}
    The last step is possible because $y^0, \hat{y}^0$ are independent since $\epsilon_i$ are independent and 
    $\hat{y}^0$ depends on D, and by probability $E[X \cdot Y] = E[X] \cdot E[Y]$ when $X,Y$ are independent.
  \end{note}
  \begin{enumerate}
    \item $\textcolor{teal}{\left( y^0 - E_{y^0} [y^0] \right)}$ -- can be thought of as the \textbf{noise} of the model that distorts the true value of y 
      from its expected value (coming from the $\epsilon$ assumed to be random gaussian noise)
    \item $E_{y^0} [y^0] - E_D[\hat{y}^0]$ -- can be thought of as the \textbf{bias}, difference between true expected output $E_{y^0} [y^0]$ 
      and the model's prediction $ E_D[\hat{y}^0]$.
    \item $ \textcolor{red}{(E_D[\hat{y}^0] - \hat{y}^0)}$ -- can be thought of as \textbf{variance}, variability of model's prediction $\hat{y}^0$ from the 
      actual expected value $E_D[\hat{y}^0]$ (recall that D is the set of all possible training sets from different train/test splits where our 
      model takes only train set).
  \end{enumerate}
  
  We can rewrite the formula since $var = \frac{1}{N} \sum_{i=1}^N (x_i - \mu)^2$:

\begin{align*}
    &= var_{y^0} (y^0) + \left[E_{y^0} [y^0] - E_D [\hat{y}^0]\right]^2 + var_D (\hat{y}^0) \\
    &= \sigma^2 + \left[h(x^0) - E_D[\hat{y}^0]\right]^2 + var_D(\hat{y}^0) \quad \text{from } N(\mu = h(x^0), \sigma) \\
    &= \sigma^2 + \left[bias\right]^2 (\hat{y}^0) + var_D (\hat{y}^0) \\
    &= \text{unavoidable error } + \text{bias}^2 + var_D(\hat{y}^0)
\end{align*}
  where bias is difference between true and predictec
  
  # TODO knn and LINEAR model notes
  \section{KNN}

  \section{Linear Regression - OLS}

  \section{Ridge Regression}
  Motivation: no longer working with unbiased estimators, trade off some biass to large decrease in variance.
  
  \begin{definition}
    Assume data $D = (X, \vec{y})$ is centered, the mean $E(X) = \vec{0}$ and $E(\vec{y}) = 0$, so 
    \[
      h(\vec{x}) = \vec{\theta}^T \vec{x} = \theta_1 x_1 + \ldots + \theta_d x_d
    .\] 
    \textbf{Ridge Regression} minimize cost function:
    \begin{align*}
      J(\vec{\theta}) &= \sum_{i=1}^n (y^i - h_\theta (\vec{x}^i))^2 + \lambda \sum_{j=1}^d \theta_j^2
                      &= (X \vec{\theta} - \vec{y})^T (X \vec{\theta} - \vec{y}) + \lambda \vec{\theta}^T \vec{\theta}
    .\end{align*}
    Taking the first derivative:
    \begin{align*}
      \frac{d}{d \vec{\theta}} J &= \ldots \\
                                 &= \ldots
    .\end{align*}
    \[
      \vec{\theta} = (X^T X + \lambda I)^{-1} X^T \vec{y}
    .\] 
  \end{definition}
  
  # TODO More examples:
  # TODO High Dimension Data and Reduction

  \section{Readings}
  1. https://mlu-explain.github.io/bias-variance/
\end{document}
