\documentclass[a4paper]{article}
\usepackage[a4paper, margin=1in]{geometry}
% Some basic packages
\usepackage[utf8]{inputenc}
\usepackage[T1]{fontenc}
\usepackage{textcomp}
\usepackage[dutch]{babel}
\usepackage{url}
\usepackage{graphicx}
\usepackage{float}
\usepackage{booktabs}
\usepackage{enumitem}

\pdfminorversion=7

% Don't indent paragraphs, leave some space between them
\usepackage{parskip}

% Hide page number when page is empty
\usepackage{emptypage}
\usepackage{subcaption}
\usepackage{multicol}
\usepackage{xcolor}

% Other font I sometimes use.
% \usepackage{cmbright}

% Math stuff
\usepackage{amsmath, amsfonts, mathtools, amsthm, amssymb}
% Fancy script capitals
\usepackage{mathrsfs}
\usepackage{cancel}
% Bold math
\usepackage{bm}
% Some shortcuts
\newcommand\N{\ensuremath{\mathbb{N}}}
\newcommand\R{\ensuremath{\mathbb{R}}}
\newcommand\Z{\ensuremath{\mathbb{Z}}}
\renewcommand\O{\ensuremath{\emptyset}}
\newcommand\Q{\ensuremath{\mathbb{Q}}}
\newcommand\C{\ensuremath{\mathbb{C}}}

% Easily typeset systems of equations (French package)
\usepackage{systeme}

% Put x \to \infty below \lim
\let\svlim\lim\def\lim{\svlim\limits}

%Make implies and impliedby shorter
\let\implies\Rightarrow
\let\impliedby\Leftarrow
\let\iff\Leftrightarrow
\let\epsilon\varepsilon

% Add \contra symbol to denote contradiction
\usepackage{stmaryrd} % for \lightning
\newcommand\contra{\scalebox{1.5}{$\lightning$}}

% \let\phi\varphi

% Command for short corrections
% Usage: 1+1=\correct{3}{2}

\definecolor{correct}{HTML}{009900}
\newcommand\correct[2]{\ensuremath{\:}{\color{red}{#1}}\ensuremath{\to }{\color{correct}{#2}}\ensuremath{\:}}
\newcommand\green[1]{{\color{correct}{#1}}}

% horizontal rule
\newcommand\hr{
    \noindent\rule[0.5ex]{\linewidth}{0.5pt}
}

% hide parts
\newcommand\hide[1]{}

% si unitx
\usepackage{siunitx}
\sisetup{locale = FR}

% Environments
\makeatother
% For box around Definition, Theorem, \ldots
\usepackage{mdframed}
\mdfsetup{skipabove=1em,skipbelow=0em}
\theoremstyle{definition}
\newmdtheoremenv[nobreak=true]{definitie}{Definitie}
\newmdtheoremenv[nobreak=true]{eigenschap}{Eigenschap}
\newmdtheoremenv[nobreak=true]{gevolg}{Gevolg}
\newmdtheoremenv[nobreak=true]{lemma}{Lemma}
\newmdtheoremenv[nobreak=true]{propositie}{Propositie}
\newmdtheoremenv[nobreak=true]{stelling}{Stelling}
\newmdtheoremenv[nobreak=true]{wet}{Wet}
\newmdtheoremenv[nobreak=true]{postulaat}{Postulaat}
\newmdtheoremenv{conclusie}{Conclusie}
\newmdtheoremenv{toemaatje}{Toemaatje}
\newmdtheoremenv{vermoeden}{Vermoeden}
\newtheorem*{herhaling}{Herhaling}
\newtheorem*{intermezzo}{Intermezzo}
\newtheorem*{notatie}{Notatie}
\newtheorem*{observatie}{Observatie}
\newtheorem*{oef}{Oefening}
\newtheorem*{opmerking}{Opmerking}
\newtheorem*{praktisch}{Praktisch}
\newtheorem*{probleem}{Probleem}
\newtheorem*{terminologie}{Terminologie}
\newtheorem*{toepassing}{Toepassing}
\newtheorem*{uovt}{UOVT}
\newtheorem*{vb}{Voorbeeld}
\newtheorem*{vraag}{Vraag}

\newmdtheoremenv[nobreak=true]{definition}{Definition}
\newtheorem*{eg}{Example}
\newtheorem*{notation}{Notation}
\newtheorem*{previouslyseen}{As previously seen}
\newtheorem*{remark}{Remark}
\newtheorem*{note}{Note}
\newtheorem*{problem}{Problem}
\newtheorem*{observe}{Observe}
\newtheorem*{property}{Property}
\newtheorem*{intuition}{Intuition}
\newmdtheoremenv[nobreak=true]{prop}{Proposition}
\newmdtheoremenv[nobreak=true]{theorem}{Theorem}
\newmdtheoremenv[nobreak=true]{corollary}{Corollary}

% End example and intermezzo environments with a small diamond (just like proof
% environments end with a small square)
\usepackage{etoolbox}
\AtEndEnvironment{vb}{\null\hfill$\diamond$}%
\AtEndEnvironment{intermezzo}{\null\hfill$\diamond$}%
% \AtEndEnvironment{opmerking}{\null\hfill$\diamond$}%

% Fix some spacing
% http://tex.stackexchange.com/questions/22119/how-can-i-change-the-spacing-before-theorems-with-amsthm
\makeatletter
\def\thm@space@setup{%
  \thm@preskip=\parskip \thm@postskip=0pt
}


% Exercise 
% Usage:
% \oefening{5}
% \suboefening{1}
% \suboefening{2}
% \suboefening{3}
% gives
% Oefening 5
%   Oefening 5.1
%   Oefening 5.2
%   Oefening 5.3
\newcommand{\oefening}[1]{%
    \def\@oefening{#1}%
    \subsection*{Oefening #1}
}

\newcommand{\suboefening}[1]{%
    \subsubsection*{Oefening \@oefening.#1}
}


% \lecture starts a new lecture (les in dutch)
%
% Usage:
% \lecture{1}{di 12 feb 2019 16:00}{Inleiding}
%
% This adds a section heading with the number / title of the lecture and a
% margin paragraph with the date.

% I use \dateparts here to hide the year (2019). This way, I can easily parse
% the date of each lecture unambiguously while still having a human-friendly
% short format printed to the pdf.

\usepackage{xifthen}
\def\testdateparts#1{\dateparts#1\relax}
\def\dateparts#1 #2 #3 #4 #5\relax{
    \marginpar{\small\textsf{\mbox{#1 #2 #3 #5}}}
}

\def\@lecture{}%
\newcommand{\lecture}[3]{
    \ifthenelse{\isempty{#3}}{%
        \def\@lecture{Lecture #1}%
    }{%
        \def\@lecture{Lecture #1: #3}%
    }%
    \subsection*{\@lecture}
    \marginpar{\small\textsf{\mbox{#2}}}
}



% These are the fancy headers
\usepackage{fancyhdr}
\pagestyle{fancy}

% LE: left even
% RO: right odd
% CE, CO: center even, center odd
% My name for when I print my lecture notes to use for an open book exam.
% \fancyhead[LE,RO]{Gilles Castel}

\fancyhead[RO,LE]{\@lecture} % Right odd,  Left even
\fancyhead[RE,LO]{}          % Right even, Left odd

\fancyfoot[RO,LE]{\thepage}  % Right odd,  Left even
\fancyfoot[RE,LO]{}          % Right even, Left odd
\fancyfoot[C]{\leftmark}     % Center

\makeatother




% Todonotes and inline notes in fancy boxes
\usepackage{todonotes}
\usepackage{tcolorbox}

% Make boxes breakable
\tcbuselibrary{breakable}

% Verbetering is correction in Dutch
% Usage: 
% \begin{verbetering}
%     Lorem ipsum dolor sit amet, consetetur sadipscing elitr, sed diam nonumy eirmod
%     tempor invidunt ut labore et dolore magna aliquyam erat, sed diam voluptua. At
%     vero eos et accusam et justo duo dolores et ea rebum. Stet clita kasd gubergren,
%     no sea takimata sanctus est Lorem ipsum dolor sit amet.
% \end{verbetering}
\newenvironment{verbetering}{\begin{tcolorbox}[
    arc=0mm,
    colback=white,
    colframe=green!60!black,
    title=Opmerking,
    fonttitle=\sffamily,
    breakable
]}{\end{tcolorbox}}

% Noot is note in Dutch. Same as 'verbetering' but color of box is different
\newenvironment{noot}[1]{\begin{tcolorbox}[
    arc=0mm,
    colback=white,
    colframe=white!60!black,
    title=#1,
    fonttitle=\sffamily,
    breakable
]}{\end{tcolorbox}}




% Figure support as explained in my blog post.
\usepackage{import}
\usepackage{xifthen}
\usepackage{pdfpages}
\usepackage{transparent}
\newcommand{\incfig}[1]{%
    \def\svgwidth{\columnwidth}
    \import{./figures/}{#1.pdf_tex}
}

% Fix some stuff
% %http://tex.stackexchange.com/questions/76273/multiple-pdfs-with-page-group-included-in-a-single-page-warning
\pdfsuppresswarningpagegroup=1

\title{\Huge{Probability 1}}
\author{\huge{Daniel Yu}}
\date{November 19, 2024}

\pdfsuppresswarningpagegroup=1

\begin{document}
\maketitle
\newpage% or \cleardoublepage
% \pdfbookmark[<level>]{<title>}{<dest>}
\tableofcontents
\pagebreak
\section{Poisson Processes}

\begin{definition}
  A stochastic process $\{N(t)\}_{t \in [0, \infty)} $ is a counting process if
\begin{enumerate}
  \item $N(t) = \{01,2,3,\ldots\} \forall t > 0 $ and
  \item $N(s) \leq N(t) \forall s \leq t$
\end{enumerate}
$N(t)$ counts the number of events that occured between time  $0$ and time  $t$.
\end{definition}


\begin{note}
  See the Appendix for the definition for asympotic notation for $O(x)$. We are using a definition that is no the traditional definition. 
\end{note}
\begin{definition}

A counting process is \textbf{simple} if $\forall t \geq 0,$
 \[
   P[N(t + h) - N(t) \geq 2] = O(h)
.\] 
meaning the probability of more than 1 arrival in the interval $[t+h, t]$ goes to 0 fairly quickly (at rate $O(h)$ )
  
\end{definition}

\noindent\hrulefill

We wish to study the generalization of \textbf{time-homogenous markov chains}. Let's make the assumptions that:
\begin{enumerate}
  \item \textbf{the distribution of (N(t+s) - N(t)) the number of arrivals that occured between t to t+s does not depend on t!}
  \item \textbf{for any $t \leq s$ the RVs N(t), N(s) - N(t) are independent} this is known as the independent increament property. This is saying that the number of arrivials in disjoint time intervals is independent. 
\end{enumerate}

\begin{definition}
  Any simple counting process that satisfies (1), (2) is a time-homogenous poisson process on $\R$
\end{definition}

To determine the process, we need only one parameter $\lambda > 0$. Consider for the interval, $[0,h]$:
 \[
   P[N(h) - N(0) = 0] = 1 - \lambda h + O(h)
.\] 
using the fact that $P[N(h) - N(0) = 1] = \lambda h $ and  $P[N(h) - N(0) \geq 2] = O(h)$

\begin{theorem}
  Let $N_{\lambda}(t)$ be the unique process which satisfies all these properties above:
  \[
    P[N_{\lambda} (t) = k] = \frac{e^{-\lambda t} (\lambda t)^{k}}{k!}
  .\] 
  i.e. $N_{\lambda}(t) \sim Poisson(\lambda t)$

\end{theorem}
  \noindent\hrulefill

  \begin{proof}
    Let $P_m (t) = P[N_{\lambda}(t) = m]$. 
    \begin{align*}
      P_m (t+ h) &= \sum_{k = 0}^{m} P[N_{\lambda} (1+ h) = m \mid N_{\lambda}(t) = k] \cdot P[N_{\lambda} (t) = k] \\
                 &= \sum_{k = 0}^{m} P[N_{\lambda} (1+ h) - N_{\lambda} (t) + N_{\lambda} (t) = m \mid N_{\lambda}(t) = k] \cdot P[N_{\lambda} (t) = k]   \\
                 &= \sum_{k = 0}^{m} P[N_{\lambda} (1+ h) - N_{\lambda } (t) = m  - k \mid N_{\lambda}(t) = k] \cdot P[N_{\lambda} (t) = k]  \\
                 &\text{ Use the property that disjoint intervals are independent}  \\
                 &= \sum_{k = 0}^{m} P[N_{\lambda} (1+ h) - N_{\lambda } (t) = m  - k] \cdot P[N_{\lambda} (t) = k]   \\
                 &= \text{using time homogenity} \\
                 &=  \sum_{k=0}^{m} P[N_{\lambda} (h) - N_{0} (t) = m  - k] \cdot P[N_{\lambda} (t) = k]   \\
                 &=  P[N_{\lambda}(h) - N_{\lambda}(0) = 0 ] \cdot P_{m}(t) + P[N_{\lambda}(h) - N_{\lambda}(0) = 0 ] \cdot P_{m-1}(t) + \\
                 &\sum_{k=0}^{m-2} P[N_{\lambda} (h) - N_{0} (t) = m  - k] \cdot P_k (t)  \\
                 &= (1- \lambda h + O(h)) P_{m} (t) + (\lambda h + O(h)) P_{m-1} (t)  + \sum_{k=0}^{m-2} O(h) \cdot P_{k} (t) \\
                 &\text{ we can group and simplify by asymptotics since $cO(x) = O(x)$ } \\
                 &= P_{m}(t) + \lambda h \left(   -P_{m}(t) + P_{m}(t) \right) + O(h) \\
     \frac{P_m (t + h) - P_m (t)}{h} &= \lambda (- P_{m} (t) + P_m (t)) + \frac{O(h)}{h}  
    .\end{align*}
    Let's take the limit as $h \to 0$:
     \[
    \frac{d P_{m} (t)}{dt} = - \lambda P_{m} (t) + \lambda P_{m} (t), m \geq 1
    .\] 
    Repeating this for $m = 0$,m 
     \[
    \frac{d P_0 (t)}{dt} = - \lambda P_0 (t)
    .\]
    What is $P_0 (t)$?
     \begin{align*}
       P_0' &= -\lambda P_0 \\
      \Rightarrow \int_{0}^{t} \frac{P_{0}' (s)}{P_{0} (s)} ds \\
            &= \int_{0}^{t}  -\lambda ds \\
            &=  \ln(P_0(t)) - \ln(P_0(0)) = - \lambda t \\
       P_0(t) &= e^{-\lambda t} 
    .\end{align*}
    For $m=1$:
    missing notes.\\

     $e^{\lambda t} P_{1} (t) - P_{1} \left(0 \right)  = \lambda t \Rightarrow P_1 (t) = \lambda t \cdot e^{ -\lambda t}$ \\
     It is possible to solve the above by induction and get the closed form formula:
     \[
       P_{m}(t) \frac{e^{-\lambda t} (\lambda t)^{m}}{m!}
     .\] 

     \noindent\hrulefill

     Method 2: Using Generating Functions\\
     \[
     G(s,t) = \sum_{m=0}^{\infty} P_{m} (t) \cdot S^{m}
     .\] 
     Then, compute the derivative 
     \begin{align*}
       \frac{d G(s,t)}{dt} &= \sum_{m=0}^{\infty} P_m (t)' S^{m} \\
                           &= P_0'(t) + \sum_{m \geq 1} P_{m}'(t) S^{m}  \\
                           &= - \lambda P_{0}(t) + \sum_{m \geq 1} [ - \lambda P_{m} (t) + \lambda P_{m+1} (t)] S^{m} \\ 
                           &= \text{missing text}  \\
                           &= \lambda G(s,t) + \lambda s G(s,t) \\
     .\end{align*}
     Then,
     \[
     \frac{d G(s,t)}{dt} = \lambda G(s,t) (S-1)
     .\] 
     Divide by $G(s,t)$ and integrat w.r.t.  $t$,
      \begin{align*}
        \ln[G(s,t)] - \ln[G(s,t)] = \lambda s-1) \cdot t
     .\end{align*}

     missing text \\

     \begin{align*}
       \sum_{m = 0}^{\infty} P_m (t) \cdot S^{m} &= e^{-\lambda t} \cdot e^{\lambda s t} \\
       &= e^{-\lambda t} \sum_{m=0}^{\infty} \frac{\left( \lambda s t \right)^{m} }{m!} \\
      &\text{canceling both sides and removing the summations} \\
      P_m (t) &= \frac{e^{-\lambda t} (\lambda t)^{m}}{m!}  
     .\end{align*}
  \end{proof}

\begin{note}
  The continuous version of this is the Brownian motion
\end{note}

\noindent\hrulefill

Let's go through an example. Let $N(t)$ be a poisson point process of increment 1. What is
 \[
   P[N(1) = 1, N(2) = 1, N(3) = 2, N(4) = 2, N(5) = 3] 
.\]?\\

\textbf{Method 1}\\
We can use the new arrivals per interval:
\[
  = P[\text{1 arrival [0,1], 0 arrival in [1,2], one arrival in [2,3], zero arrivasl [3,4], 1 arrova; [4,5]}]
.\] 

By tyhe time independent increment property:
\begin{align*}
  P[\text{ events above}] &= P[\text{1 arrival [0,1]}]P[\text{0 arrival [1,2]}] P[\text{1 arrival [2,3]} P[\text{0 arrival [3,4]}] P[\text{1 arrival [4,5]}] \\
                          &= P[Poisson(1) = 1]^{3} P[Poisson(1) = 0]^{2} \\
                          &= \frac{1}{e^{5}} \\
.\end{align*}

\textbf{Method 2}\\
Using joints intelligently
\begin{align*}
  P[N(1) = 1, N(2) = 1, N(3) = 2, N(4) = 2, N(5) = 3] &=
P[N(1) - N(0) = 1] \cdot P[N(2) - N(1) = 0] \cdot P[N(3) - N(2) = 1] \\
&= \cdot P[N(4) - N(3) = 0] \cdot P[N(5) - N(4) = 1]. \\
.\end{align*}
\[
P[X = k] = \frac{\mu^k e^{-\mu}}{k!}.
\]
For each interval $[a, b]$, the mean is $\mu = \lambda (b-a)$. Since $\lambda = 1$ and each interval is of length 1, we have $\mu = 1$ for each increment:
\begin{align*}
  P[N(1) - N(0) = 1] &= P[\text{Poisson}(1) = 1] = \frac{1^1 e^{-1}}{1!} = \frac{1}{e}, \\
  P[N(2) - N(1) = 0] &= P[\text{Poisson}(1) = 0] = \frac{1^0 e^{-1}}{0!} = \frac{1}{e}, \\
  P[N(3) - N(2) = 1] &= P[\text{Poisson}(1) = 1] = \frac{1}{e}, \\
  P[N(4) - N(3) = 0] &= P[\text{Poisson}(1) = 0] = \frac{1}{e}, \\
  P[N(5) - N(4) = 1] &= P[\text{Poisson}(1) = 1] = \frac{1}{e}.
\end{align*}

\textbf{Step 4: Combine Probabilities}\\
The total joint probability is:
\[
P[N(1) = 1, N(2) = 1, N(3) = 2, N(4) = 2, N(5) = 3] =
\frac{1}{e} \cdot \frac{1}{e} \cdot \frac{1}{e} \cdot \frac{1}{e} \cdot \frac{1}{e} = \frac{1}{e^5}.
\]

\textbf{Final Result:}\\
Both methods yield the same result:
\[
P[N(1) = 1, N(2) = 1, N(3) = 2, N(4) = 2, N(5) = 3] = \frac{1}{e^5}.
\]
\section{Appendix: Asysmpotics}
\begin{definition}
  $f(x) = O(x)$ if  $\lim_{x \to 0} \frac{f(x)}{x} = 0$

  \begin{note}
    This not equivalent to the usual definition of big-O notation i.e. that $f(x)$ is  $O(g(x))$ if $f(x) \leq cg(x)$. 
    \begin{enumerate}
      \item consider $x^{2} \Rightarrow \lim_{x \to 0} \frac{x^{2}}{x} = \lim_{x \to 0} x = 0$ but is not $O(x)$ in the traditional definition.
      \item Consider $x \Rightarrow \lim_{x \to 0} \frac{x}{x} = 1 \neq 0$ but is $O(x)$ in the traditional definion.
    \end{enumerate}  
  \end{note}
\end{definition}

\noindent\hrulefill 

In this notation if $f(x)$ is twice differentiable at $x=0$, then we can use the taylor series expansion with the error term $r(x) = O(x^{2})$:
 \[
f(x) = f(0) + f'(0) + \frac{f''(0)}{2} x^{2} + O(x^{2})
.\] 

\[
  (\frac{1}{x} + 1 + O(1)) \cdot \left( 1 + O(x) \right)  = \left( \frac{2}{x} + 2 \right) + O(1) 
.\] 

some missing text here
$\frac{f(x)}{x} + f(x) \to 0$

\end{document} 
