\documentclass[a4paper]{article}
\usepackage[a4paper, margin=1in]{geometry}
% Some basic packages
\usepackage[utf8]{inputenc}
\usepackage[T1]{fontenc}
\usepackage{textcomp}
\usepackage[dutch]{babel}
\usepackage{url}
\usepackage{graphicx}
\usepackage{float}
\usepackage{booktabs}
\usepackage{enumitem}

\pdfminorversion=7

% Don't indent paragraphs, leave some space between them
\usepackage{parskip}

% Hide page number when page is empty
\usepackage{emptypage}
\usepackage{subcaption}
\usepackage{multicol}
\usepackage{xcolor}

% Other font I sometimes use.
% \usepackage{cmbright}

% Math stuff
\usepackage{amsmath, amsfonts, mathtools, amsthm, amssymb}
% Fancy script capitals
\usepackage{mathrsfs}
\usepackage{cancel}
% Bold math
\usepackage{bm}
% Some shortcuts
\newcommand\N{\ensuremath{\mathbb{N}}}
\newcommand\R{\ensuremath{\mathbb{R}}}
\newcommand\Z{\ensuremath{\mathbb{Z}}}
\renewcommand\O{\ensuremath{\emptyset}}
\newcommand\Q{\ensuremath{\mathbb{Q}}}
\newcommand\C{\ensuremath{\mathbb{C}}}

% Easily typeset systems of equations (French package)
\usepackage{systeme}

% Put x \to \infty below \lim
\let\svlim\lim\def\lim{\svlim\limits}

%Make implies and impliedby shorter
\let\implies\Rightarrow
\let\impliedby\Leftarrow
\let\iff\Leftrightarrow
\let\epsilon\varepsilon

% Add \contra symbol to denote contradiction
\usepackage{stmaryrd} % for \lightning
\newcommand\contra{\scalebox{1.5}{$\lightning$}}

% \let\phi\varphi

% Command for short corrections
% Usage: 1+1=\correct{3}{2}

\definecolor{correct}{HTML}{009900}
\newcommand\correct[2]{\ensuremath{\:}{\color{red}{#1}}\ensuremath{\to }{\color{correct}{#2}}\ensuremath{\:}}
\newcommand\green[1]{{\color{correct}{#1}}}

% horizontal rule
\newcommand\hr{
    \noindent\rule[0.5ex]{\linewidth}{0.5pt}
}

% hide parts
\newcommand\hide[1]{}

% si unitx
\usepackage{siunitx}
\sisetup{locale = FR}

% Environments
\makeatother
% For box around Definition, Theorem, \ldots
\usepackage{mdframed}
\mdfsetup{skipabove=1em,skipbelow=0em}
\theoremstyle{definition}
\newmdtheoremenv[nobreak=true]{definitie}{Definitie}
\newmdtheoremenv[nobreak=true]{eigenschap}{Eigenschap}
\newmdtheoremenv[nobreak=true]{gevolg}{Gevolg}
\newmdtheoremenv[nobreak=true]{lemma}{Lemma}
\newmdtheoremenv[nobreak=true]{propositie}{Propositie}
\newmdtheoremenv[nobreak=true]{stelling}{Stelling}
\newmdtheoremenv[nobreak=true]{wet}{Wet}
\newmdtheoremenv[nobreak=true]{postulaat}{Postulaat}
\newmdtheoremenv{conclusie}{Conclusie}
\newmdtheoremenv{toemaatje}{Toemaatje}
\newmdtheoremenv{vermoeden}{Vermoeden}
\newtheorem*{herhaling}{Herhaling}
\newtheorem*{intermezzo}{Intermezzo}
\newtheorem*{notatie}{Notatie}
\newtheorem*{observatie}{Observatie}
\newtheorem*{oef}{Oefening}
\newtheorem*{opmerking}{Opmerking}
\newtheorem*{praktisch}{Praktisch}
\newtheorem*{probleem}{Probleem}
\newtheorem*{terminologie}{Terminologie}
\newtheorem*{toepassing}{Toepassing}
\newtheorem*{uovt}{UOVT}
\newtheorem*{vb}{Voorbeeld}
\newtheorem*{vraag}{Vraag}

\newmdtheoremenv[nobreak=true]{definition}{Definition}
\newtheorem*{eg}{Example}
\newtheorem*{notation}{Notation}
\newtheorem*{previouslyseen}{As previously seen}
\newtheorem*{remark}{Remark}
\newtheorem*{note}{Note}
\newtheorem*{problem}{Problem}
\newtheorem*{observe}{Observe}
\newtheorem*{property}{Property}
\newtheorem*{intuition}{Intuition}
\newmdtheoremenv[nobreak=true]{prop}{Proposition}
\newmdtheoremenv[nobreak=true]{theorem}{Theorem}
\newmdtheoremenv[nobreak=true]{corollary}{Corollary}

% End example and intermezzo environments with a small diamond (just like proof
% environments end with a small square)
\usepackage{etoolbox}
\AtEndEnvironment{vb}{\null\hfill$\diamond$}%
\AtEndEnvironment{intermezzo}{\null\hfill$\diamond$}%
% \AtEndEnvironment{opmerking}{\null\hfill$\diamond$}%

% Fix some spacing
% http://tex.stackexchange.com/questions/22119/how-can-i-change-the-spacing-before-theorems-with-amsthm
\makeatletter
\def\thm@space@setup{%
  \thm@preskip=\parskip \thm@postskip=0pt
}


% Exercise 
% Usage:
% \oefening{5}
% \suboefening{1}
% \suboefening{2}
% \suboefening{3}
% gives
% Oefening 5
%   Oefening 5.1
%   Oefening 5.2
%   Oefening 5.3
\newcommand{\oefening}[1]{%
    \def\@oefening{#1}%
    \subsection*{Oefening #1}
}

\newcommand{\suboefening}[1]{%
    \subsubsection*{Oefening \@oefening.#1}
}


% \lecture starts a new lecture (les in dutch)
%
% Usage:
% \lecture{1}{di 12 feb 2019 16:00}{Inleiding}
%
% This adds a section heading with the number / title of the lecture and a
% margin paragraph with the date.

% I use \dateparts here to hide the year (2019). This way, I can easily parse
% the date of each lecture unambiguously while still having a human-friendly
% short format printed to the pdf.

\usepackage{xifthen}
\def\testdateparts#1{\dateparts#1\relax}
\def\dateparts#1 #2 #3 #4 #5\relax{
    \marginpar{\small\textsf{\mbox{#1 #2 #3 #5}}}
}

\def\@lecture{}%
\newcommand{\lecture}[3]{
    \ifthenelse{\isempty{#3}}{%
        \def\@lecture{Lecture #1}%
    }{%
        \def\@lecture{Lecture #1: #3}%
    }%
    \subsection*{\@lecture}
    \marginpar{\small\textsf{\mbox{#2}}}
}



% These are the fancy headers
\usepackage{fancyhdr}
\pagestyle{fancy}

% LE: left even
% RO: right odd
% CE, CO: center even, center odd
% My name for when I print my lecture notes to use for an open book exam.
% \fancyhead[LE,RO]{Gilles Castel}

\fancyhead[RO,LE]{\@lecture} % Right odd,  Left even
\fancyhead[RE,LO]{}          % Right even, Left odd

\fancyfoot[RO,LE]{\thepage}  % Right odd,  Left even
\fancyfoot[RE,LO]{}          % Right even, Left odd
\fancyfoot[C]{\leftmark}     % Center

\makeatother




% Todonotes and inline notes in fancy boxes
\usepackage{todonotes}
\usepackage{tcolorbox}

% Make boxes breakable
\tcbuselibrary{breakable}

% Verbetering is correction in Dutch
% Usage: 
% \begin{verbetering}
%     Lorem ipsum dolor sit amet, consetetur sadipscing elitr, sed diam nonumy eirmod
%     tempor invidunt ut labore et dolore magna aliquyam erat, sed diam voluptua. At
%     vero eos et accusam et justo duo dolores et ea rebum. Stet clita kasd gubergren,
%     no sea takimata sanctus est Lorem ipsum dolor sit amet.
% \end{verbetering}
\newenvironment{verbetering}{\begin{tcolorbox}[
    arc=0mm,
    colback=white,
    colframe=green!60!black,
    title=Opmerking,
    fonttitle=\sffamily,
    breakable
]}{\end{tcolorbox}}

% Noot is note in Dutch. Same as 'verbetering' but color of box is different
\newenvironment{noot}[1]{\begin{tcolorbox}[
    arc=0mm,
    colback=white,
    colframe=white!60!black,
    title=#1,
    fonttitle=\sffamily,
    breakable
]}{\end{tcolorbox}}




% Figure support as explained in my blog post.
\usepackage{import}
\usepackage{xifthen}
\usepackage{pdfpages}
\usepackage{transparent}
\newcommand{\incfig}[1]{%
    \def\svgwidth{\columnwidth}
    \import{./figures/}{#1.pdf_tex}
}

% Fix some stuff
% %http://tex.stackexchange.com/questions/76273/multiple-pdfs-with-page-group-included-in-a-single-page-warning
\pdfsuppresswarningpagegroup=1

\title{\Huge{Probability 1}}
\author{\huge{Daniel Yu}}
\date{September 24,2024}

\pdfsuppresswarningpagegroup=1

\begin{document}
\maketitle
\newpage% or \cleardoublepage
% \pdfbookmark[<level>]{<title>}{<dest>}
\tableofcontents
\pagebreak
  
\section{Conditional Expectation}
\begin{definition}
  Let $X,Y$ be random variables on the same space.
   \[
     E[X|Y=y] = \sum_{k \in range(x)} k \cdot P[X=k|Y=y]
  .\]
  We can think of conditional expectations as random variables.
\end{definition}
\begin{note}{Example}\\
  Toss two four sided fair dice, and let $X$= 1st toss and  $M$ =maximum. 

  \begin{enumerate}
    \item What $E[X|M=1]$?  
      $E[X|M=1] = 1 \cdot P[X=1 | M=1]= 1 \cdot \frac{P[X=1 \cap M=1]}{P[M=1]} = 
      1 \cdot \frac{\{(1,1)\}}{\{(1,1)\} } =1$
    \item What about $E[X|M=2]$? 
      \begin{align*}
        &= 1 \cdot P[X=1 | M =2] + 2 \cdot P[X=2|M=2] \\
        &= 1 \cdot \frac{P[X=1 \cap M=2]}{P[M=1]} + 2 \cdot \frac{P[X=2 \cap M=2]}{P[M=2]} \\
        &= 1 \cdot \frac{P[\{(1,2)\}]}{P[\{(1,2),(2,2),(2,1)\}} + 2 \cdot \frac{P[\{(2,1),(2,2)\}}{P[\text{same}]} \\
        &= \frac{\frac{1}{16}}{\frac{3}{16}} + 2 \cdot \frac{\frac{2}{16}}{\frac{3}{16}} \\
        &= \frac{5}{3}
      .\end{align*}
    \item What about $E[M | X=2]$?
      \begin{align*}
        E[M|X=2] &= 2 \cdot \frac{P[M =2 \cap X =2]}{P[X=2]} + 3 \cdot \frac{P[M=3 \cap X=2]}{P[X=2]} + 4 \cdot \frac{P[M=4 \cap X=2]}{P[X=2]} \\
                 &= \frac{11}{4}
      .\end{align*}
  \end{enumerate}
 \end{note}

 \begin{definition}
   The \textbf{law of total expectation}, when $X,Y$ are random variables on the same space:
    \[
      E[X] = \sum_{r \in range(x)} E[X|Y=r] \cdot P[Y=r]
   .\] 
   \begin{proof}
     By writing it out:
     \begin{align*}
       E[X] &= \sum_{r \in range(x)} E[X|Y=r] \cdot P[Y=r] \\
            &= \sum_{r \in range(x)} \sum_{k \in range(x)} k \cdot P[X=k|Y=r] \cdot P[Y=r]  \\
            &= \sum_{k \in range(x)} k \sum_{r \in range(x)}  P[X=k|Y=r] \cdot P[Y=r] \\
            &= \sum_{k \in range(x)} k \cdot P[X=k] \\
            &= E[X]
     .\end{align*}
   \end{proof}
 \end{definition}

 \begin{lemma}
   Another way of writing the previous example is as follows and is known as \textbf{law of iterated expectations}:
   \[
     E[X] = E_Y[E_X[X|Y]]
   .\] 
 \end{lemma}
 \begin{remark}
   The use of these two expectation laws is to be able to now "condition" on a random variable. 
 \end{remark}
 \begin{note}{Example}\\
   Consider a rat in a maze with three intial paths to choose. Let $T$ be the amount of time required for the
   mouse to leave. What is  $E[T]$?

  \begin{proof}
     Let $C$ = 1st choice of the mouse. 
      \[
        E[T] = E[T | C=1] \cdot P[C=1] + E[T|C=2] \cdot P[C=2] + E[T|C=3] \cdot P[C=3] 
      .\]
      Since we don't know anything about the maze other than there are 3 options (this is blc the values
      are not relevant and just for plugging in). Let's just assume that the following values hold (but
      really they could be any values). For choices 2,3 the mouse goes around in a circle and 
      arrives back assume the mouse is dumb and again independently makes a decision, i.e. the experiment 
      resets and T' has the same distribution as T.
      \[
      \begin{cases}
        & c=1, T=3, E[T | C=1] = 3 \\
        & c=2, T=4 + T' \\
        & E[T|c=2] = 4 + E[T] \\
        & c=3, T = 5 + t' \\
        & E[T|c=3] = 5+E[T] 
      \end{cases}
      .\] 
      Substituting these values in,
      \begin{align*}
        E[T] &= 3 \cdot \frac{1}{3} + (4+E[T]) \cdot \frac{1}{3} + (5 + E[T]) \cdot \frac{1}{3} \\
             &= 4 + \frac{2}{3} E[T] \\
        E[T] &= 12
      .\end{align*}
  \end{proof}  
 \end{note}

 \begin{note}{Example}\\
   Consider a similar experiment design. We flip a coin, with $P[H] = p$, each  toss independent. Let
    $G =$ first time we get a $H$. Let $X_1$ be the result of the first coin toss:

    \begin{proof}
      Recall that $G$ is a geometric random variable. So as expected:
       \begin{align*}
         E[G] &= E[G | X_1 = H] \cdot P[X_1=H] + E[G|X_1 = T] \cdot P[X_1 = T] \\
              &= 1 \cdot p + (1+E[G])(1-p) \\
          E[G] &= p + (1-p) -p E[G] \\
               &= \frac{1}{p}
      .\end{align*}
    \end{proof}
\end{note}

\begin{remark}
  What if the above was modified to find $E[D]$ where  $D$ = first time we see two H's in a row. 

  \begin{proof}
    Recall: $E[D] = E_G[E_D[D|G]]$. The \textbf{key to reasoning} about this is to consider what happens
    when you know $G$. Consider the $X_{G+1}$ the result of the coin after $G$. We condition  $E[D|G]$ 
    now on  $X_{G+1}$
     \[
       E[D|G, X_{G+1} =H] = G + 1 \text{ conditional probability reached heads then next roll heads}
    .\] 
    \[
      E[D|G, X_{G+1} = T] = G + 1 + E[D] \text{ full reset of the experiment}
    .\] 
    Then, we substituting into the law of iterated expectations gives us:
    \begin{align*}
      E[D|G] &= E[D|G, X_{G+1} = H] \cdot P[X_{G+1} = H] + E[D|G, X_{G+1}=T] \cdot P[X_{G+1} =T] \\
             &= (G+1) \cdot p + (G+1+E[D]) (1-p) \\
      E[D|G] &= G+1 + (1-p)E[D]
    .\end{align*}
    Now we substitute back for $E[D]$:
     \begin{align*}
       E_{G}[E[D|G]] &= E_G[G+1 + (1+p)E[D]] \\
       E[D] &= \left( \frac{1}{p} + 1 \right) + (1-p)E[D] \\
       E[D] &= \frac{1}{p^2} + \frac{1}{p}
    .\end{align*}
  \end{proof}
\end{remark}

\begin{remark}
  What if now we consider $S$ = first time we see  $HT$, what is  $E[S]$?

  \begin{proof}
    $E[S] = E_G[E_S[S|G]]$. We can keep the same definitions of  $G$ and  $X_{G+1}$ as before. But
    think very carefully about the pattern:
    \[
    \begin{cases}
      & E[S|G, X_{G+1} = T] = G+1 \\
      & E[S|G, X_{G+1} = H] = G + 1 + E[Geo(1-p)] 
    \end{cases}
    .\]  
    This changes becasue now the pattern is $T \ldots T H H$ and we \textbf{don't} restart because $H$ is
     the beginning of a  $HT$ pattern. So what we are really now searching for is the first time we 
     get a tails $H H \ldots H T$! This is $Geo(1-p)$.
    \begin{align*}
      E[S|G] &= (G+1) \cdot (1-p) + (G+1+\frac{1}{1-p}) \cdot p \\ 
             &= G+1 + \frac{p}{1-p}
    .\end{align*}
    Then, 
    \[
      E[S] = \frac{1}{p} + 1 + \frac{p}{1-p} = \frac{1}{p(1-p)}
    .\] 
  \end{proof} 
\end{remark}

\begin{note}{Example} \\
  Suppose $N$ is the number of customers that walk into a bank on thursday. Each customer deposits 2 or withdraws
  1, independently at random. Let  $Y$ = total cashflow on Thursday. What is $E[Y]$
  
  \begin{proof}
    Let $X_i \in \{2,-1\} $ represent the random variable of the customers behavior. Note that we can break up
    the expected value in this way because we are implicitly assuming (based one the problem) that the number of 
    people $N$ doesn't affect each customer's decision  $X_i$
    \begin{align*}
      E[Y|N=k] &= E[\sum_{i=1}^{k} X_i] = k \cdot E[X_1] = k \cdot \left( \frac{1}{2} \cdot 2 - \frac{1}{2} \cdot 1 \right) = \frac{k}{2} \\
      E[Y] &= \sum_{k=1}^{\infty} E[Y|N=k] \cdot P[n=k] \\
           &= \sum_{k=1}^{\infty} \frac{k}{2} \cdot P[N=k] \\
           &= \frac{1}{2}\sum_{k=1}^{\infty} k \cdot P[N=k] \\
           &= \frac{1}{2} E[N]
    .\end{align*}
    completely expected! 
  \end{proof}

  What is the variance of $Y$?
  \begin{proof}
    First compute, $E[Y^2]$
    \begin{align*}
      E[Y^2] &= \sum_{k=1}^{\infty} E[Y^2 | N=k] \cdot P[N=k] \\
             &= \sum_{k=1}^{\infty} E[\left( \sum_{i=1}^{k} X_i \right)^{2} | N=k] \cdot P[N\cdot k] \\
             &= \sum_{k=1}^{\infty} E[\sum_{i=1}^{k} X_i^{2} + \sum_{i \neq j} X_i X_j | N=k] \cdit P[N \cdot k] \\
             &= \sum_{k=1}^{\infty} [\sum_{i=1}^{k} E[X_i^{2}| N=k] + \sum_{i\neq j} E[X_i X_j | N=k]] \cdot P[N=k] \\
             &= \sum_{k=1}^{\infty} [\sum_{i=1}^{k} E[X_i^{2}] + \sum_{i\neq j} E[X_i X_j]] \cdot P[N=k] 
    .\end{align*}
    We know that $X_i, X_j$ are independent, so $E[X_i X_j] = \frac{1}{2} \cdot \frac{1}{2} = \frac{1}{4}$. Now
    let's analyze $X_i^{2}$ (since all $X_i$ have the same distribution):
    \[
    X_i = \begin{cases}
      4, P[X_2 = 4] = \frac{1}{2} \\
      1, P[X_2 = 1] = \frac{1}{2}
    \end{cases}
    .\] 
    Then, $E[X_i]^{2}=\frac{5}{2}$ :
    \begin{align*}
      E[Y^2] &= \sum_{k=1}^{\infty} [\sum_{i=1}^{k} \frac{5}{2} + \sum_{i\neq j} \frac{1}{4}] \cdot P[N=k] \\
             &= \sum_{k=1}^{\infty} [k \cdot \frac{5}{2} + k \cdot (k-1) \frac{1}{4}] \cdot P[N=k] \\
             &= \sum_{k=1}^{\infty} (k \cdot \frac{1}{4}) \cdot P[N=k] + \sum_{k=1}^{\infty} k \cdot (\frac{5}{2}-\frac{1}{4}) \cdot P[N=k] \\
             &= E[X_i]^{2} \cdot E[N^{2} + Var(X_1) + E[N] \\
    .\end{align*}
    and the $Var(Y) = $
  \end{proof}
\end{note}

\end{document}
