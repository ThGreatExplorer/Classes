\documentclass[a4paper]{article}
\usepackage[a4paper, margin=1in]{geometry}
\input{preamble}
\title{\Huge{Intro to Probability Models, Ross}\\Chapter 1 Problems}
\author{\huge{Daniel Yu}}
\date{September 25,2024}

\pdfsuppresswarningpagegroup=1

\begin{document}
\maketitle
\newpage% or \cleardoublepage
% \pdfbookmark[<level>]{<title>}{<dest>}
\pagebreak
\begin{enumerate}
  \item Drawing marbles with replacement
    \begin{note}
      There are three marbles $R,G,B$. The experiment is to draw a marble from the box and then 
      replace it and draw a second marble. The sample space  $\Omega = \{R,G,B\}^2$. The probability
      of each point in the sample space given each marble is equally likely to be drawn (i.e.
      uniform probability measure) is $\frac{1}{|\Omega|}=\frac{1}{9}$.
    \end{note}
  \item If the marble is drawn without replacement the second time.
    \begin{note}
      If the marble is drawn without replacement, then $\Omega = \{\left( R,G \right), \left( R,B \right), 
      \left( G,B \right), (G,R), (B,R), (B,G) \}$. The porbability given each marble is equally likely to
      be drawn is $\frac{1}{\Omega}= \frac{1}{6}$
    \end{note}
  \item Problem 5
    \begin{note}
      The probability that the man is a winner is $P[\text{wins}] = P[\text{wins first time}] + P[\text{wins second time}] 
      = P[\text{wins first time}] + P[\text{loses the first time and wins second time}]$:
       \[
      = \frac{1}{2} + \frac{1}{2} \cdot \frac{1}{2} = \frac{3}{4}
      .\]
      While this has a greater than $\frac{1}{2}$ chance of winning, let $X$ be the random variable  denoting 
      the money won or lost:
      $E[X] = 1 \cdot \frac{1}{2} + 1 \cdot \frac{1}{4} - 3 \cdot \frac{1}{4} = 0$
      So while this strategy gives a greater than fair chance of winning, because you will either make  $1$ or 
      lose $3$ with no edge, the strategy isn't for everyone.
    \end{note}
  \item Problem 17
    \begin{note}
      If the coins are fair, then the probability that the game ends in the first round:
      \[
        P[\text{ends in 1}] = 1 - P[\text{all heads}] - P[\text{all tails}] = 1 - (\frac{1}{2}^{3} + \frac{1}{2}^{3}]
        = \frac{3}{4}
      .\] 
      Alternatively,
      \[
        P[\text{ends in 1}] = P[\text{2H 1T}] + P[\text{1H 2T}] = \begin{pmatrix} 3 \\ 2 \end{pmatrix} \cdot
        \frac{1}{2}^{3} + \begin{pmatrix} 3 \\ 2 \end{pmatrix} \cdot \frac{1}{2}^{3} = \frac{6}{8} = \frac{3}{4}
      .\]
      The reason we use combinatorics is because we know each outcome is equally likely (uniform) so we count all the 
      different combinations of $2H,1T$ and vice versa that are possible. \\

      Now consider if all 3 coins are biased with  $P[H] = \frac{1}{4}$. Then,
      \[
        P[\text{ends in 1} = 1 - \frac{1}{4}^{3} - \frac{3}{4}^{3} = \frac{36}{64} = \frac{9}{16}
      .\] 
    \end{note}
  \item Problem 19: What is $P(\text{ > 1 dice is 6})$? What is  $P(\text{ 1 > dice is 6} | \text{2 dice face diff})$?
    \begin{note}
      $\Omega = \{1,2,3,4,5,6\}^{2}$. Fair dice. Then,
      \[
        P[\text{at least 1 6}] = \frac{\mid \{\text{at least 1 6} \}\mid }{|\Omega|} = \frac{11}{36}
      .\] 
      and,
      \[
        P[\text{at least 1 6} | \text{2 dice face diff}] = \frac{P[\text{at least 1 6} \cap \text{2 dice face diff}]}{
        P[\text{2 dice face diff}]} = \frac{\frac{10}{36}}{\frac{30}{36}} = \frac{1}{3}
      .\] 
    \end{note}
  \item Problem 25: 2 cards are randomly selected from a deck of 52 playing cards:
    \begin{enumerate}
      \item What is the probability they form a pair?
        \begin{note}
          This can be thought of no. ways to choose valid pairs / no. ways to choose 2 cards. We know there 
          are $4$ cards of the same denomination :
          \[
            P[\text{valid pair}] = \frac{52 \cdot 3}{52 \cdot 51} = \frac{3}{51}
          .\] 
        \end{note}
      \item What is the conditional probability that they constitute a pair given different suits?
        \begin{note}
          Given from different suits,
          \[
            P[\text{valid pair} | \text{diff suits}] = \frac{P[\text{valid pair and diff suits}]}{P[\text{diff suits}]} 
            = \frac{\frac{3}{51}}{\frac{52 \cdot 39}{52 \cdot 51}} = \frac{\frac{3}{51}}{\frac{39}{51}} = \frac{3}{39} = \frac{1}{13}
          .\] 
        \end{note}
    \end{enumerate}
  \item Problem 27:
    \begin{note}
      We know from the previous question:
      \[
        P[E_1 E_2 E_3 E_4] = P[E_1] P[E_2|E_1] P[E_3|E_1 E_2] P[E_4 | E_1 E_2 E_3]
      .\] 
      $P[E_1]=1$ because the ace of spades has to be in one of the four piles.  
       \[
         P[E_2|E_1] = P[E_2] = \frac{52 \cdot 39 \cdot 50!}{52!} = \frac{39}{51} 
      .\] 
      since there are 52 options for where the ace of spaces can be placed, whichever place the ace is placed down at,
      the ace of hearts to be in a different pile, can't occupy the 12 cards in that pile. This works because
      the number of cards in each pile are fixed and the cards are not indistinguishable (so stars and bars 
      does not apply). 
      \[
        P[E_3 | E_1 \cap E_2] = \frac{P[E_3 \cap \left( E_1 \cap E_2 \right)] }{P[E_1 \cap E_2]} 
        = \frac{P[E_3 \cap E_2]}{P[E_2]} = \frac{P[E_3]}{P[E_2]} = \frac{\frac{52 \cdot 39 \cdot 26 \cdot 49!}{52!}}{\frac{39}{51}}
        = \frac{\frac{39 \cdot 26}{51 \cdot 50}}{\frac{39}{51}} = \frac{26}{50}
      .\] 
      and from definition of $E_1,E_2,E_3,E_4$:
      \[
        P[E_4 | E_1 \cap E_2 \cap E_3] = P[E_4 | E_2 \cap E_3] = P[E_4 | E_3] =
        \frac{P[E_4 \cap E_3]}{P[E_3]} = \frac{P[E_4]}{P[E_3]} = \frac{\frac{52 \cdot 39 \cdot 26 \cdot 13 \cdot 48!}{52!}}{
        \frac{39 \cdot 26}{51 \cdot 50}} =  \frac{\frac{39 \cdot 26 \cdot 13}{51 \cdot 50 \cdot 49}}{\frac{39 \cdot 26}{51 \cdot 50}}
        = \frac{13}{49} 
      .\]
      This ends up giving us:
      \[
        P[E_1 E_2 E_3 E_4] = 1 \cdot \frac{39}{51} \cdot \frac{26}{50} \cdot \frac{13}{49} 
      .\]
      This is equivalent to asking what the probability that each pile has an ace is!!!
    \end{note}
  \item Problem 30:
    \begin{enumerate}
      \item Given, one shot hit target what was probability it was George?
        \begin{note}
          Let $X_1$ be random variable representing if George hit target and  $X_2$ be random variable representing
          if Bill hit the target.  $P[X_1 = 1] = .4$ and  $P[X_2=1]=.7$. We know that $X_1$ is independent of  $X_2$
          since George and Bill fire independently.
          \[
            P[X_1 = 1| \text{only 1 shot hit}] = \frac{P[X_1 = 1 \cap \text{only one shot hit}]}{P[\text{only 1 shot hit}]}
            = \frac{P[X_1=1 \cap X_2 =0]}{P[X_1=1 \cap X_2=0] + P[X_1 =0 \cap X_2 =1]}
          .\] 
          This gives,
          \[
          = \frac{.4 \cdot .3}{.4 \cdot .3 + .6 \cdot .7} = \frac{2}{9}
          .\] 
        \end{note}
      \item Give the target is hit, what is the probability George hit it?
        The difference here is that both George and Bill could have hit, i.e. $X_1 =1 \cap X_2=1$
        \[
          P[X_1 = 1 | \text{target hit}] = \frac{P[X_1 = 1 \cap \text{target hi}]}{P[\text{target hit}]} = 
          \frac{P[X_1 = 1 \cap X_2 = 0] + P[X_1 = 1 \cap X_2 = 1]}{P[X_1 =1 \cap X_2 = 0] + P[X_1 = 1 \cap X_2 = 1] 
          + P[X_1 = 0 \cap X_2 = 1]}
        .\]
        \[
        = \frac{.4}{.4 + .6 \cdot .7} = \frac{20}{41}
        .\] 
    \end{enumerate}
  \item Problem 32
    \begin{note}
      Let's quickly verify that this makes sense. Let $X_1$ be the random variable representing if each person
      selected their hat or not. For $n=2$:
       \[
         P[X_1 = 0 \cap X_2 = 0] = \frac{1}{2} \cdot \frac{1}{1} = \frac{1}{2} = \frac{1}{2!} 
      .\] 
      For $n=3$:
       \[
         P[X_1 = 0 \cap X_2 = 0 \cap X_3=0] = \frac{2}{3} \cdot \frac{1}{2} \cdot 1 = \frac{1}{3} =
         \frac{1}{2} - \frac{1}{3!} 
      .\] 
      For $n=4$:
       \[
         P[X_1 = 0 \cap X_2 = 0 \cap X_3=0 \cap X_4 = 0] = \frac{3}{4} \cdot \frac{2}{3} \cdot \frac{1}{2} = \frac{1}{4}
         = 
      .\] 
    \end{note}
\end{enumerate}

\end{document}
