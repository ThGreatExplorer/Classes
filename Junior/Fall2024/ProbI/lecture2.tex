\documentclass[a4paper]{article}
% Some basic packages
\usepackage[utf8]{inputenc}
\usepackage[T1]{fontenc}
\usepackage{textcomp}
\usepackage[dutch]{babel}
\usepackage{url}
\usepackage{graphicx}
\usepackage{float}
\usepackage{booktabs}
\usepackage{enumitem}

\pdfminorversion=7

% Don't indent paragraphs, leave some space between them
\usepackage{parskip}

% Hide page number when page is empty
\usepackage{emptypage}
\usepackage{subcaption}
\usepackage{multicol}
\usepackage{xcolor}

% Other font I sometimes use.
% \usepackage{cmbright}

% Math stuff
\usepackage{amsmath, amsfonts, mathtools, amsthm, amssymb}
% Fancy script capitals
\usepackage{mathrsfs}
\usepackage{cancel}
% Bold math
\usepackage{bm}
% Some shortcuts
\newcommand\N{\ensuremath{\mathbb{N}}}
\newcommand\R{\ensuremath{\mathbb{R}}}
\newcommand\Z{\ensuremath{\mathbb{Z}}}
\renewcommand\O{\ensuremath{\emptyset}}
\newcommand\Q{\ensuremath{\mathbb{Q}}}
\newcommand\C{\ensuremath{\mathbb{C}}}

% Easily typeset systems of equations (French package)
\usepackage{systeme}

% Put x \to \infty below \lim
\let\svlim\lim\def\lim{\svlim\limits}

%Make implies and impliedby shorter
\let\implies\Rightarrow
\let\impliedby\Leftarrow
\let\iff\Leftrightarrow
\let\epsilon\varepsilon

% Add \contra symbol to denote contradiction
\usepackage{stmaryrd} % for \lightning
\newcommand\contra{\scalebox{1.5}{$\lightning$}}

% \let\phi\varphi

% Command for short corrections
% Usage: 1+1=\correct{3}{2}

\definecolor{correct}{HTML}{009900}
\newcommand\correct[2]{\ensuremath{\:}{\color{red}{#1}}\ensuremath{\to }{\color{correct}{#2}}\ensuremath{\:}}
\newcommand\green[1]{{\color{correct}{#1}}}

% horizontal rule
\newcommand\hr{
    \noindent\rule[0.5ex]{\linewidth}{0.5pt}
}

% hide parts
\newcommand\hide[1]{}

% si unitx
\usepackage{siunitx}
\sisetup{locale = FR}

% Environments
\makeatother
% For box around Definition, Theorem, \ldots
\usepackage{mdframed}
\mdfsetup{skipabove=1em,skipbelow=0em}
\theoremstyle{definition}
\newmdtheoremenv[nobreak=true]{definitie}{Definitie}
\newmdtheoremenv[nobreak=true]{eigenschap}{Eigenschap}
\newmdtheoremenv[nobreak=true]{gevolg}{Gevolg}
\newmdtheoremenv[nobreak=true]{lemma}{Lemma}
\newmdtheoremenv[nobreak=true]{propositie}{Propositie}
\newmdtheoremenv[nobreak=true]{stelling}{Stelling}
\newmdtheoremenv[nobreak=true]{wet}{Wet}
\newmdtheoremenv[nobreak=true]{postulaat}{Postulaat}
\newmdtheoremenv{conclusie}{Conclusie}
\newmdtheoremenv{toemaatje}{Toemaatje}
\newmdtheoremenv{vermoeden}{Vermoeden}
\newtheorem*{herhaling}{Herhaling}
\newtheorem*{intermezzo}{Intermezzo}
\newtheorem*{notatie}{Notatie}
\newtheorem*{observatie}{Observatie}
\newtheorem*{oef}{Oefening}
\newtheorem*{opmerking}{Opmerking}
\newtheorem*{praktisch}{Praktisch}
\newtheorem*{probleem}{Probleem}
\newtheorem*{terminologie}{Terminologie}
\newtheorem*{toepassing}{Toepassing}
\newtheorem*{uovt}{UOVT}
\newtheorem*{vb}{Voorbeeld}
\newtheorem*{vraag}{Vraag}

\newmdtheoremenv[nobreak=true]{definition}{Definition}
\newtheorem*{eg}{Example}
\newtheorem*{notation}{Notation}
\newtheorem*{previouslyseen}{As previously seen}
\newtheorem*{remark}{Remark}
\newtheorem*{note}{Note}
\newtheorem*{problem}{Problem}
\newtheorem*{observe}{Observe}
\newtheorem*{property}{Property}
\newtheorem*{intuition}{Intuition}
\newmdtheoremenv[nobreak=true]{prop}{Proposition}
\newmdtheoremenv[nobreak=true]{theorem}{Theorem}
\newmdtheoremenv[nobreak=true]{corollary}{Corollary}

% End example and intermezzo environments with a small diamond (just like proof
% environments end with a small square)
\usepackage{etoolbox}
\AtEndEnvironment{vb}{\null\hfill$\diamond$}%
\AtEndEnvironment{intermezzo}{\null\hfill$\diamond$}%
% \AtEndEnvironment{opmerking}{\null\hfill$\diamond$}%

% Fix some spacing
% http://tex.stackexchange.com/questions/22119/how-can-i-change-the-spacing-before-theorems-with-amsthm
\makeatletter
\def\thm@space@setup{%
  \thm@preskip=\parskip \thm@postskip=0pt
}


% Exercise 
% Usage:
% \oefening{5}
% \suboefening{1}
% \suboefening{2}
% \suboefening{3}
% gives
% Oefening 5
%   Oefening 5.1
%   Oefening 5.2
%   Oefening 5.3
\newcommand{\oefening}[1]{%
    \def\@oefening{#1}%
    \subsection*{Oefening #1}
}

\newcommand{\suboefening}[1]{%
    \subsubsection*{Oefening \@oefening.#1}
}


% \lecture starts a new lecture (les in dutch)
%
% Usage:
% \lecture{1}{di 12 feb 2019 16:00}{Inleiding}
%
% This adds a section heading with the number / title of the lecture and a
% margin paragraph with the date.

% I use \dateparts here to hide the year (2019). This way, I can easily parse
% the date of each lecture unambiguously while still having a human-friendly
% short format printed to the pdf.

\usepackage{xifthen}
\def\testdateparts#1{\dateparts#1\relax}
\def\dateparts#1 #2 #3 #4 #5\relax{
    \marginpar{\small\textsf{\mbox{#1 #2 #3 #5}}}
}

\def\@lecture{}%
\newcommand{\lecture}[3]{
    \ifthenelse{\isempty{#3}}{%
        \def\@lecture{Lecture #1}%
    }{%
        \def\@lecture{Lecture #1: #3}%
    }%
    \subsection*{\@lecture}
    \marginpar{\small\textsf{\mbox{#2}}}
}



% These are the fancy headers
\usepackage{fancyhdr}
\pagestyle{fancy}

% LE: left even
% RO: right odd
% CE, CO: center even, center odd
% My name for when I print my lecture notes to use for an open book exam.
% \fancyhead[LE,RO]{Gilles Castel}

\fancyhead[RO,LE]{\@lecture} % Right odd,  Left even
\fancyhead[RE,LO]{}          % Right even, Left odd

\fancyfoot[RO,LE]{\thepage}  % Right odd,  Left even
\fancyfoot[RE,LO]{}          % Right even, Left odd
\fancyfoot[C]{\leftmark}     % Center

\makeatother




% Todonotes and inline notes in fancy boxes
\usepackage{todonotes}
\usepackage{tcolorbox}

% Make boxes breakable
\tcbuselibrary{breakable}

% Verbetering is correction in Dutch
% Usage: 
% \begin{verbetering}
%     Lorem ipsum dolor sit amet, consetetur sadipscing elitr, sed diam nonumy eirmod
%     tempor invidunt ut labore et dolore magna aliquyam erat, sed diam voluptua. At
%     vero eos et accusam et justo duo dolores et ea rebum. Stet clita kasd gubergren,
%     no sea takimata sanctus est Lorem ipsum dolor sit amet.
% \end{verbetering}
\newenvironment{verbetering}{\begin{tcolorbox}[
    arc=0mm,
    colback=white,
    colframe=green!60!black,
    title=Opmerking,
    fonttitle=\sffamily,
    breakable
]}{\end{tcolorbox}}

% Noot is note in Dutch. Same as 'verbetering' but color of box is different
\newenvironment{noot}[1]{\begin{tcolorbox}[
    arc=0mm,
    colback=white,
    colframe=white!60!black,
    title=#1,
    fonttitle=\sffamily,
    breakable
]}{\end{tcolorbox}}




% Figure support as explained in my blog post.
\usepackage{import}
\usepackage{xifthen}
\usepackage{pdfpages}
\usepackage{transparent}
\newcommand{\incfig}[1]{%
    \def\svgwidth{\columnwidth}
    \import{./figures/}{#1.pdf_tex}
}

% Fix some stuff
% %http://tex.stackexchange.com/questions/76273/multiple-pdfs-with-page-group-included-in-a-single-page-warning
\pdfsuppresswarningpagegroup=1

\title{\Huge{Probability I - Lecture 2}}
\author{\huge{Daniel Yu}}
\date{September 12, 2024}

\pdfsuppresswarningpagegroup=1

\begin{document}
\maketitle
\newpage% or \cleardoublepage
% \pdfbookmark[<level>]{<title>}{<dest>}
\tableofcontents
\pagebreak

\section{Expectation}
\begin{definition}
  If $X$ is a discrete random variable, then the expectation (\textbf{expected value}) is:
  \[
    E[X] = \sum_{k \in range(X)} k \cdot P[X=k]
  .\]
  aka a space average. If $X$ is a continuous random variable with pdf $f_x\left(t \right)$,
  \[
    E[X] = \int_{-\infty}^\infty t \cdot f_x(t) dt
  .\] 
\end{definition}

\begin{remark}{Examples}\\
  \begin{itemize}
    \item $E[Bernoulli(\theta)] = 0 \cdot (1-p) + 1 \cdot p = p$
    \item Let $Y$ be the outcome of a fair 3 sided die. $E[Y] = \frac{1+2+3+4}{4} = 2.5$. 
    \item M = maxiumum of two fair 4-sided dice. Let $M$ be random variable representing the maximum of two fair
      dice rolls (independent). The $\Omega= \{(1,1), \ldots \}$. Also, $P$ is uniform for dice rolls. $E[M]= 1 \cdot P[M=1] + 2 \cdot P[M=2] + 3 \cdot P[M=3] + 4 \cdot P[M=4] = 1 \cdot \frac{1}{16} +
      2 \frac{3}{16} + 3 \frac{5}{16} + 4 \frac{7}{16}$
    \item $R \sim Geo(p)$ geometric distribution. Define $P[R=k]= p(1-p)^{k}$ and $k \in \{1,2,3, \ldots\} $. 
    \item $x \sim \exp(\lambda)$ i.e. $f_x(t) = \lambda e^{-\lambda t}$ when $t \geq 0$ and $0$ otherwise. 
  \end{itemize}
\end{remark}

\begin{theorem}
  If $X$ and $Y$ are two random varirables defined on the same space and $a,b \in \R$. Then,
  \[
    E[aX + bY] = aE[X] + b E[Y]
  .\] 
\end{theorem}
\begin{proof}{Discrete case}\\
  \begin{align*}
    E[Z] &= \sum_{k \in range(z)} k \cdot P[Z=k] \\
         &= \sum_{r \in range(X), t \in range(y)} (ar + bt)P[Z = ar + bt] \\ 
         &= \sum_{r,t} (ar + bt) \cdot P[X=r, Y=t] \text{ ---  Note r,t is shorthand} \\ 
         &= \sum_r \sum_t ar P[X=r, Y=t] + \sum_r \sum_t bt \cdot P[X=r, Y=t] \\
         &= \sum_r ar \cdot (\sum_t P[X=r, Y=t) + \sum_t bt \sum_r P[X=r,Y=t] \\
         &= a \sum_r r \cdot P[X=r] + b \sum_t t \cdot P[Y=t] \\
         &= a E[X] + b E[Y]
  .\end{align*}
\end{proof}

This is useful! For example, back to card dealing, assuming we agin deal 3 cards from 52 cards. Observe
$A = X + Y + Z$, we can use $E[A] = E[X] + E[Y] + E[Z] = \frac{3}{13}$

\section{Higher Moments}
\[
  E[x^k] = \text{ kth moment of X}
.\] 
\subsection{Variance}
\begin{definition}
  The second moment known as the \textbf{variance}: $E[x^2]$:
   \[
     var\left( x \right) = E[x^2] - E[x]^2 
  .\] 
\end{definition}

\begin{lemma}
  \[
    var\left( x \right) = E[(X - E[X])^2] 
  .\] 
\end{lemma}

\begin{remark}
  The two methods of computation are equivalent but different. For example:
  \[
    E[Ber(p)^2] = p, E[Ber(p)]= p^2 \to var(Ber(p)) = p - p^2
  .\] 
  Then, 
  \[
   Y = Ber(p) - E[Ber(p)] = Ber(p) - p
  .\] 
  where $Y$ is the distribution:
  \[
    P[Y=p-1] = p
    P[Y=-p] = 1-p
  .\] 
  and the distribution of $Y^2$:
  \[
    P[Y^2 = (1-p)^2] = p^2 
    P[Y^2 = p^2] = 1-p
  .\] 
  so when we compute we get the same answer:
  \begin{align*}
    var(Ber(p)) &= E[(Ber(p) - E[Ber(p)])] = E[Y^2] \\
                &= (1-p)^2 \cdot p + p^2 \cdot (1-p) \\
                &= p (1-p) \\
                &= p - p^2
  .\end{align*}
\end{remark}

\begin{proof}{Let $\mu = E[X]$
  \begin{align*}
    E[(X - \mu)^2] &= E[X^2 - 2 \mu X + \mu^2] \\
                   &= E[X^2] - 2 \mu E[X] + \mu^2 \\
                   &= E[X^2] - 2 \mu^2 + \mu^2 \\
                   &= E[X^2] - \mu^2 \\
                   &= E[X^2] - E[X]^2 
  .\end{align*}
\end{proof}

\begin{remark}
  Variance can be thought of as a measure of how \textbf{random} a R.V. is.
\end{remark}

\begin{prop}
  If $var(x) = 0 \iff P[x=a]=1$ for some $a$ i.e. $X$ is not a random variable at all!

  \begin{proof}
    Assume $var(x) = 0$. Then this means $E[(X - \mu)^2] = 0$. 
  \end{proof}
\end{prop}

\begin{remark}{Example} \\
  $X = \{1,2,3,4\}$ with P uniform. $E[X] = \frac{5}{2}$. 
  \[
  E[X^2] = 1 \cdot P[X^2 = 1] + 4 \cdot P[X^2 = 4] + 9 \cdot P[X^2 = 9] + 16 \cdot P[X^2 = 16]
  .\] 
  \[
  = \frac{1+4+9+16}{4} = \frac{15}{2}
  .\] 
  So,
  \[
  var(X) = \frac{15}{2} - \frac{5}{2}^2 = \frac{5}{4}
  .\] 
\end{remark}

\begin{remark}{Example} \\
  $U \sim uniform(0,2)$ continuous. $E[U] = \int_0^2 t \cdot \frac{1}{2}$ since uniform probability would be .5 
  across interval from 0 to 2. 
  \[
    var(u) = E[(U-1)^2] = \int_0^2 (t-1)^2 \cdot .5 dt  
  .\] 
  \[
    = \frac{1}{2} \int_0^2 t^2 - 2t + 1 dt
  .\] 
  \[
    = \frac{1}{2} [\frac{t}{3}^3 - t^2 + t]_0^2
  .\] 
  \[
   = \frac{1}{3}
  .\] 
  
\end{remark}

\begin{prop}
  \begin{enumerate}
    \item Variance is not preserved under scalar multiplication \begin{align*}
    var(aX) &= E[(aX)^2] - E[aX]^2 \\ 
            &= E[a^2 X_2] - (aE[x])^2 \\
            &= a^2 (E[X^2] - E[X]^2) \\
            &= a^2 \cdot var(X)
  .\end{align*}
    \item variance is preserved under additivity \[
  var(X + b) = var(X)
  .\] 

    \item Variance is nonlinear  \[
    var(X + Y) \neq var(X) + \var(Y) \text{ in general}
  .\]  
  \end{enumerate}
\end{prop}

\begin{definition}
  We have a special term to measure the relationship of between the variance sof 2 random variables known as the 
  \textbf{covariance}
  \[
    cov(X,Y) = E[XY] - E[X] E[Y]
  .\] 
  The matrix covariance formula for $X,Y$ matrices is:
  $Cov(X,Y) = E[(X-E[X])(Y-E[Y])^T]
\end{definition}
\begin{lemma}
  \[
  Var(X + Y) = Var(X) + Var(Y) + 2Cov(X,Y)
  .\] 
\end{lemma}

\begin{proof}
\begin{align*}
  E[(X + Y)^2] &= E[X^2] + 2E[XY] + E[Y^2] \\
  E[X+Y]^2 &= E[X]^2 + 2E[X]E[Y] + E[Y]^2 \\
  \text{Var}(X + Y) &= (E[X^2] + 2E[XY] + E[Y^2]) - (E[X]^2 + 2E[X]E[Y] + E[Y]^2) \\
  &= E[X^2] - E[X]^2 + 2(E[XY] - E[X]E[Y]) + E[Y^2] - E[Y]^2 \\
  &= \text{Var}(X) + 2\text{Cov}(X,Y) + \text{Var}(Y)
\end{align*}
\end{proof}

Recall the example from the previous lecture
\begin{note}
  For example, consider the discrete random variables $x,y,z$ over $\Omega = \left\{ 1,2,3,4 \right\}$ on a uniform
  probability measure:
  \[
  X = \begin{cases}
    1, w \in \left\{ 1,2 \right\} \\
    0, w \in \left\{ 3,4 \right\} 
  \end{cases}
  .\] 
  \[
  Y = \begin{cases}
    1, w \in \left\{ 3,4 \right\} \\
    0, w \in \left\{ 1,2 \right\} 
  \end{cases}
  .\] 
  \[
  z = \begin{cases}
    1, w \in \left\{ 1,3 \right\} \\
    0, w \in \left\{ 2,4 \right\} 
  \end{cases}
  .\] 
  What is the variance of $X+Y$?
  \[
  var(X+Y) = Var(x) + Var(y) + 2 Cov(X,Y) = \frac{1}{4} + \frac{1}{4} + 2 Cov(X,Y)
  .\] 
  \[
    Cov(X,Y) = E[XY] - E[X]E[Y] = E[XY] \cdot \frac{1}{4} = -\frac{1}{4}
  .\]
  \[
  var(X+Y) = \frac{1}{2} - 2 \cdot \frac{1}{4} = 0
  !\]
\end{note}

\begin{definition}
  If $Cov(X,Y) = 0$ we say $X,Y$ are uncorrelated
\end{definition}

\end{document}
