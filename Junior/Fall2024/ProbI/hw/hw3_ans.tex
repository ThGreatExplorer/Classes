\documentclass[a4paper]{article}
\usepackage[a4paper, margin=1in]{geometry}
% Some basic packages
\usepackage[utf8]{inputenc}
\usepackage[T1]{fontenc}
\usepackage{textcomp}
\usepackage[dutch]{babel}
\usepackage{url}
\usepackage{graphicx}
\usepackage{float}
\usepackage{booktabs}
\usepackage{enumitem}

\pdfminorversion=7

% Don't indent paragraphs, leave some space between them
\usepackage{parskip}

% Hide page number when page is empty
\usepackage{emptypage}
\usepackage{subcaption}
\usepackage{multicol}
\usepackage{xcolor}

% Other font I sometimes use.
% \usepackage{cmbright}

% Math stuff
\usepackage{amsmath, amsfonts, mathtools, amsthm, amssymb}
% Fancy script capitals
\usepackage{mathrsfs}
\usepackage{cancel}
% Bold math
\usepackage{bm}
% Some shortcuts
\newcommand\N{\ensuremath{\mathbb{N}}}
\newcommand\R{\ensuremath{\mathbb{R}}}
\newcommand\Z{\ensuremath{\mathbb{Z}}}
\renewcommand\O{\ensuremath{\emptyset}}
\newcommand\Q{\ensuremath{\mathbb{Q}}}
\newcommand\C{\ensuremath{\mathbb{C}}}

% Easily typeset systems of equations (French package)
\usepackage{systeme}

% Put x \to \infty below \lim
\let\svlim\lim\def\lim{\svlim\limits}

%Make implies and impliedby shorter
\let\implies\Rightarrow
\let\impliedby\Leftarrow
\let\iff\Leftrightarrow
\let\epsilon\varepsilon

% Add \contra symbol to denote contradiction
\usepackage{stmaryrd} % for \lightning
\newcommand\contra{\scalebox{1.5}{$\lightning$}}

% \let\phi\varphi

% Command for short corrections
% Usage: 1+1=\correct{3}{2}

\definecolor{correct}{HTML}{009900}
\newcommand\correct[2]{\ensuremath{\:}{\color{red}{#1}}\ensuremath{\to }{\color{correct}{#2}}\ensuremath{\:}}
\newcommand\green[1]{{\color{correct}{#1}}}

% horizontal rule
\newcommand\hr{
    \noindent\rule[0.5ex]{\linewidth}{0.5pt}
}

% hide parts
\newcommand\hide[1]{}

% si unitx
\usepackage{siunitx}
\sisetup{locale = FR}

% Environments
\makeatother
% For box around Definition, Theorem, \ldots
\usepackage{mdframed}
\mdfsetup{skipabove=1em,skipbelow=0em}
\theoremstyle{definition}
\newmdtheoremenv[nobreak=true]{definitie}{Definitie}
\newmdtheoremenv[nobreak=true]{eigenschap}{Eigenschap}
\newmdtheoremenv[nobreak=true]{gevolg}{Gevolg}
\newmdtheoremenv[nobreak=true]{lemma}{Lemma}
\newmdtheoremenv[nobreak=true]{propositie}{Propositie}
\newmdtheoremenv[nobreak=true]{stelling}{Stelling}
\newmdtheoremenv[nobreak=true]{wet}{Wet}
\newmdtheoremenv[nobreak=true]{postulaat}{Postulaat}
\newmdtheoremenv{conclusie}{Conclusie}
\newmdtheoremenv{toemaatje}{Toemaatje}
\newmdtheoremenv{vermoeden}{Vermoeden}
\newtheorem*{herhaling}{Herhaling}
\newtheorem*{intermezzo}{Intermezzo}
\newtheorem*{notatie}{Notatie}
\newtheorem*{observatie}{Observatie}
\newtheorem*{oef}{Oefening}
\newtheorem*{opmerking}{Opmerking}
\newtheorem*{praktisch}{Praktisch}
\newtheorem*{probleem}{Probleem}
\newtheorem*{terminologie}{Terminologie}
\newtheorem*{toepassing}{Toepassing}
\newtheorem*{uovt}{UOVT}
\newtheorem*{vb}{Voorbeeld}
\newtheorem*{vraag}{Vraag}

\newmdtheoremenv[nobreak=true]{definition}{Definition}
\newtheorem*{eg}{Example}
\newtheorem*{notation}{Notation}
\newtheorem*{previouslyseen}{As previously seen}
\newtheorem*{remark}{Remark}
\newtheorem*{note}{Note}
\newtheorem*{problem}{Problem}
\newtheorem*{observe}{Observe}
\newtheorem*{property}{Property}
\newtheorem*{intuition}{Intuition}
\newmdtheoremenv[nobreak=true]{prop}{Proposition}
\newmdtheoremenv[nobreak=true]{theorem}{Theorem}
\newmdtheoremenv[nobreak=true]{corollary}{Corollary}

% End example and intermezzo environments with a small diamond (just like proof
% environments end with a small square)
\usepackage{etoolbox}
\AtEndEnvironment{vb}{\null\hfill$\diamond$}%
\AtEndEnvironment{intermezzo}{\null\hfill$\diamond$}%
% \AtEndEnvironment{opmerking}{\null\hfill$\diamond$}%

% Fix some spacing
% http://tex.stackexchange.com/questions/22119/how-can-i-change-the-spacing-before-theorems-with-amsthm
\makeatletter
\def\thm@space@setup{%
  \thm@preskip=\parskip \thm@postskip=0pt
}


% Exercise 
% Usage:
% \oefening{5}
% \suboefening{1}
% \suboefening{2}
% \suboefening{3}
% gives
% Oefening 5
%   Oefening 5.1
%   Oefening 5.2
%   Oefening 5.3
\newcommand{\oefening}[1]{%
    \def\@oefening{#1}%
    \subsection*{Oefening #1}
}

\newcommand{\suboefening}[1]{%
    \subsubsection*{Oefening \@oefening.#1}
}


% \lecture starts a new lecture (les in dutch)
%
% Usage:
% \lecture{1}{di 12 feb 2019 16:00}{Inleiding}
%
% This adds a section heading with the number / title of the lecture and a
% margin paragraph with the date.

% I use \dateparts here to hide the year (2019). This way, I can easily parse
% the date of each lecture unambiguously while still having a human-friendly
% short format printed to the pdf.

\usepackage{xifthen}
\def\testdateparts#1{\dateparts#1\relax}
\def\dateparts#1 #2 #3 #4 #5\relax{
    \marginpar{\small\textsf{\mbox{#1 #2 #3 #5}}}
}

\def\@lecture{}%
\newcommand{\lecture}[3]{
    \ifthenelse{\isempty{#3}}{%
        \def\@lecture{Lecture #1}%
    }{%
        \def\@lecture{Lecture #1: #3}%
    }%
    \subsection*{\@lecture}
    \marginpar{\small\textsf{\mbox{#2}}}
}



% These are the fancy headers
\usepackage{fancyhdr}
\pagestyle{fancy}

% LE: left even
% RO: right odd
% CE, CO: center even, center odd
% My name for when I print my lecture notes to use for an open book exam.
% \fancyhead[LE,RO]{Gilles Castel}

\fancyhead[RO,LE]{\@lecture} % Right odd,  Left even
\fancyhead[RE,LO]{}          % Right even, Left odd

\fancyfoot[RO,LE]{\thepage}  % Right odd,  Left even
\fancyfoot[RE,LO]{}          % Right even, Left odd
\fancyfoot[C]{\leftmark}     % Center

\makeatother




% Todonotes and inline notes in fancy boxes
\usepackage{todonotes}
\usepackage{tcolorbox}

% Make boxes breakable
\tcbuselibrary{breakable}

% Verbetering is correction in Dutch
% Usage: 
% \begin{verbetering}
%     Lorem ipsum dolor sit amet, consetetur sadipscing elitr, sed diam nonumy eirmod
%     tempor invidunt ut labore et dolore magna aliquyam erat, sed diam voluptua. At
%     vero eos et accusam et justo duo dolores et ea rebum. Stet clita kasd gubergren,
%     no sea takimata sanctus est Lorem ipsum dolor sit amet.
% \end{verbetering}
\newenvironment{verbetering}{\begin{tcolorbox}[
    arc=0mm,
    colback=white,
    colframe=green!60!black,
    title=Opmerking,
    fonttitle=\sffamily,
    breakable
]}{\end{tcolorbox}}

% Noot is note in Dutch. Same as 'verbetering' but color of box is different
\newenvironment{noot}[1]{\begin{tcolorbox}[
    arc=0mm,
    colback=white,
    colframe=white!60!black,
    title=#1,
    fonttitle=\sffamily,
    breakable
]}{\end{tcolorbox}}




% Figure support as explained in my blog post.
\usepackage{import}
\usepackage{xifthen}
\usepackage{pdfpages}
\usepackage{transparent}
\newcommand{\incfig}[1]{%
    \def\svgwidth{\columnwidth}
    \import{./figures/}{#1.pdf_tex}
}

% Fix some stuff
% %http://tex.stackexchange.com/questions/76273/multiple-pdfs-with-page-group-included-in-a-single-page-warning
\pdfsuppresswarningpagegroup=1

\title{\Huge{Probability 1}\\HW 3}
\author{\huge{Daniel Yu}}
\date{September 27, 2024}

\pdfsuppresswarningpagegroup=1

\begin{document}
\maketitle
\newpage% or \cleardoublepage
% \pdfbookmark[<level>]{<title>}{<dest>}
\pagebreak

\begin{enumerate}
  \item Let $E_n$ be the number of empty bins. Show that  
    \[
      \lim_{n \to \infty} \frac{1}{n} E[E_n]
    .\] exists and compute as function of $c$.
    \begin{proof}
      We know from the previous homework that the probability of $k$ empty bins is 
       \[
         P[E_n = k] = \frac{(n-k)^{r}}{n^{r}}
      .\] 
      Alternatively, consider $E_n = B_1 + B_2 + \ldots + B_n$ where $B_i$ is a random variable representing if the  $i$th bin is empty. Since expected value is linear, 
      \begin{align*}
        E[E_n] &= E[B_1] + E[B_2] + \ldots + E[B_n] \\
               &= \sum_{i=1}^{n} E[B_i] \\
               &= \sum_{i=1}^{n} [0 + 1 \cdot P[B_i]=1 ] \\
               &= n \frac{(n-1)^{r}}{n^{r}}
      .\end{align*}
      As $n, r$ go to infinity as  $\frac{r}{n} \to c \iff r \to cn$. 
      \begin{align*}
        E[E_n] &= n \frac{(n-1)^{cn}}{n^{cn}}
      .\end{align*}
      Then,
      \begin{align*}
        \lim_{n \to \infty} \frac{1}{n} E[E_n] &= \lim_{n \to \infty} \frac{1}{n} n \frac{(n-1)^{cn}}{n^{cn}} \\
                                               &= \lim_{n \to \infty} \frac{(n-1)^{cn}}{n^{cn}}\\
                                               &= \lim_{n \to \infty} (\frac{n-1}{n})^{cn} \\
                                               &= \lim_{n \to \infty} \exp(n \ln(1 - \frac{1}{n}))^{c} \\
                                               &= \lim_{n \to \infty} \exp(n \cdot -\frac{1}{n})^{c} \\
                                               &= \lim_{n \to \infty} \exp(-1)^{c} \\
                                               &= e^{-c}
      .\end{align*} 
    \end{proof}
  \item Let $N$ be the number of rolls required for a fair six sided die to have the same number show up twice in a row. Find expected value of $N$
    \begin{proof}
      Denote $G_i$ as the random variable  representing the number of rolls until the first $i$th value shows up. Since  $G_i \sim Geo(\frac{1}{6})$, so $E[G_i] = \frac{1}{6}$. By the law of iterated expectations, we know that 
\[
  E[N] = E_{G_i}[E_N[N|G_i]]      
.\] 
Now, consider $X_{G+1}$ the result of the roll after $G_i$. Condition $E[N|G_i]$ on  $X_{G+1}$.
\begin{align*}
  & E[N|G_i, X_{G+1} = i] = G + 1 \\
  & E[N|G_i, X_{G+1} \neq 1] = G + 1 + E[N] \text{ ,we reset the experiment with $X_{G+1}$ as the new $X_1$} \\
.\end{align*}
This gives,
\begin{align*}
  E[N|G_i] &= E[N|G_i, X_{G+1} = i] \cdot P[X_{G+1} = i] + E[N|G_i, X_{G+1} \neq i] \cdot P[X_{G+1} \neq i] \\
           &= (G+1) \cdot \frac{1}{6} + (G+1 + E[N]) \cdot \frac{5}{6} \\
           &= G+1 + \frac{5}{6}E[N]
.\end{align*}
Backsubsituting for $E[N]$:
 \begin{align*}
   E[N] = E_{G_i}[E_N[N | G_i]] &= E_{G_i}[G+1 + \frac{5}{6}E[N]] \\
                                &= 1 + 1 + \frac{5}{6} E[N] \\
                                &= 2 + \frac{5}{6}E[N] \\
   \frac{1}{6}E[N] &= 2 \\
                   &= 12
.\end{align*}
The expected value of $N$ is 12.
    \end{proof}
  \item For a variant of the Monty Hall problem, should the contestant switch screens?
    \begin{proof}
      Consider the following setup. There are 5 screens, $A,B,C,D,E,F$ and without loss of generality assume $A,B$ have prizes and  $D,E,F$ have goats. If the contestant picks  $x \in \{A,B\}$ then Monty Hall has to open the one of $\{A,B\} $ which the contestant did not choose. If the contestant switches, then he gets a goat. If the contestant stays, then he gets the prize. \\


      Alternatively, if the contestant picks $x \in \{D,E,F\} $, then Monty Hall has to open $y \in \{X,Y\}$. If the contestant stays, he loses, but if the contestant switches then there is a $\frac{1}{3}$ chance of getting a prize. \\


      Now, let's got back to the setup. Assuming the contestant uniformly chooses one of $\{A,B,C,D,E,F\} $ uniformly then if he doesn't switch, $P[\text{win} | \text{No switch}, x \in \{A,B\} ] = 1$,  $P[\text{win} | \text{switch}, x \in \{A,B\} ] = 0$ and $P[x \in \{A,B\}] = \frac{2}{5}$ while $P[\text{win} | \text{No switch}, x \in \{D,E,F\} ] = 0$ and $P[\text{win} | \text{Switch}, x \in \{D,E,F\}] = \frac{1}{3}$, $P[x \in \{D,E,F\}] $. Then

\begin{align*}
  P[\text{win}|\text{switch}] &= P[\text{win} | \text{switch}, x \in \{A,B\} ] + P[\text{win} | \text{switch}, x \in \{D,E,F\} ] \\ 
                              &= 0 \cdot \frac{2}{5} + \frac{1}{3} \cdot \frac{3}{5} \\
                              &= \frac{1}{5}
.\end{align*}
and,
\begin{align*}
  P[\text{win} | \text{no switch}] &= P[\text{win} | \text{no switch}, x \in \{A,B\} ] + P[\text{win} | \text{no switch}, x \in \{D,E,F\} ] \\ 
                                   &= 1 \cdot \frac{2}{5} + 0 \cdot \frac{3}{5}\\
                                   &= \frac{2}{5}
.\end{align*}
Thus, the contestant should not switch as $P[\text{win} | \text{no switch}] = \frac{2}{5} > P[\text{win} | \text{switch}] = \frac{1}{5}$
    \end{proof}
  \item Let $p,q$ be two real numbers in  $(0,1)$. Let  $V \sim Geo(p)$ be number of people that visit a store in a given day. Let  $q$ be the probability each customer buys a bar. Each customer is independent.
     \begin{enumerate}
       \item What is the expected number of chocolate bars sold in a day?
         \begin{proof}
           Let $C$ be the number of chocolate bars sold in a day.  Represent each customer as an independent identically distributed random variable $X_i$ representing whether or not they bought a chocolate bar. Then,  $P[x_i = 1] = q$  and $P[X_i = 0]=1-q$. Then, $C = \sum_{i=1}^{V} X_i$ ad $P[C= c| V] = \begin{pmatrix} V\\ c \end{pmatrix} q^{c}(1-q)^{V-c}$ .
           \begin{align*}
             E[C] = E_v[E_c[C|V]] &= E_V[\sum_{k=1}^{v} k \cdot P[C=k|V=v]]  \\
                                  &= E_V[\sum_{k=1}^{v} \begin{pmatrix} v\\ k \end{pmatrix} q^{k}(1-q)^{v-k}] \\
                                  &= \sum_{v=1}^{\infty} [\sum_{k=1}^{v}  \begin{pmatrix} v\\ k \end{pmatrix} q^{k}(1-q)^{v-k}] \cdot P[V=v] \\
                                  &= \sum_{k=1}^{v} q^{k}  \sum_{v=1}^{\infty} \begin{pmatrix} v\\ k \end{pmatrix} (1-q)^{v-k} \cdot P[V=v] \\
           .\end{align*}
           \begin{align*}
             E[C] = E_v[E_C[C|V]] &= E_v[E_C[bin(V,q)]] \\
                                  &= E_v[V \cdot q] \\
                                  &= q \cdot E_v[V] \\
                                  &= q \cdot \frac{1}{p} \\
                                  &= \frac{q}{p}
           .\end{align*}
         \end{proof}
       \item What is the probability that the number of chocolate bars sold is equal to the number of customers that visited the store on a particular day?
         \begin{proof}
           This is asking $P[C=V|V] = \begin{pmatrix} V \\ V \end{pmatrix} \cdot q^{V} = q^{V}$. So we are considering, $P[C=V]$. 
            \begin{align*}
              P[C=V] &= \sum_{k}^{\infty} P[C = k \cap V=k] \\
                     &= \sum_{k}^{\infty} P[C=k | V=k] \cdot P[V=k] \\
                     &= \sum_{k}^{\infty} q^{k} \cdot P[V=k]\\
                     &= \sum_{k}^{\infty} q^{k} \left( 1-p \right)^{k-1} p \\
                     &= qp \sum_{k}^{\infty}  q^{k-1} \left( 1-p \right)^{k-1} \\
                     &= qp \sum_{k}^{\infty} (q (1-p))^{k-1}  \\
                     &= \frac{qp}{1-q(1-p)}
           .\end{align*}
           As $k \to \infty$,  $P[C=V] \to 0$ because $q,p < 0$
         \end{proof}
    \end{enumerate}
  \item Let $X,Y$ be two independent exponential random variables with  $\lambda = 1$. Conditional on  $X,Y$, let  $Z$ be a uniform random variable on  $[-X,Y]$ what is the mean and variance of  $Z$?
    \begin{proof}
      Define the distribution of $Z$ as:
       \[
         P[Z=z] = \begin{cases}
        \frac{1}{Y+X}, $-X \leq z \leq Y$ \\
        0, \text{ otherwise}
      \end{cases}
      .\]
      Substituting $e^{-t}$ for $X,Y$:
       \[
         P[Z=z] = \begin{cases}
           \frac{1}{2e^{-t}}, z \in [-e^{-t}, e^{t}], t \geq 0 \\
           0, \text{ otherwise}
         \end{cases}
      .\] 
      The cdf of $Z$ is:
      \[
      F_Z(z) = \begin{cases}
        \frac{z + e^{-t}}{2e^{-t}}, z \in [-e^{-t}, e^{t}], t \geq 0 \\
        0, \text{ otherwise}
      \end{cases}
      .\] 
      The pdf of $Z$ is:
       \[
         f_Z(z) = \frac{d}{dz}F_Z(z) = \frac{d}{dz} [\frac{z}{2e^{-t}} + \frac{e^{-t}}{2e^{-t}}] = \frac{1}{2e^{-t}}
      .\] 
      In this case, as expected the expected value is the center of the interval, 0.    
      \begin{align*}
        E[Z]    &= \int_{-\infty}^{\infty} z \cdot \frac{1}{2e^{-t}} dz \\
                &= \int_{-e^{-t}}^{e^{-t}} z \cdot \frac{1}{2e^{-t}} dz \\
                &= \frac{1}{2}  [\frac{z^{2}}{2e^{-t}}]_{-e^{-t}}^{e^{-t}} \\
                &= \frac{1}{2} [\frac{e^{-2t}}{2e^{-t}} - \frac{e^{-2t}}{2e^{-t}}] \\
                &= 0 
      .\end{align*}
      The variance of $Z$:
       \begin{align*}
         Var(Z) &= E[Z^{2}] - E[Z]^{2} \\
                &= \int_{-\infty}^{\infty} z^{2} \cdot \frac{1}{2e^{-t}} dz - 0 \\
                &= \int_{-e^{-t}}^{e^{-t}} z^{2} \cdot \frac{1}{2e^{-t}} dz \\
                &= \frac{1}{3}  [\frac{z^{3}}{2e^{-t}}]_{-e^{-t}}^{e^{-t}} \\
                &= \frac{1}{3} [\frac{e^{-3t}}{2e^{-t}} - \frac{-e^{-3t}}{2e^{-t}}] \\
                &= \frac{1}{3} \cdot \frac{e^{-3t}}{e^{-t}} \\
                &= \frac{1}{3} e^{-2t}
       .\end{align*} 
    \end{proof}
\end{enumerate}
  
\end{document}
