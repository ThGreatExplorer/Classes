\documentclass[a4paper]{article}
\usepackage[a4paper, margin=1in]{geometry}
% Some basic packages
\usepackage[utf8]{inputenc}
\usepackage[T1]{fontenc}
\usepackage{textcomp}
\usepackage[dutch]{babel}
\usepackage{url}
\usepackage{graphicx}
\usepackage{float}
\usepackage{booktabs}
\usepackage{enumitem}

\pdfminorversion=7

% Don't indent paragraphs, leave some space between them
\usepackage{parskip}

% Hide page number when page is empty
\usepackage{emptypage}
\usepackage{subcaption}
\usepackage{multicol}
\usepackage{xcolor}

% Other font I sometimes use.
% \usepackage{cmbright}

% Math stuff
\usepackage{amsmath, amsfonts, mathtools, amsthm, amssymb}
% Fancy script capitals
\usepackage{mathrsfs}
\usepackage{cancel}
% Bold math
\usepackage{bm}
% Some shortcuts
\newcommand\N{\ensuremath{\mathbb{N}}}
\newcommand\R{\ensuremath{\mathbb{R}}}
\newcommand\Z{\ensuremath{\mathbb{Z}}}
\renewcommand\O{\ensuremath{\emptyset}}
\newcommand\Q{\ensuremath{\mathbb{Q}}}
\newcommand\C{\ensuremath{\mathbb{C}}}

% Easily typeset systems of equations (French package)
\usepackage{systeme}

% Put x \to \infty below \lim
\let\svlim\lim\def\lim{\svlim\limits}

%Make implies and impliedby shorter
\let\implies\Rightarrow
\let\impliedby\Leftarrow
\let\iff\Leftrightarrow
\let\epsilon\varepsilon

% Add \contra symbol to denote contradiction
\usepackage{stmaryrd} % for \lightning
\newcommand\contra{\scalebox{1.5}{$\lightning$}}

% \let\phi\varphi

% Command for short corrections
% Usage: 1+1=\correct{3}{2}

\definecolor{correct}{HTML}{009900}
\newcommand\correct[2]{\ensuremath{\:}{\color{red}{#1}}\ensuremath{\to }{\color{correct}{#2}}\ensuremath{\:}}
\newcommand\green[1]{{\color{correct}{#1}}}

% horizontal rule
\newcommand\hr{
    \noindent\rule[0.5ex]{\linewidth}{0.5pt}
}

% hide parts
\newcommand\hide[1]{}

% si unitx
\usepackage{siunitx}
\sisetup{locale = FR}

% Environments
\makeatother
% For box around Definition, Theorem, \ldots
\usepackage{mdframed}
\mdfsetup{skipabove=1em,skipbelow=0em}
\theoremstyle{definition}
\newmdtheoremenv[nobreak=true]{definitie}{Definitie}
\newmdtheoremenv[nobreak=true]{eigenschap}{Eigenschap}
\newmdtheoremenv[nobreak=true]{gevolg}{Gevolg}
\newmdtheoremenv[nobreak=true]{lemma}{Lemma}
\newmdtheoremenv[nobreak=true]{propositie}{Propositie}
\newmdtheoremenv[nobreak=true]{stelling}{Stelling}
\newmdtheoremenv[nobreak=true]{wet}{Wet}
\newmdtheoremenv[nobreak=true]{postulaat}{Postulaat}
\newmdtheoremenv{conclusie}{Conclusie}
\newmdtheoremenv{toemaatje}{Toemaatje}
\newmdtheoremenv{vermoeden}{Vermoeden}
\newtheorem*{herhaling}{Herhaling}
\newtheorem*{intermezzo}{Intermezzo}
\newtheorem*{notatie}{Notatie}
\newtheorem*{observatie}{Observatie}
\newtheorem*{oef}{Oefening}
\newtheorem*{opmerking}{Opmerking}
\newtheorem*{praktisch}{Praktisch}
\newtheorem*{probleem}{Probleem}
\newtheorem*{terminologie}{Terminologie}
\newtheorem*{toepassing}{Toepassing}
\newtheorem*{uovt}{UOVT}
\newtheorem*{vb}{Voorbeeld}
\newtheorem*{vraag}{Vraag}

\newmdtheoremenv[nobreak=true]{definition}{Definition}
\newtheorem*{eg}{Example}
\newtheorem*{notation}{Notation}
\newtheorem*{previouslyseen}{As previously seen}
\newtheorem*{remark}{Remark}
\newtheorem*{note}{Note}
\newtheorem*{problem}{Problem}
\newtheorem*{observe}{Observe}
\newtheorem*{property}{Property}
\newtheorem*{intuition}{Intuition}
\newmdtheoremenv[nobreak=true]{prop}{Proposition}
\newmdtheoremenv[nobreak=true]{theorem}{Theorem}
\newmdtheoremenv[nobreak=true]{corollary}{Corollary}

% End example and intermezzo environments with a small diamond (just like proof
% environments end with a small square)
\usepackage{etoolbox}
\AtEndEnvironment{vb}{\null\hfill$\diamond$}%
\AtEndEnvironment{intermezzo}{\null\hfill$\diamond$}%
% \AtEndEnvironment{opmerking}{\null\hfill$\diamond$}%

% Fix some spacing
% http://tex.stackexchange.com/questions/22119/how-can-i-change-the-spacing-before-theorems-with-amsthm
\makeatletter
\def\thm@space@setup{%
  \thm@preskip=\parskip \thm@postskip=0pt
}


% Exercise 
% Usage:
% \oefening{5}
% \suboefening{1}
% \suboefening{2}
% \suboefening{3}
% gives
% Oefening 5
%   Oefening 5.1
%   Oefening 5.2
%   Oefening 5.3
\newcommand{\oefening}[1]{%
    \def\@oefening{#1}%
    \subsection*{Oefening #1}
}

\newcommand{\suboefening}[1]{%
    \subsubsection*{Oefening \@oefening.#1}
}


% \lecture starts a new lecture (les in dutch)
%
% Usage:
% \lecture{1}{di 12 feb 2019 16:00}{Inleiding}
%
% This adds a section heading with the number / title of the lecture and a
% margin paragraph with the date.

% I use \dateparts here to hide the year (2019). This way, I can easily parse
% the date of each lecture unambiguously while still having a human-friendly
% short format printed to the pdf.

\usepackage{xifthen}
\def\testdateparts#1{\dateparts#1\relax}
\def\dateparts#1 #2 #3 #4 #5\relax{
    \marginpar{\small\textsf{\mbox{#1 #2 #3 #5}}}
}

\def\@lecture{}%
\newcommand{\lecture}[3]{
    \ifthenelse{\isempty{#3}}{%
        \def\@lecture{Lecture #1}%
    }{%
        \def\@lecture{Lecture #1: #3}%
    }%
    \subsection*{\@lecture}
    \marginpar{\small\textsf{\mbox{#2}}}
}



% These are the fancy headers
\usepackage{fancyhdr}
\pagestyle{fancy}

% LE: left even
% RO: right odd
% CE, CO: center even, center odd
% My name for when I print my lecture notes to use for an open book exam.
% \fancyhead[LE,RO]{Gilles Castel}

\fancyhead[RO,LE]{\@lecture} % Right odd,  Left even
\fancyhead[RE,LO]{}          % Right even, Left odd

\fancyfoot[RO,LE]{\thepage}  % Right odd,  Left even
\fancyfoot[RE,LO]{}          % Right even, Left odd
\fancyfoot[C]{\leftmark}     % Center

\makeatother




% Todonotes and inline notes in fancy boxes
\usepackage{todonotes}
\usepackage{tcolorbox}

% Make boxes breakable
\tcbuselibrary{breakable}

% Verbetering is correction in Dutch
% Usage: 
% \begin{verbetering}
%     Lorem ipsum dolor sit amet, consetetur sadipscing elitr, sed diam nonumy eirmod
%     tempor invidunt ut labore et dolore magna aliquyam erat, sed diam voluptua. At
%     vero eos et accusam et justo duo dolores et ea rebum. Stet clita kasd gubergren,
%     no sea takimata sanctus est Lorem ipsum dolor sit amet.
% \end{verbetering}
\newenvironment{verbetering}{\begin{tcolorbox}[
    arc=0mm,
    colback=white,
    colframe=green!60!black,
    title=Opmerking,
    fonttitle=\sffamily,
    breakable
]}{\end{tcolorbox}}

% Noot is note in Dutch. Same as 'verbetering' but color of box is different
\newenvironment{noot}[1]{\begin{tcolorbox}[
    arc=0mm,
    colback=white,
    colframe=white!60!black,
    title=#1,
    fonttitle=\sffamily,
    breakable
]}{\end{tcolorbox}}




% Figure support as explained in my blog post.
\usepackage{import}
\usepackage{xifthen}
\usepackage{pdfpages}
\usepackage{transparent}
\newcommand{\incfig}[1]{%
    \def\svgwidth{\columnwidth}
    \import{./figures/}{#1.pdf_tex}
}

% Fix some stuff
% %http://tex.stackexchange.com/questions/76273/multiple-pdfs-with-page-group-included-in-a-single-page-warning
\pdfsuppresswarningpagegroup=1

\title{\Huge{Some Class}}
\author{\huge{Daniel Yu}}
\date{}

\pdfsuppresswarningpagegroup=1

\begin{document}
\maketitle
\newpage% or \cleardoublepage
% \pdfbookmark[<level>]{<title>}{<dest>}
\tableofcontents
\pagebreak
\section{Poisson Distribution}
\begin{note}{Balls to Bins} \\
  Given this setup: there are two bins with $n$ number of balls and the balls are distributed uniformly at random across the 2 bins. Let's say that 
   \[
     P[N=2] = \frac{1}{2}, P[N=3] = \frac{1}{2}
   .\] 
Let the random variables $X$= number of balls in first bin and  $Y$= number of balls in the second bin. What is the distribution of  $X$: 
\begin{proof}Consider the conditional probability \\
  \begin{enumerate}
    \item Distribution of $X$ given  $N=2$:  $X \sim Bin(2, .5)$
    \item Distribution of  $X$ give  $N =3$:  $X \sim Bin(3, .5)$
  \end{enumerate}
  Use the law of total probability to construct the probability of $X$ from the conditional probabilities of  $X$:
   \[
     P[X=0] = P[N=2] \cdot P[X=0|N=2] + P[N=2] \cdot P[X=0|N=2] = \frac{1}{2} \cdot \frac{1}{2}^{2} + \frac{1}{2} \frac{1}{2}^{3} = \frac{3}{16} 
  .\] 
  \[
    P[X=1] = \frac{1}{2} \cdot \begin{pmatrix} 2 \\ 1 \end{pmatrix} \frac{1}{2}^{2} + \frac{1}{2} \cdot \begin{pmatrix} 3 \\ 1\end{pmatrix} \cdot \frac{1}{2}^{3} = \frac{1}{4} + \frac{3}{16} = \frac{7}{16}
  .\] 
\[
  P[X=2] = \frac{5}{16}
.\] 
\[
  P[X=3] = \frac{1}{16}
.\]
Note that while the conditional probabilities are binomial, the combined probabilities is not binomial
\end{proof}
What about the distirbution of $Y$?
\begin{proof}
  $Y \sim X$  
\end{proof}
Are $X,Y$ independent?
\begin{proof}
  Consider $P[X=0,Y=0]=0$ and  $P[X=0] > 0 $,  $P[Y = 0] > 0$   
\end{proof}
\end{note}

\begin{remark}
  However there is a special case where the above situation would have $X,Y$ independent.
\end{remark}

\begin{definition}
  $N \sim Poisson(\lambda)$,  $\lambda \geq 0$.
   \[
     P[N=k] = \frac{e^{-\lambda}\lambda^{k}}{k!}
  .\] 
\end{definition}

\begin{remark}{Intuition}\\
Let's repeat the experiment with this distribution for $N$.
\begin{align*}
  P[X=r] &= \sum_{n=0}^{\infty} P[N=n] \cdot P[X=r|N=n] \\
         &= \sum_{n=0}^{\infty} P[N=n] \cdot P[Bin(n,\frac{1}{2})=r] \\
         & \text{ if n > r, this is impossible} \\
         &= \sum_{n=r}^{\infty} P[N=n] \cdot P[Bin(n, \frac{1}{2}) = r] \\
         &= \sum_{n=r}^{\infty} (\frac{e^{-\lambda} \lambda^{n}}{n!}) [\begin{pmatrix} n \\ r \end{pmatrix} \frac{1}{2}^{n}] \\
         &= \frac{e^{-\lambda}}{r!} \sum_{n=r}^{\infty} - \frac{\frac{\lambda}{2}^{n}}{(n-r)!} \\
         &\text{ set m = n -r} \\
         &=  \frac{e^{-\lambda}}{r!} \sum_{m=0}^{\infty} \frac{\frac{\lambda}{2}^{m+r}}{m!} \\
         &=   \frac{e^{-\lambda}}{r!} \frac{\lambda}{2}^r \sum_{m=0}^{\infty} \frac{\frac{\lambda}{2}^{m}}{m!}  \\
         &\text{ since $e^{x} = \sum_{m=0}^{\infty} \frac{x^{m}}{m!}$ } \\
         &= \frac{e^{-\frac{\lambda}{2}} \cdot \left( \frac{\lambda}{2}^{n} \right) }{r!} \\
         &=\text{ Poisson($\frac{\lambda}{2})$}
.\end{align*}

Similarly, $Y \sim Pos(\frac{\lambda}{2})$. Now, the claim is that for any $s,r$,  $P[X=s, Y=r] = P[X=s] \cdot P[Y=r]$ that is  $X,Y$ independent.
 \begin{align*}
   P[X=s, Y=r] &= \frac{e^{-\frac{\lambda}{2}} \cdot \left( \frac{\lambda}{2}^{n} \right) }{r!} \cdot \frac{e^{-\frac{\lambda}{2}} \cdot \left( \frac{\lambda}{2}^{n} \right) }{s!} \\
                &=  \frac{e^{-\frac{\lambda}{2}} \cdot \left( \frac{\lambda}{2}^{n} \right) }{s! r!}  \\
   P[X=s, Y=r] &= P[N = r + s] \cdot P[X=s, Y=r | N = r+ s] \\
               &= \frac{e^{-\frac{\lambda}{2}} \cdot \left( \frac{\lambda}{2}^{n} \right) }{(r+s)!} \cdot P[X=s, Y=r | N = r+s] \\
               &= \frac{e^{-\frac{\lambda}{2}} \cdot \left( \frac{\lambda}{2}^{n} \right) }{(r+s)!} \cdot P[X=s | N=r+s] \\
               &= \frac{e^{-\frac{\lambda}{2}} \cdot \left( \frac{\lambda}{2}^{n} \right) }{(r+s)!} \cdot P[bin(r+s, \frac{1}{2})=s] \\
               &= \frac{e^{-\frac{\lambda}{2}} \cdot \left( \frac{\lambda}{2}^{n} \right) }{(r+s)!}  \cdot [\begin{pmatrix} r+s \\ s \end{pmatrix} \cdot \left( \frac{1}{2} \right)^{r+s} ] \\
               &= \frac{e^{-\lambda} \left( \frac{\lambda}{2}^{r+s} \right) }{r!s!}
.\end{align*} 
\end{remark}

\section{Exponential Distribution}
  \begin{definition}
    $X \sim \exp(\lambda)$ if the pdf of $X$, 
     \[
    f_{x}(t) = \begin{cases}
      &\lambda e^{-\lambda t}, t \geq 0 \\
      & 0, t < 0
    \end{cases}
    .\] 
    Let $W \sim \exp(\lambda)$ the waiting between trains.
  \end{definition}
\begin{note}{Example}\\
  I arrived at the platform. Let $W'$ be the amount of time I  waiting between trains. I arrive to the platform.  Let's suppose that the wat has $2$ values $:$ Long/short.  $S=5, L=15$. Suppose you are told that 5 seconds have elapsed since the arrival of the previous train. Then $w' \sim w - s$ conditioned upon  $w > s$. Let's understand,
   \begin{align*}
     P[w' > t] &= P[W > t +s \mid  W > s] \\
               &= \frac{P[W > t + s \cap W > s]}{P[W > s]} \\
               &= \frac{P[W > t +s]}{P[W > s]} \\
               &= \frac{e^{- \lambda (t-s)}}{e^{-\lambda s}} \\
               &= e^{\lambda t}
  .\end{align*}
  So $w' \sim \exp(\lambda)$
\end{note}

\section{Indepence and Identically Distributed}
i.e. (iid) Random Variables

\begin{note}{IID Important Example}\\
Suppose $\{X_i\}_{i=1}^{n}$ is a sequence of i.i.d random variables s.t. $P[X_i=1] = P[X_i=2] = P[X_i=3] = P[X_i=4]$. Let $k_i(n) = \mid \{j: x_j = i\} \mid $ = number of times $i$ comes out. What do you expect  $k_i(n)$ to be if  $n$ is large?
 \[
k_i(n) \sim \frac{1}{4} n
.\] 
Hence, $P[k_1(n) = n] = \frac{1}{2}^{n} > 0$ (all $1$s for each  $n$ rolls). As we will show later,
\[
  P[k_1(n) = \frac{n}{4}] \approx \frac{1}{\sqrt{\pi} n } 
.\] 
\end{note}

\begin{definition}
  Let $\{X_n\} $ be a sequence of random variables. We say that \textbf{$X_n$ converges in probability} to constant  $c$ if  $\forall \epsilon > 0$,
   \[
     \lim_{n \to \infty} P[\mid X_n - c \mid > \epsilon] = 0
  .\] 
\end{definition}

\begin{note}{Example}\\
  $G_n = Geo(\frac{n-1}{n})$, $P[G_n = k] = (\frac{n-1}{n}) \cdot (\frac{1}{n})^{k-1}$ to show that $G_n \to 1$ in probability.

\begin{proof}
\[
  \forall \epsilon > 0, P[|G_n - 1| > \epsilon] 
.\]  is small. as events $\{\mid G_n - 1 \mid > \epsilon\} \subseteq \{G_n \geq 2\}$ (by set theory). So 
\[
  P[\mid G_n - 1 \mid > \epsilon] \subseteq P[G_n \geq 2] = \sum_{k=2}^{\infty} \left( \frac{n-1}{n} \right) \left( \frac{1}{n}^{k-1} \right) = \frac{1}{n}
.\] 
\[
  \lim_{n \to \infty} P[\mid G_n -1 \mid \geq \epsilon] \subseteq \lim_{n \to \infty} \frac{1}{n} = 0
.\] 
\end{proof}
\end{note}

\begin{note}{Example}\\
  $U_n$ is a uniform RV on  $[-\frac{1}{n}, \frac{1}{n}]$. Claim $U_n \to 0$ in probability. 
  \begin{proof}
    \begin{align*}
      P[|U_n| > \epsilon] \to 0 \iff P[\mid U_n\mid  \leq \epsilon] &= \int_{-\epsilon}^{\epsilon} f_{U_n}(t) dt \\
                                                                     &= \int_{-\epsilon}^{\epsilon} (\frac{n}{2}) 1|_{t \in [-\frac{1}{n}, \frac{1}{n}]} dt \\
                                                                     &= 1
    .\end{align*}
    Why? if $n > \frac{1}{\epsilon}$ then $[-\frac{1}{n}, \frac{1}{n}] \subseteq (-\epsilon, \epsilon)$.
  \end{proof}
\end{note}

\subsection{Fat Tail}
incredibly rare event that affects the average outcome
\begin{note}{Example}
  Let $P[R_n = 0] = 1 - \frac{1}{n}$, $P[R_n =n^2] = \frac{1}{n}$.
  \begin{proof}
    \[
      P[|R_n| > \epsilon] = P[R_n =n^2] = \frac{1}{n} \to 0
    .\] 
    but,
    \[
      E[R_n] = 0 \cdot (1-\frac{1}{n}) + n^2 \cdot \frac{1}{n} = n \to \infty
    .\] 
  \end{proof}
\end{note}

Saint Petersburg Problem: https://www.wikiwand.com/en/articles/St._Petersburg_paradox
\end{document}
