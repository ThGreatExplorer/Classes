\documentclass[a4paper]{article}
\input{preamble}
\title{\Huge{Some Class}}
\author{\huge{Daniel Yu}}
\date{}

\begin{document}
\maketitle
\newpage% or \cleardoublepage
% \pdfbookmark[<level>]{<title>}{<dest>}
\tableofcontents
\pagebreak

\section{Reference}
https://rajagopalvenkat.com/teaching/resources/AI/ch2.html

Consider search problems. They often fall into several categories based on the algorithm and purpose. The one I'm most familiar with is: 
\[
  \text{Optimal Search}
.\] 

\begin{definition}
  Searching is the process of navigating and often narrowing the state space to intelligently find a solution.
\end{definition}

\section{Search}
\begin{enum}
  \item Breath First Search
  \item Depth First Search
  \item Iterative Deepening
  \item Uniform Search Cost -- Special case of dijistra's algorithm
\end{enum}

https://rajagopalvenkat.com/teaching/resources/AI/ch2.html

\section{Informed Search}
\begin{definition}
  Informed search relies on heuristics to estimate the cost of the cheapest path from the agent to the goal instead of cumulative cost. This is most useful when
  heuristics are inexpesive and have constant time lookup.
\end{definition}

Take the example of the the 8 tile game. One heuristic that could be considered is using the number of 
tiles out of place to estimate how far the state space is from goal state. However this does have problems, 
consider the left game where there is only 1 piece out of place but the configuration is unplayable 
as compared to the right game which has more pieces out of place but completely solvable.
\begin{figure}[h]
  \centering
  \includegraphics[width=0.8\textwidth]{assets/2024-09-12-09-45-42.png}
  \caption{8 Tile}
  \label{fig:2024-09-12-09-45-42}
\end{figure}
\end{document}
