\documentclass[a4paper]{article}
% Some basic packages
\usepackage[utf8]{inputenc}
\usepackage[T1]{fontenc}
\usepackage{textcomp}
\usepackage[dutch]{babel}
\usepackage{url}
\usepackage{graphicx}
\usepackage{float}
\usepackage{booktabs}
\usepackage{enumitem}

\pdfminorversion=7

% Don't indent paragraphs, leave some space between them
\usepackage{parskip}

% Hide page number when page is empty
\usepackage{emptypage}
\usepackage{subcaption}
\usepackage{multicol}
\usepackage{xcolor}

% Other font I sometimes use.
% \usepackage{cmbright}

% Math stuff
\usepackage{amsmath, amsfonts, mathtools, amsthm, amssymb}
% Fancy script capitals
\usepackage{mathrsfs}
\usepackage{cancel}
% Bold math
\usepackage{bm}
% Some shortcuts
\newcommand\N{\ensuremath{\mathbb{N}}}
\newcommand\R{\ensuremath{\mathbb{R}}}
\newcommand\Z{\ensuremath{\mathbb{Z}}}
\renewcommand\O{\ensuremath{\emptyset}}
\newcommand\Q{\ensuremath{\mathbb{Q}}}
\newcommand\C{\ensuremath{\mathbb{C}}}

% Easily typeset systems of equations (French package)
\usepackage{systeme}

% Put x \to \infty below \lim
\let\svlim\lim\def\lim{\svlim\limits}

%Make implies and impliedby shorter
\let\implies\Rightarrow
\let\impliedby\Leftarrow
\let\iff\Leftrightarrow
\let\epsilon\varepsilon

% Add \contra symbol to denote contradiction
\usepackage{stmaryrd} % for \lightning
\newcommand\contra{\scalebox{1.5}{$\lightning$}}

% \let\phi\varphi

% Command for short corrections
% Usage: 1+1=\correct{3}{2}

\definecolor{correct}{HTML}{009900}
\newcommand\correct[2]{\ensuremath{\:}{\color{red}{#1}}\ensuremath{\to }{\color{correct}{#2}}\ensuremath{\:}}
\newcommand\green[1]{{\color{correct}{#1}}}

% horizontal rule
\newcommand\hr{
    \noindent\rule[0.5ex]{\linewidth}{0.5pt}
}

% hide parts
\newcommand\hide[1]{}

% si unitx
\usepackage{siunitx}
\sisetup{locale = FR}

% Environments
\makeatother
% For box around Definition, Theorem, \ldots
\usepackage{mdframed}
\mdfsetup{skipabove=1em,skipbelow=0em}
\theoremstyle{definition}
\newmdtheoremenv[nobreak=true]{definitie}{Definitie}
\newmdtheoremenv[nobreak=true]{eigenschap}{Eigenschap}
\newmdtheoremenv[nobreak=true]{gevolg}{Gevolg}
\newmdtheoremenv[nobreak=true]{lemma}{Lemma}
\newmdtheoremenv[nobreak=true]{propositie}{Propositie}
\newmdtheoremenv[nobreak=true]{stelling}{Stelling}
\newmdtheoremenv[nobreak=true]{wet}{Wet}
\newmdtheoremenv[nobreak=true]{postulaat}{Postulaat}
\newmdtheoremenv{conclusie}{Conclusie}
\newmdtheoremenv{toemaatje}{Toemaatje}
\newmdtheoremenv{vermoeden}{Vermoeden}
\newtheorem*{herhaling}{Herhaling}
\newtheorem*{intermezzo}{Intermezzo}
\newtheorem*{notatie}{Notatie}
\newtheorem*{observatie}{Observatie}
\newtheorem*{oef}{Oefening}
\newtheorem*{opmerking}{Opmerking}
\newtheorem*{praktisch}{Praktisch}
\newtheorem*{probleem}{Probleem}
\newtheorem*{terminologie}{Terminologie}
\newtheorem*{toepassing}{Toepassing}
\newtheorem*{uovt}{UOVT}
\newtheorem*{vb}{Voorbeeld}
\newtheorem*{vraag}{Vraag}

\newmdtheoremenv[nobreak=true]{definition}{Definition}
\newtheorem*{eg}{Example}
\newtheorem*{notation}{Notation}
\newtheorem*{previouslyseen}{As previously seen}
\newtheorem*{remark}{Remark}
\newtheorem*{note}{Note}
\newtheorem*{problem}{Problem}
\newtheorem*{observe}{Observe}
\newtheorem*{property}{Property}
\newtheorem*{intuition}{Intuition}
\newmdtheoremenv[nobreak=true]{prop}{Proposition}
\newmdtheoremenv[nobreak=true]{theorem}{Theorem}
\newmdtheoremenv[nobreak=true]{corollary}{Corollary}

% End example and intermezzo environments with a small diamond (just like proof
% environments end with a small square)
\usepackage{etoolbox}
\AtEndEnvironment{vb}{\null\hfill$\diamond$}%
\AtEndEnvironment{intermezzo}{\null\hfill$\diamond$}%
% \AtEndEnvironment{opmerking}{\null\hfill$\diamond$}%

% Fix some spacing
% http://tex.stackexchange.com/questions/22119/how-can-i-change-the-spacing-before-theorems-with-amsthm
\makeatletter
\def\thm@space@setup{%
  \thm@preskip=\parskip \thm@postskip=0pt
}


% Exercise 
% Usage:
% \oefening{5}
% \suboefening{1}
% \suboefening{2}
% \suboefening{3}
% gives
% Oefening 5
%   Oefening 5.1
%   Oefening 5.2
%   Oefening 5.3
\newcommand{\oefening}[1]{%
    \def\@oefening{#1}%
    \subsection*{Oefening #1}
}

\newcommand{\suboefening}[1]{%
    \subsubsection*{Oefening \@oefening.#1}
}


% \lecture starts a new lecture (les in dutch)
%
% Usage:
% \lecture{1}{di 12 feb 2019 16:00}{Inleiding}
%
% This adds a section heading with the number / title of the lecture and a
% margin paragraph with the date.

% I use \dateparts here to hide the year (2019). This way, I can easily parse
% the date of each lecture unambiguously while still having a human-friendly
% short format printed to the pdf.

\usepackage{xifthen}
\def\testdateparts#1{\dateparts#1\relax}
\def\dateparts#1 #2 #3 #4 #5\relax{
    \marginpar{\small\textsf{\mbox{#1 #2 #3 #5}}}
}

\def\@lecture{}%
\newcommand{\lecture}[3]{
    \ifthenelse{\isempty{#3}}{%
        \def\@lecture{Lecture #1}%
    }{%
        \def\@lecture{Lecture #1: #3}%
    }%
    \subsection*{\@lecture}
    \marginpar{\small\textsf{\mbox{#2}}}
}



% These are the fancy headers
\usepackage{fancyhdr}
\pagestyle{fancy}

% LE: left even
% RO: right odd
% CE, CO: center even, center odd
% My name for when I print my lecture notes to use for an open book exam.
% \fancyhead[LE,RO]{Gilles Castel}

\fancyhead[RO,LE]{\@lecture} % Right odd,  Left even
\fancyhead[RE,LO]{}          % Right even, Left odd

\fancyfoot[RO,LE]{\thepage}  % Right odd,  Left even
\fancyfoot[RE,LO]{}          % Right even, Left odd
\fancyfoot[C]{\leftmark}     % Center

\makeatother




% Todonotes and inline notes in fancy boxes
\usepackage{todonotes}
\usepackage{tcolorbox}

% Make boxes breakable
\tcbuselibrary{breakable}

% Verbetering is correction in Dutch
% Usage: 
% \begin{verbetering}
%     Lorem ipsum dolor sit amet, consetetur sadipscing elitr, sed diam nonumy eirmod
%     tempor invidunt ut labore et dolore magna aliquyam erat, sed diam voluptua. At
%     vero eos et accusam et justo duo dolores et ea rebum. Stet clita kasd gubergren,
%     no sea takimata sanctus est Lorem ipsum dolor sit amet.
% \end{verbetering}
\newenvironment{verbetering}{\begin{tcolorbox}[
    arc=0mm,
    colback=white,
    colframe=green!60!black,
    title=Opmerking,
    fonttitle=\sffamily,
    breakable
]}{\end{tcolorbox}}

% Noot is note in Dutch. Same as 'verbetering' but color of box is different
\newenvironment{noot}[1]{\begin{tcolorbox}[
    arc=0mm,
    colback=white,
    colframe=white!60!black,
    title=#1,
    fonttitle=\sffamily,
    breakable
]}{\end{tcolorbox}}




% Figure support as explained in my blog post.
\usepackage{import}
\usepackage{xifthen}
\usepackage{pdfpages}
\usepackage{transparent}
\newcommand{\incfig}[1]{%
    \def\svgwidth{\columnwidth}
    \import{./figures/}{#1.pdf_tex}
}

% Fix some stuff
% %http://tex.stackexchange.com/questions/76273/multiple-pdfs-with-page-group-included-in-a-single-page-warning
\pdfsuppresswarningpagegroup=1

\title{\Huge{Analysis I}}
\author{\huge{Daniel Yu}}
\date{September 9, 2024}

\begin{document}
\maketitle
\newpage% or \cleardoublepage
% \pdfbookmark[<level>]{<title>}{<dest>}
\tableofcontents
\pagebreak

\section{Metric Spaces}
\begin{definition}
  Let X be a non-empty set and a function:
  \[
  S: X \times X \to \R
  .\]
  \[
    \left( x,y \right) \longmapsto S\left( x,y \right) 
  .\] such that:
  \begin{itemize}
    \item $\forall x,y \in X$,  $S(x,y) \geq 0$ and  $S(x,y) = 0 \iff x =y$
    \item $S(y,x) = S(x,y)$
    \item \textbf{Triangle Inequality} $\forall x,y,z \in X$, $S(x,y) \leq S(x,z) + S(z,y)$ \\
      (imagine x,y,z as corners of a triangle)
  \end{itemize}
\end{definition}

\begin{remark}{Example}
  \begin{enumerate}
    \item Distance on a number line: \[
      S(x,y) = |x - y|, x,y \in \R 
    .\] 
    \item Let $\left( X,  \mid  \mid \cdot  \mid  \mid  \right) $ be a normed space then \[
    S(x,y) =  \mid  \mid x - y \mid  \mid 
  .\]Then $S$ is a distance function on $X$. \textbf{This means all norm spaces are metric spaces}.
\item A metric space does not have to be a \textbf{Vector Space}, for example a metric space could simply be the 
  collection of points defining a sphere and the distance would be the distance across the surface of the sphere 
  of two points!
  \end{enumerate}
\end{remark}

\subsection{Balls in Metric Spaces}

\begin{definition}
 Denote metric space as $\left( X, S \right)$. Then an (open) ball of radius r centered at x:
\[
  B_r (x) = \left\{ y \in X | S(y,x) < r \right\}  
.\]  
\end{definition}

\begin{remark}
  Let $X = \R$, $S(x,y) =  |x - y|$:
  \[
    B_{.5}(1) = \left\{ .5 < y < 1.5 | y \in R\right\} = \left( .5,1.5 \right)  
  .\] 
  
\end{remark}

\begin{definition}
  The punctured ball is a ball without the center and is defined as:
  \[
  B_r^o(x) = B_r(x) \setminus \left\{ x \right\} 
  .\] 
\end{definition}

\begin{note}
  Balls are a special type of \textbf{neighborhoods} (defined later)
\end{note}

\subsection{Open and Closed Sets in Metric Space}
\begin{definition}
  A subset $U \subseteq X$ is \textbf{open} if $\forall x \in U$ $\exists r > 0$ s.t.:
  \[
  B_r(x) \subseteq U
  .\]
  
\end{definition}
\begin{remark}
  The analogy is that a closed interval $[0,1) \subseteq \R$  would  \textbf{not be an open set}. For example, then there is no ball centered at
  $B_r(0)$ because there exist $y < 0 \in \R$ but there are no points to the left of 0 in $[0,1)$ and $r \not\geq 0$.  
\end{remark}

\begin{remark}
  However, if X is restricted to $[0,1)$ and we consider $U = [0,1) \subseteq  X$, this is an \textbf{open set} because 
  there does not existed $y < 0 \in X = [0,1)$ so a ball centered at 0 would be $B_{.5}(0) = [0,.5)$. 
\end{remark}

\begin{definition}
  Let $E \subseteq X$. $x \in X$ is a \textbf{limit point} of $E$ if $\forall r > 0$:
  \[
    (B_r^o (x) \cap E) \neq \emptyset
  .\] 
\end{definition}

\begin{note}
  We are working with points (and distances) that may be infinitesimially small. For example, if a point is a limit point,
  then any ball (neighborhood) centered at the limit point contains infinitely many limit points, because we could take $r' < r$.
\end{note}

\begin{remark}{Example} \\
  Consider $E = [0,1) \subseteq X = \R$:
  All the interior points $0 < x < 1$ are limit points. \\
  $0$ is a limit point \\
  $1$ is a limit point \\
  The set of all limit points in E would be $[0,1]$. Note that this is different from open subset definition as 0 is included
  because the intersection $B_r^o (x) \cap E = (0,r)$, we don't have to restrict the ball (or neighborhood) to be inside $E$!
\end{remark}

\begin{definition}
  $E \subseteq X$ is \textbf{closed} if the set of limit points $E' \subseteq E$.
\end{definition}

\begin{note}
  Give a metric space $\left( X, S \right)$, $X \subseteq X$ and $X$ is open and closed by definition. Similarly, $\emptyset$ is 
  both open and closed (vaccously).
\end{note}

\begin{remark}{Example}\\
 
  \begin{enumerate}
    \item $X = \R, E = [0,1)$, this set is not closed in in $(\R,S)$ as $1$ is a limit point but is not in $E$. 
    \item However, for $X=[0,1)$, $E=[.5,1)$, $E$ is closed in $\left( X,S \right)$ since $1$ is not a limit point (in fact it doesn't exist
      in $X$)
  \end{enumerate}
\end{remark}

\begin{prop}
Complement Rules -- Metric Space is $(X, S)$ 
  \begin{enumerate}
    \item $U$ is open $\iff$ $X \setminus U$ is closed
      \begin{proof}
        $\to$ Assume $U$ open set, $U \subseteq X$.  Take $x \in U$, by def. since $U$ is open, $\exists r > 0$ such 
        that $B_r(x) \subseteq U$ and so $B_r^0 (x) \subseteq U$. Then, 
        $B_r^0 \cap (X \setminus U) = \emptyset$, so x cannot be a limit point in $X \setminus U$.  \\ \\
        $\leftarrow$ Assume $X \setminus U$ is a closed set, then use forward direction of (2) that the complement of the closed set is open.
      \end{proof}
    \item $E$ is closed $\iff$ $X\setminus E$ is open
      \begin{proof}
        $\to$ Assume that $E$ is closed. Take a point in $x \in X \setminus E$. By definition, $x$ is not a limit point 
        of $E$ because all the limit points of $E$ are in $E$ since $E$ is closed. Thus, $\exists r > 0$ such that
        $B_{r}^0 (x) \cap E = \emptyset$ and $B_{r} (x) \cap E = \emptyset$ (the center $x$ also not in $E$). Then, 
        the ball $B_r(x) \subseteq X \setminus E$. As this is true for any x, then by def. $U$ is open. \\ \\
        $\leftarrow$ Assume $X \setminus E$ is open, then use forward direction of (1) that the complement of the open set is closed!  
      \end{proof}
  \end{enumerate}
\end{prop}

\begin{prop}
  Union Rules -- Metric Space $(X, S)$
  \begin{enumerate}
    \item If $U_1 \subseteq X$, $U_2 \subseteq X$ are open, then $U_1 \cup U_2$ and $U_1 \cap U_2$ are open.
      \begin{proof}
        Assume $U_1, U_2 \subseteq X$ and are open. Consider $U_1 \cup U_2$, take $x \in U_1 \cup U_2$. Then $x \in
        U_1$. Since they are open, by definition, $\exists r > 0$ such that $B_r (x) \subseteq U_1$ for example and 
        $B_r (x) \subseteq U_1 \cup U_2$. The same argument follows for $x \in U_2$. Since this is for any $x$, then $U_1 \cup U_2$ is open. \\ \\
        Consider $U_1 \cap U_2$, take $x \in U_1 \cap U_2$ so $x \in U_1$ and $x \in U_2$, then by definition $\exists r_1,r_2 > 0$ such that $B_{r_1}(x) \subseteq U_1$
        and $B_{r_2}(x) \subseteq U_2$. We can simply take the minimum $r = min(r_1,r_2)$ (since $r_1,r_2$ are non-negative),so 
        $B_r(x) \subseteq U_1 \cap U_2$ and this is true for all $x$, so $U_1 \cap U_2$ is open. 
      \end{proof}
    \item If $U_{\alpha}, \alpha \in I$ set of indicies, then $\cup_{\alpha \in I} U_{\alpha}$ is open.
      \begin{proof}
        Assume $U_\alpha$ is open for all $\alpha in I$. Take $x \in \cup_{alpha \in I} U_\alpha$ then there exists $\beta \in I$ such that 
        $x \in U_{\beta}$. Since $U_\beta$ is open $\exists r > 0$ such that $B_r(x) \subseteq U_\beta \subseteq 
        \cup_{\alpha \in I} U_\alpha$ and this is true for any $x$, then $\cup_{\alpha \in I}$ is an open set in $X$.
      \end{proof}
  \end{enumerate}
\end{prop}

\begin{remark}
  Is infinite intersection $\cap_{\alpha}^\infty U_{\alpha}$ also open?
  \begin{proof}{Counterexample}\\
    Take $X = \R$, $S(x,y) = \| x -y\|$. Then define $U_n =\left( - \frac{1}{n} -1 , 1 + \frac{1}{n} \right)$ which
    is clearly open. As we take $n = 1,2,3,\ldots$, $\cap_{n \geq 1} U_n = [-1,1]$ which is closed, a counterexample.
  \end{proof}
  This doesn't contradict the above (one could argue that you could repeatedly take $U_1 \cap U_2$ which we proved is open)
  because the above is for finitely many sets!
\end{remark}

\begin{note}
  $\text{normed spaces} \subseteq \text{metric space} \subseteq \text{topological space}$
\end{note}

\begin{definition}
  \textbf{Topological Space}. \\
  Let $X$ be a set. A \textbf{topology on X} is a set of subsets that are called 
  "open" (not all possible sets -- Ex: $\tau = \{u_1,v_1,\ldots\}$) and satisfy the following properties.  
  \begin{enumerate}
    \item $X, \emptyset$ are open
    \item $U_1, U_2$ are open $\to$ $U_1 \cup U_2$ and $U_1 \cap U_2$ are open.
    \item If $U_\alpha, \alpha \in I$ are open $\to U_{\alpha \in I} U_\alpha$ is open.
  \end{enumerate}
\end{definition}

\begin{remark}
  A metric space is a a topological space with open sets following the definition given above. 
\end{remark}

\begin{note}{Example}\\
  $X = \{a, b\}$. Then let $\tau = \{\emptyset, \{a\}, X \}$. Property (1) is clearly satisfied. Property (2) is
  satisfied $\emptyset \cup \{a\} = \{a\} \in \tau$ so it is open. The same follows for all other unions. Then 
  checking intersections: $X \cap \{a\} = \{a\} \in \tau$ so it is open. Property (3) follows quickly from (2).  
\end{note}

\end{document}
