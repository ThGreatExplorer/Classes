\documentclass[a4paper]{article}
\usepackage[a4paper, margin=1in]{geometry}
\input{preamble}
\title{\Huge{Analysis I}}
\author{\huge{Daniel Yu}}
\date{September 23, 2024}

\pdfsuppresswarningpagegroup=1

\begin{document}
\maketitle
\newpage% or \cleardoublepage
% \pdfbookmark[<level>]{<title>}{<dest>}
\tableofcontents
\pagebreak
  
\section{Closure of a Set}
\begin{definition}
  Let $(X, \rho)$ be a metric space where  $E \subseteq X$. Denote $\overline{E}$ as the \textbf{closure}.
  \[
    \overline{E} = \cap_{E \subseteq F} F \text{ and F is closed}
  .\]
  And since, the intersection of any number of (including infinite) closed sets is closed. 
\end{definition}
\begin{note}
  It's obvious that if $E$ is closed, then  $E = \overline{E}$.
\end{note}
\end{document}
