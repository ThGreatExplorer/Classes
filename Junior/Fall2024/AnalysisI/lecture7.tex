\documentclass[a4paper]{article}
\usepackage[a4paper, margin=1in]{geometry}
\input{preamble}
\title{\Huge{Analysis I}}
\author{\huge{Daniel Yu}}
\date{October 21, 2024}

\pdfsuppresswarningpagegroup=1

\begin{document}
\maketitle
\newpage% or \cleardoublepage
% \pdfbookmark[<level>]{<title>}{<dest>}
\tableofcontents
\pagebreak
  
\section{Connected Sets}
Define connected sets as a set that is not a not connected sets
\begin{definition}
  $\left( X, \rho \right) $ is not connected if  $\exists U_{\text{open}} \neq \emptyset, V_{\text{open}} \neq \emptyset$ such that
  \[
  X = U \cup V, U \cap V = \emptyset
  .\] 
  This is also written as the disjoint union:
  \[
  X = U \amalg V
  .\] 
  \textbf{Not-connectedness} is an intrinsic propery of the metric space.
\end{definition}

\begin{definition}
  let $E \subseteq X$.  $E$ is not connected in $X$ if  $\left( E, \rho \right) $ is not connected.
\end{definition}

\begin{prop}
  $\left( \to \right) E \subseteq X$ is not connected w.r.t to $X$  $\iff$ $\exists A \neq \emptyset, B \neq \emptyset$ such that $E = A \cup B$ and $\overline{A} \cap B = \emptyset, A \cap \overline{B}$ (closures are in $X$)

  \begin{proof}
    Assume that $E \subseteq X$ is not connected in $X$ $\Rightarrow$ $\left( E, \rho \right) $ is not connected $\Rightarrow$ $\exists U_{\text{open}} \neq \emptyset \subseteq E, V_{\text{open}} \neq \emptyset \subseteq E$ and $E = U \cup V$ where  $U \cap V = \emptyset$. \\

    By homework 1, $\exists \tilde{U} \subseteq X$, $\tilde{U} \cap E = U$. Then,
     \[
       \tilde{U} \cap V = \left( \tilde{U} \cap E \right) \cap V = U \cap V = \emptyset 
    .\]
    Then,
    \[
      V \subseteq \left( X \setminus \tilde{U} \right) \Rightarrow \overline{V}^{X} \subseteq \left( X \setminus \tilde{U} \right) \Rightarrow \overline{V}  \cap \tilde{U} \Rightarrow \overline{V} \cap U = \emptyset
    .\] 
    We can apply the same argument to $V$ and obtain that  $\overline{U}^{X} \cap V = \emptyset$. In addition, $U,V$ are nonempty and  $E = U \cap V$. \\


    $\left( \leftarrow \right) $ Assume that $\exists A \neq \emptyset, B \neq \emptyset$ such that $E = A \cup B$ and $\overline{A}^{X} \cap B = \emptyset, A \cap \overline{B}^{X}$. Since 
    \[
    \overline{A} \cap B = \emptyset \Rightarrow B \subseteq X \setminus \overline{A}
    .\] 
    Note that $X \setminus \overline{A}$ must then be open since the complement of a closed set is open.
    \[
      \left( X \setminus \overline{A} \right) \cap E = \left( X \cap E \right) \setminus \left( \overline{A} \cap E \right) = \left( A \cup B \right)  \setminus \overline{A} = B
    .\]
    Since $B = \left( X \setminus \overline{A} \right)  \cap E$, the intersection of an open set with an open set (a metric space is open and closed), then $B$ is an open set in  $\left( E, \rho \right) $. We can apply the same arugment to $A$ to show that  $A$ is open in  $\left( E, \rho \right) $. \\

    We also know that $A \neq \emptyset, B \neq \emptyset, E = A \cup B$ where $A,B$ are open, and  $A \cap B = \emptyset$ because $\overline{A} \cap B = \emptyset, A \cap \overline{B} = \emptyset$. Thus, $E$ is not connected in  $X$ as  $(E, \rho)$ is not connected
  \end{proof}
\end{prop}

\begin{note}{Example}\\
  Consider the set in $\R^{2}$. Define $E = \{\left( 0,y \right) \mid -1 \leq y \leq 1 \} \coprod \{\sin \frac{1}{x} \mid  0 < x \leq \frac{2}{\pi}\}$ where $\coprod$ is the disjoint union. This is an example of a \textbf{connected set}

\begin{figure}[H]
  \centering
  \includegraphics[width=0.8\textwidth]{assets/connected_set_example.png}
  \caption{$E$ is connected!}
  \label{fig:connected_set_example}
\end{figure}
Note that the red line never reaches the blue, so they don't intersect. \\
\end{note} 


\begin{theorem}
  Let $f:X \to Y$ a continous map and  $E \subseteq X$ (connected in  $X$) $\Rightarrow f(E)$ is connected in  $Y$.

  \begin{proof}
    Since $f:X \to Y$ is continuous, then  $f:E \to f(E)$ is also continuous. We are given that $\left( E,\rho \right) $ is connected. Consider the strongest case where $E =X$, so  $\left( X, \rho_X \right) $ is connected and $Y = f(X)$. Since by subspace topology, we can always restrict to $E \subseteq X, f:E \to f(E)$, then the above case is enough to prove the statement for all cases. \\


    Thus, we will prove that if $f:X \to Y$ is continuous and  $X$ is not connected and  $f(X) = Y$ then  $Y$ is connected. Prove this by contradiction, assume that the above holds but $Y$ is not connected. Then,
    \[
    Y = U \amalg V, U_{\text{open}} \neq \emptyset, V_{\text{open}} \neq \emptyset
    .\] 
    Consider the pre-images which we know are open due to continuous map theorems.
    \[
    f^{-1}\left( U \right) \subseteq X, f^{-1}\left( V \right) \subseteq X
    .\]
    We also know that both pre-images are not empty because the image is not empty and the map is onto (we construct $f(X) = Y$).
    \[
      f^{-1}\left( U \right) \neq \emptyset, f^{-1} \left( V \right)  \neq \emptyset
    .\]
    We can use the following result from set theory:
    \[
    f^{-1} \left( U \cap V \right) = f^{-1} (U) \cap f^{-1} (V)
    .\]
    And we know that $U \cap V = \emptyset$ because $U \amalg V$ (disjoint union), so
    \begin{align*}
      f^{-1} \left( U \cap V \right) &=    f^{-1} \left( \emptyset \right) \\
                                     &= \emptyset  \\
                                     &=  f^{-1} (U) \cap f^{-1} (V)
    .\end{align*}
    This means that open set $f^{-1}(U), f^{-1}(V)$ in $X$ and which $f^{-1}(U) \cup f^{-1}(V) = X$ and $f^{-1}(U) \cap f^{-1}(V) = \emptyset$, so $f^{-1}(U) \amalg f^{-1}(V) = X$ and $X$ is not connected. This is a contradiction. Thus,  if $f:X \to Y$ is continuous and  $X$ is not connected and  $f(X) = Y$ then  $Y$ is connected which is enough to prove the more general statement $f: X \to Y$ continuous map and  $E \subseteq X$  $\Rightarrow f(E)$ connected in  $Y$.
       
  \end{proof}
\end{theorem}

\subsection{connected sets in $\R$}
\begin{theorem}
  $E \subseteq \R$ is not connected $\iff$ $\exists x,y \in E,$ s.t. $ x < y$ and $x < z < y$, where  $z \not\in E$ 
  \begin{proof}
    $\left( \Rightarrow \right) $ Assume that $E$ is not connected. Then, $\exists U \neq \emptyset, V \neq \emptyset$, $U,V$ open in $E$,  $E = U \amalg V $ ($U = X \setminus V$ but $V$ is open, so $U$ is open and closed and by the same arugment  $V$ is also open and closed). Take $x \in U, y \in V$. Without loss of generality, assume that  $x < y$. Consider
     \[
       \alpha = sup\left( [x,y] \cap U \right) 
     .\]
     which must exist as $[x,y]$ is bounded and  $U$ is bounded. Consider the following cases:
      \begin{enumerate}
       \item If $\alpha \not\in E$ then  $x < \alpha < y$ (since  $x,y \in E$ and  $\alpha \not\in E$). The theorem thus holds for all $z = \alpha$
       \item   If $\alpha \in E$ then either:
         \begin{enumerate}
           \item $\alpha \in U$ then since $U$ is open in  $E$,  $\exists  \epsilon > 0$ such that \[
           B_\epsilon (\alpha)^{E} = (\left( \alpha - \epsilon, \alpha + \epsilon \right) \cap E) \subseteq U
           .\] and $\alpha + \epsilon < y$. Take $z \in \left( \alpha, \alpha + \epsilon \right) $,
           \begin{enumerate}
             \item If $z \not\in E$ then $x \leq \alpha < z < \alpha + \epsilon < y$. Hence, the theorem follows in this case
             \item   If $z \in E$, then  $z > \alpha$,  $e \in [x,y] \cap U \Rightarrow \alpha$ is not the $sup([x,y] \cap U)$  so this arugment is not possible. 
           \end{enumerate}
         \item If $\alpha \in V$ then this means that $\alpha \not\in ([x,y] \cap U)$. Then $\alpha$ is a limit point of  $[x,y] \cap U \Rightarrow$ $\alpha \in E$ is a limit point of $U$. Note that  $U$ is both open and closed in  $E$ which implies that $\alpha \in U$, but we know  $V \cap U = \emptyset$, so this is a contradiction!
         \end{enumerate}
     \end{enumerate}
  \end{proof}
\end{theorem}

\begin{definition}
  A subset $E$ is convex if  $\forall x,y \in E$ we have that  $[x,y] = \{tx + (1-t)y \mid 0 \leq t \leq 1\}  \subseteq E$ 
\end{definition}

\begin{theorem}
  $E \subseteq \R$ is connected $\iff E$ is convex. 
  
  \begin{proof}

  \end{proof}
\end{theorem}

\begin{corollary}
  A subset $E \subseteq \R$ is connected $\iff$ $E$ is one of the following sets:  \\
  $\emptyset, \R, (a,b), [a,b], [a,b), (a,b], (-\infty,a], (a, \infty), [a, \infty)$
\end{corollary}

\subsection{Intermediate Value Theorem}
\begin{corollary}
  If $f:[a,b] \to \R$ be continuous and $f(a) < f(b).$ Then if  $f(a) < c < f(b)$ then  $\exists x_{*} \in [a,b]$ such that $f(x_{*})  = c$. 

  \begin{proof}
    The interval $[a,b]$ is connected and compact  (in $[a,b]$, we restrict to $[a,b]$ because the function only defined as  $[a,b]$) then since $f:[a,b] \to \R$ is continuous, $f([a,b])$ must also be connected and compact in  $\R$ $\Rightarrow f([a,b]) = [A,B]$. Since the interval exists then $f(x_{*}) \in [A,B]$ exists and must be the image of $x_{*} \in [a,b]$. 

    \begin{note}
      We don't necessarily need compactness just to prove the above theorem, we use compactness for the stronger conclusion that the image of a closed interval under a continuous map is also a closed interval (where closed interval in $\R$ is connected and compact). 
    \end{note}
  \end{proof}
\end{corollary}

\section{Path Connected}
\begin{definition}
  $X$ is path connected if  $\forall x,y \in X$  $\exists $ a continuous map $\gamma [0,1] \to X$ such that  $\gamma(0) = x, \gamma(1) =y$.
\end{definition}

\begin{prop}
  If $(X,\rho)$ is path connected  $\Rightarrow$  $X$ is connected i.e.  $\{\text{path connected sets}\} \subseteq \{\text{connected sets}\}  $
\end{prop}

\end{document}
