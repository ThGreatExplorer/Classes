\documentclass[a4paper]{article}
\usepackage[a4paper, margin=1in]{geometry}
\input{preamble}
\title{\Huge{Analysis I}\\ Compactness}
\author{\huge{Daniel Yu}}
\date{September 16, 2024}

\pdfsuppresswarningpagegroup=1

\begin{document}
\maketitle
\newpage% or \cleardoublepage
% \pdfbookmark[<level>]{<title>}{<dest>}
\tableofcontents
\pagebreak
\section{Compact Sets}

\begin{definition}
  A set of \textbf{open} sets $\{U_{\alpha}\}_{\alpha \in I}$ where $I$ is a set of indicies is called an 
  \textbf{open cover of E} if 
  \[
    E \subseteq \cup_{\alpha \in I} U_\alpha 
  .\]  
\end{definition}

\begin{definition}
  If $\{U_{\alpha}\}_{\alpha \in I}$ is an open cover in $E$ and  $I_1 \subseteq I$ such that 
   \[
     E \subseteq \cup_{\alpha \in I_1} U_\alpha 
  .\] 
  then $\{U_{\alpha}\}_{\alpha \in I_1}$ is called a \textbf{subcover} of $\{U_{\alpha}\}_{\alpha \in I}$ 
\end{definition}

\begin{definition}
  The set $E \subseteq X$ is \textbf{compact} in $X$ if any \textbf{open cover} of $E$ has a \textbf{finite subcover} 
\end{definition}

\begin{figure}[h]
  \centering
  \includegraphics[width=0.8\textwidth]{assets/compact_set_ex.png}
  \caption{Compact Set}
  \label{fig:compact_set_ex}
\end{figure}

\begin{definition}
  A subset $E \subseteq X$ is \textbf{bounded} if  $\exists x_0 \in X, r >0$, such that  $E \subseteq B_r(x_0)$
\end{definition}

\begin{prop}
  If $E \subseteq X$ is \textbf{compact} then E is \textbf{bounded}
  \begin{proof}{By Contradiction}\\
    Assume that $E$ is compact but not bounded.  Take $x_0 \in X$ and consider the open cover of $E$,  $U_n = \{ B_n(x_0) \}$ 
    for $n=1,\ldots,n$. We know that each of these balls $B_n(x_0)$ is open by definition. Then $E \subseteq \cup_{n \geq 1} B_n(x_0)$.
    Since $E$ is compact, then there exists finite subcover, so $\cup_{n \geq 1} B_n(x_0)$ is composed of finitely many balls.
    Then, $\exists 1 \leq n_1 < n_< \ldots < n_k$ such that $E \subseteq \cup_{j=1}^k B_{n_j}(x_0) = B_{n_k} (x_0)$ 
    and $E$ is contained in a ball with largest radius and this means $E$ is bounded which is a contradiction!  \end{proof}
\end{prop}
\begin{note} 
    In this proof, we are constructing a finite subcover of concentric balls starting from the origin with increasing radius,
    one possible out of many.
\end{note}

\begin{definition}
  The \textbf{haussdorff property} is defined as follows. For any $x \neq y$ in  $\left( X, \rho \right) $ metric space.
  Then $\exists r_1,r_2>0$ such that $B_{r_1}(x) \cap B_{r_2}(y) = \emptyset$. This \textbf{Haussdorff space} is also known
  as the \textbf{T2 axiom}.
\end{definition}
\begin{note}
  This is always true for \textbf{metric spaces}

  \begin{proof}{By contradiction} \\
    Take $0 < r < \frac{\rho(x,y)}{2}$. Assume $\exists z \in X$ such that $z \in B_r(x) \cap B_r(y)$.  Then,
    by the triangle inequality:
    \[
    \rho(x,y) \leq \rho(x,z) + \rho(z,y) < r + r = 2r 
    .\] 
    \[
    \frac{\rho(x,y)}{2} < r
    .\] 
    However, this contradicts the assumption $0 < r < \frac{\rho(x,y)}{2}$. Therefore, such a point $z$ 
    cannot exist. Any metric space must be haussdorff.
  \end{proof}
\end{note}

\begin{prop}
  If $E \subseteq X$ is compact then it is a closed set in $X$. 

  \begin{proof}
    Assume that $E$ is compact. We will use the fact that we already proved if a set $U$ is open, then it's complement
    $U^c$ is closed. So we will show $X \setminus E$ is an open set. If $E = X$ then  $E$ is closed since it is 
    the whole metric space and metric spaces are closed sets. Assume $E \neq X$. Then, take  $x \in X \setminus E$.
    For any $y \in E$,  $\exists r_y > 0$ such that $B_{r_y}(x) \cap B_{r_y}(y) = \emptyset$. Then,
    $\{B_{r_y} (y)\}_{y \in E} $ is an open cover of $E$ (in fact, even the centers would cover $E$). Since
    $E$ is compact  $\exists y_1, \ldots, y_N \in E$ such that $E \subseteq \cup_{j=1}^N B_{r_{y_j}} (y_j)$, a finite
    subcover.  Take $r = min \{r_{y_i}, r_{y_2}, \ldots, r_{y_N}\} > 0$. This means $\forall 1 \leq j \leq N$:
     \[
    B_r(x) \cap B_{y_j}(y_j) \subseteq B_{r_j} (x) \cap B_{r_j}(y_j) = \emptyset
    .\]
    So,
    \[
      B_r(x) \cap (\cup_{j=1}^N B_{r_j}(y_j)) = \emptyset
    .\] Then,
  $B_r(x) \cap E = \emptyset$ and  $B_r(x) \subseteq X \setminus E$. Since we can make this argument for any
  $x \in X \setminus E$, then $X \setminus E$ is open. And $E$ must be closed set in $X$.
  \end{proof}
\end{prop}

\begin{prop}
  Let $E \subseteq X$ compact and  $A \subseteq E$ and $A$ closed in  $X$. Then, $A$ is compact in $X$. 
  \begin{proof}
  Let $\{ U_\alpha\}_{\alpha \in I}$ be an open cover of $A$. We will prove that it has a finite subcover. 
Since $A$ is closed, then  $X \setminus A$ is open and we get $\{\{U_{\alpha} \}_{\alpha \in I} , X \setminus A \}$
    which is an open cover of $E$. Since  $E$ is compact,  $\exists $ a finite subcover of $E$, 
    $\{U_{\alpha_1}, U_{\alpha_2}, \ldots, U_{\alpha_N}, X\setminus A\}$. Then, $\{U_{\alpha_1}, \ldots, U_{\alpha_N} \}
     \subseteq \{U_{\alpha_1}, U_{\alpha_2}, \ldots, U_{\alpha_N}, X\setminus A\}$ and is a finite subcover of
    $A$.  
  \end{proof}
\end{prop}

\begin{prop}
  Let $E$ be a compact set and let  $S \subseteq E$ compact that has infinitely many element. Then  $\exists $
  a limit point of $S$ inside $E$. (will be used later on when talking about convergence).
  \begin{proof}{By Contradiction}\\
    Assume $E$ is compact and  $S \subseteq E$ and  $S$ has infinitely many points and $S$ does not have a 
    limit point in  $E$. This means that since $S \subseteq E$, if there is no limit point in $E$,
    then $S$ does not have a limit point in general. Then for $\forall y \in E$ we have that  $y$ is not
    a limit point of  $S$ meaning, 
     \[
        \forall y \in E, \exists r_y > 0 \text{ such that } B_{r_y} (y) \text{ contains only finnitely many points from S}
    .\] 
    Then, 
    \[
    \{B_{r_y} \left( y \right) \}_{y \in E}
    .\] 
    is an open cover of $E$. Since $E$ is compact, $\exists $ a finite subcover of $E$:
     \[
       \{ B_{r_{y_1}} (y_1), \ldots, B_{r_{y_N}}(y_N) \} \text{ where } y_1,\ldots,y_N \in E 
     .\] 
     Then, $E \subseteq \cup_{i=1}^N B_{r_{y_j}}(y_j)$ and since $N$ is a finite number and for each  $B_{r_{y_j}}(y_j)$ 
     there only exist finitely many elements from $S$, then this means that $|E|$ is finite and since $S \subseteq E$,
      $S$ has only finitely many elements, but this is a contradiction!
  \end{proof}
\end{prop}

\section{Characterization of Compact Sets in $\R^n$}

\subsection{Suprenum and Infinitum in $\R$}
\begin{definition}
  M is an \textbf{upper bound} of $E \subseteq \R$ if $\forall x \in E$,  $x \leq M$
\end{definition}

\begin{definition}
  Let  $E \subseteq \R$ and let $E$ be bounded above (i.e.  $\exists $ an upper bound of $E$), then
  $alpha = sup(E)$ is known as the \textbf{suprenum} of $E$ if it satisfies:
    \begin{enumerate}
     \item $\alpha$ is an upper bound
     \item  $\forall \epsilon > 0$, the interval $(\alpha - \epsilon, \alpha] \cap E \neq \emptyset$
   \end{enumerate}
\end{definition}

\begin{remark}{Maximum vs Suprenum} \\
  Maximum can be thought of as the suprenum that belongs to the set. However, \textbf{the suprenum may not 
  necessarily belong to the set} (think limits). The same holds for the minimum and infinitum.
\end{remark}

\begin{definition}
  Then the infinitum $\beta = inf(E)$is defined as:
  \begin{enumerate}
    \item $\beta$ is a lower bound
    \item  $\forall \epsilon > 0$, the interval  $[\beta, \beta + \epsilon) \cap E \neq \emptyset$ 
  \end{enumerate}
\end{definition}

\begin{figure}[h]
  \centering
  \includegraphics[width=0.8\textwidth]{assets/sup_inf_ex.png}
  \caption{Suprenum and Infinitum vs Maximum and Minimum}
  \label{fig:sup_inf_ex}
\end{figure}

\begin{theorem}
  If $E \subseteq \R$ is bounded above then $\exists $ $sup(E)$. A similar statement holds for  $inf(E)$ 
  if  $E$ is bounded below.
\end{theorem}

\begin{prop}
The $sup(E)$,  $E$ is bounded above, is unique.
\begin{proof}{By Contradiction} \\
  Assume that there exists two suprenum $\alpha_1, \alpha_2$, $\alpha_1 \neq \alpha_2$. Then WLOG, let
  $\alpha_1 < \alpha_2$. This means that for $\alpha_1$ in  $E$, by definition,  $\exists \epsilon_1$ 
  such that the interval $(\alpha_1 - \epsilon_1, \alpha_1] \cap E \neq \theta$ and for $\alpha_2$ in $E$,
  $\exists \epsilon_2$ such that $( \alpha_2 - \epsilon_2, \alpha_2 ] \neq \theta$. However, because 
  $\alpha_1 < \alpha_2$ then $\alpha_1 \in ( \alpha_2 - \epsilon_2, \alpha_2 ]$ which is non-empty with intersection
  with $E$, so there are elements  $\alpha_1 < \alpha_1 + \epsilon_1 \in E$. This means that $\alpha_1$ 
  is not an upper bound and can't be a suprenum! 
\end{proof}
\end{prop}

\begin{prop}
  If  $M$ is an upper bound of  $E \subseteq \R$ then $sup(E) \leq M$.
   \begin{proof}
    
  \end{proof}
\end{prop}

\begin{lemma}
  Let $I_1 \supseteq I_2 \supseteq \ldots \supset I_k \supseteq \ldots$ be a sequence of closed intervals
  \[
    I_k = [a_k,b_k] \subseteq \R \text{ for all } k=1,2, \ldots
  .\] 
  Then, 
  \[
  \cap_{k=1}^\infty I_k \neq \emptyset
  .\] 
  \begin{proof}
    Assume $I_1 \supseteq I_2 \supseteq \ldots \supset I_k \supseteq \ldots$ are closed intervals. 
    Then for any given $k \geq 1$,  $a_l \leq b_k$,  $\forall l \geq 1$ (i.e. all the "lower bound"'s are smaller 
    than any "upper bound"). Then $b_k$ is an upper bound of  $\{a_1,a_2, \ldots\}$ so through \textit{Theorem 1}:
     $\alpha = sup_{l \geq 1} \{a_l\}$ exists and  
     \[
       \alpha \leq b_k \forall k \in 1,2,\ldots  
     .\]. Then, $\alpha$ is a lower bound for  $\{b_k | k \in \N\}$. And by \textit{theorem 1} again,
     \[
     \exists \beta = inf \{b_k | k \in \N\}  
     .\] 
     and, $\alpha \leq \beta$. Then, $\forall k \in \N$,
     \[
       a_k \leq \alpha \leq \beta \leq b_k \iff [\alpha, \beta] \subseteq I_k 
     .\] 
     So, 
     \[
       \cap_{k=1}^{\infty} I_k \supset [\alpha, \beta]
     .\]
     and the intersection is non-empty as $[\alpha, \beta]$ contains at least 1 element (when  $\alpha = \beta$)
  \end{proof}
\end{lemma}

\begin{note}
  If the intervals were not closed. For example: $I_n = \left( 0, \frac{1}{n} \right) $ for $n=1,2,3,\ldots$,
  then $\cap_{n \geq 0} I_n = \emptyset$
\end{note}

\begin{theorem}
  If $a \leq b$, then  $I = [a,b] \subseteq \R$ is compact.

  \begin{proof}{By contradiction}\\
    Assume that $I$ is not compact, then  $\exists $ an open cover $\{ U_{\alpha}\}_{\alpha \in A}$ of the interval
    $I$ that does not have a finite subcover. Take the midpoint of the interval $\frac{a+b}{2}$ and split $I_1 = I$ into
    $$I_1' = [a,\frac{a+b}{2}]$$ and $$I_1'' = [\frac{a+b}{2}, b]$$. One of these intervals cannot be covered by
    finitely many sets from the collection $\{U_{\alpha}\}_{\alpha \in A}$, otherwise if both could be covered
    by finitely many, then $I$ would have a finite subcover. Let us call this non-finitely covered interval $I_2 \subseteq I_1$.
    Continuing with this process for $I_n$, we construct $I_1 \supseteq I_2 \ldots \supseteq I_k \supseteq \ldots$,
    where:
    \begin{enumerate}
      \item $I_k \supset I_{k+1} \forall k \geq 1$
      \item $\forall k \geq 1, I_k$ can not be covered by finitely many sets from  $\{U_{\alpha}\}_{\alpha \in A}$ 
      \item $\mid I_k\mid  = \frac{b-a}{2^{k-1}}, k = 1,2,\ldots$
    \end{enumerate}
    By \textit{Lemma 1}, $\exists x \in \cap_{k=1}^\infty I_k$. Then $\exists \beta \in A$ such that $x \in U_{\beta}$.
    Since $U_\beta$ is open,  $\exists \epsilon > 0$ such that  $U_{\beta} \supseteq \left( x - \epsilon, x + \epsilon \right)$ for
    some $\epsilon > 0$. It now follows from property (3), that $\exists k \geq 1$ such that $I_k \subseteq \left( x - \epsilon,
    x + \epsilon \right) \subseteq U_\beta$. This is a contradiction with property (2) because $I_k$ can be covered by  $\{U_\beta\}$ 
    which is a finite subcover of size 1! Thus, $I = [a,b]$ is compact
  \end{proof}
\end{theorem}

\section{Compact Sets in $\R^n$}
Consider the rectangular box $I_n$:
\[
  [a_1,b_1] \times [a_2, b_2] \times \ldots \times [a_n,b_n] = \{\left( x_1,\ldots,x_n \right) \in \R^n | 
  a_k \leq x_k \leq b_k \forall k = 1,2,\ldots\} 
.\] 

\begin{theorem}
  $n \geq 1, I^n$ is bounded in $\R^n$
  The proof follows the argument in theorem (2) 
\end{theorem}

\begin{theorem}
$E \subseteq \R^n$ is compact $\iff E$ is closed and bounded.
\begin{proof}
   $\rightarrow$: Since $E$ is compact  $\to$  $E$ is closed and bounded by prop 1 and 2. \\

   $\leftarrow$: Assume that $E$ is closed and bounded in  $\R^n$. Then $\exists $ a rectangular box $B$ 
    inside $\R^n$ such that $E \subseteq B$ (we can consider the ball centered at $E$, then draw a box around it).
    But now realize that $E$ is closed and  $B$ is compact (by theorem 3) and by Proposition 3, this means that
    $E$ is compact. \\
\end{proof}
\end{theorem}

\begin{note}
  The above is not true in general, i.e. $E \subseteq X \not\iff$ E is closed and bounded. For example, consider
  $X = [0,1)$ and  $\rho(x,y) = \mid x-y\mid $, so $(X, \rho)$ is the metric space. Define  $E= [\frac{1}{2},1)$.
  Then $E$ is closed and bounded  in $ X$. But $E$ is not compact. Consider the open cover of E,
   \[
   U_k = \left( 0, 1 - \frac{1}{k+2} \right)  k = 1,2,3,\ldots 
   .\]
   Then, $\cup_{k \geq 1} U_k \supseteq E$. But this open cover does not have a finite subcover that covers $E$!
   \[
   \lim_{k \to \infty} U_k = \left( 0, 1 \right) 
   .\] 
   So there is no way to contain $\left( 1 - \epsilon,1 \right) \subseteq E$ as $\epsilon \to 0$ with finite subsets, it gets
   closer and closer but never reaches. 
\end{note}


\end{document}
