\documentclass[a4paper]{article}
\usepackage[a4paper, margin=1in]{geometry}
\input{preamble}
\title{\Huge{Analysis I}\\ Continuous Maps}
\author{\huge{Daniel Yu}}
\date{Octoboer 7, 2024}

\pdfsuppresswarningpagegroup=1

\begin{document}
\maketitle
\newpage% or \cleardoublepage
% \pdfbookmark[<level>]{<title>}{<dest>}
\tableofcontents
\pagebreak
  
\section{Continuous Maps}
\begin{definition}
 Let $\left( X, \rho_X \right) $ and $\left( Y, \rho_Y \right) $ be metric spaces. Define the following function
 \begin{align*}
   f: X &\longrightarrow Y \\
   x_0 &\longmapsto f(x_0) = y_0 
 .\end{align*}
 The map $f$ is \textbf{continuous at the point} $x_0 \in X$  $\iff$ $\forall \epsilon > 0$,  $\exists \delta > 0,$ 
 \[
 f\left( B_{\delta}^{X}(x_0) \right) \subseteq B_{\epsilon}^{Y} (f(x_0))
 .\]
 So the image of some ball around $x_0$ in $X$ is always a subset of a ball around $f(x_0)$ in $Y$
\end{definition}

\begin{figure}[h]
  \centering
  \includegraphics[width=0.8\textwidth]{assets/continuous_map_diagram.png}
  \caption{Continuous Map Diagram}
  \label{fig:continuous_map_diagram}
\end{figure}

\begin{note}
  when we say a map is continous, we mean either the map is continuous over the domain ($\forall x_0 \in X$) or that it is continuous for some specific $x_0$. For whichever case it is, it will be specified.
  \begin{enumerate}
    \item Non-continuous function: 
      \[
      D(x) = \begin{cases}
        1, x \in \Q \\
        0, \text{ else} 
      \end{cases}
      .\] 
      and $D: \R \to \R$ known as the dirichlet function
    \item Consider $f(x) = x \cdot D(x)$, this function is only only continuous at 0, $f \in C(0)$ or consider the Riemann function 
      \[
      R(x) = \begin{cases}
        \frac{1}{q}, x = \frac{p}{q} \\
        0, x \in \frac{\R}{\Q}
      \end{cases}
      .\] 
  \end{enumerate}
\end{note}

\begin{note}{Example}\\
  \begin{enumerate}
    \item  Take $f(x) = x$,  $f: [0,1] \to \R$. Without loss of generality, let $x_0 = \frac{1}{2}$ and $f(x_0) = y_0 = \frac{1}{2}$. Then, $\forall \epsilon > 0, B_{\epsilon}^{Y}(y_0) = (\frac{1}{2} - \epsilon, \frac{1}{2} + \epsilon)$ and we can take $\delta = \epsilon$ so that  $f(B_{\delta}^{X} (x_0)) = f( \left( \frac{1}{2} - \epsilon, \frac{1}{2} + \epsilon \right)) = (\frac{1}{2} - \epsilon, \frac{1}{2} + \epsilon) \subseteq B_{\epsilon}^{Y} (y_0) $.
    \item Consider 
      \begin{align*}
        f : \R &\longrightarrow \R^{2} \\
        x &\longmapsto f (x) = (x,x) 
      .\end{align*}
      this function takes the line in  $\R$ to a "diagonal" line in $\R^{2}$. 
    \item Consider 
      \begin{align*}
        g: \R^{2} &\longrightarrow \R \\
        (x,x) &\longmapsto g((x,x)) = x^{2} 
      .\end{align*}
  \end{enumerate}
\end{note}

\begin{remark}{Exercise}\\
  Consider the maps and show they are continuous
  \begin{enumerate}
    \item $\left( x,y \right) \longmapsto xy$, $f: \R^{2} \to \R$ 
    \item $\left( x,y \right) \longmapsto x+y$, $f: \R^{2} \to \R$
    \item $\left( x,y \right) \longmapsto \frac{x}{y}$, $f: \R^{2} \setminus \{y=0\} \to \R $
  \end{enumerate}
\end{remark}

\begin{theorem}
  The composition of continuous maps is continous. The map  $(g \cdot f)(x) = g(f(x))$,  $x \in X$ is called the composed map where if $f: X \to Y$ and  $g: Y \to Z$, then $g \cdot f: X \to Z$ (Kind of just  follows out from set theory)
\end{theorem}

 \begin{definition}
   If $f: X \to Y$  is a continuous map at the point $x_0$, then $f \in C(x_0)$ where $C(x_0)$ is the class of all maps from  $X \to Y$ that are continuous at $x_0$
\end{definition}

 \begin{theorem}
If $f \in C(x_0)$ and  $g \in C(f(x_0))$ then $g \cdot f \in C(x_0)$.   

\begin{proof}
  Take $\epsilon > 0$ and consider the ball  $B_{\epsilon}^{Z} (z_0)$. Since $g \in C(f(x_0)), \exists \tilde{\delta} > 0$ such that 
  \[
    g\left( B_{\tilde{\delta}}^{Y} (y_0) \right) \subseteq B_{\epsilon}^{Z} (z_0)  
  .\] Similarly, since $f \in C(x_0)$ then  $\exists \delta > 0$ such that
  \[
  f\left( B_{\delta}^{X} (x_0) ) \subseteq B_{\tilde{\delta}}^{Y} (y_0) 
  .\] 
  If then follows from (1) and (2) that 
  \[
    \left( g \cdot f \right) \left( B_{\delta}^{X} (x_0) \right) = g(f\left( B_{\delta}^{X} \left( x_0 \right) ) \right) \subseteq   g\left( B_{\tilde{\delta}}^{Y} (y_0) \right) \subseteq B_{\epsilon}^{Z} (z_0)  
  .\]
  Hence this implies $g \cdot f \in C(x_0)$
\end{proof}
 \end{theorem}

\begin{theorem}
  $f: X \to Y$ is continous on $X$ (i.e. continous at all points $x_0 \in X$) $\iff$ $\forall U_{\text{open}} \subseteq Y$, then $f^{-1}(U)$ is also open in $X$. Note  $f^{-1}$ not necessarily a function.

  \begin{proof}
    $\to$ Assume  $f: X \to Y$ is continous. Take $U_{\text{open}} \subseteq Y$ and consider $f^{-1}(U) = \{x \in X \mid f(x) \in U\} $, called the pre-image. Let us prove that $f^{-1}(U)$ is open in $X$. Take  $x_0 \in f^{-1}(U)$. Then, $f(x_0) \in U$. Since $U$ is open in $Y, \exists \epsilon > 0$ such that $B_{\epsilon}^{Y} (f(x_0)) \subseteq U$. It now follows that since $f \in C(x_0)$,  $\exists \delta > 0$ such that $f(B_{\delta}^{X} (x_0))\subseteq B_{\epsilon}^{Y} (f(x_0)) \subseteq U \Rightarrow f^{-1}(U) \supseteq B_{\delta}^{X}(x_0) \Rightarrow f^{-1}(U)$ is open since this is true for any $x_0 \in f^{-1}(U)$. \\


    $\leftarrow$ Assume that $\forall U_{\text{open}} \in Y$, we have $f^{-1}(U)$ is open in $X$. Take $x_0 \in X$ and consider the corresponding point  $y_0 = f(x_0)$. Take $\epsilon > 0$ and consider the ball $B_{\epsilon}^{Y}(y_0)$. Since a ball is an open set, we can apply our assumption,
     \[
       x_0 \in f^{-1}(B_{\epsilon}^{Y} (y_0))_{\text{open}} \subseteq X
     .\] 
Since $f^{-1}(B_{\epsilon}^{Y} (y_0))$ is open, $\exists \delta > 0$ such that 
\[
B_{\delta}^{X} (x_0) \subseteq f^{-1}(B_{\epsilon}^{Y} (y_0)) \Rightarrow f(B_{\delta}^{X} (x_0)) \subseteq B_{\epsilon}^{Y} (f(x_0))
.\]
Hence, $f \in C(x_0)$ and since this is true from any  $x_0 \in X$, $f \in C(X,Y)$.
  \end{proof}
 \end{theorem}

 \begin{theorem}
   Let $f:X \to Y$ be a continous map and $K_{\text{compact}} \subseteq X$. Then $f(K) \subseteq Y$ is compact in $Y$. 

    \begin{proof}
      Assume $f: X \to Y$ is continuous and  $K_{\text{compact}} \subseteq X$. Let $\{U_{\alpha}\}_{\alpha \in T} $ be an open cover of $f(K)$. Consider the set of open sets in  $X$ (by theorem 2):
      \[
      \{f^{-1}(U_{\alpha})\}_{\alpha \in I} 
      .\]
 Then this is an open cover of $K$,  $\cup_{\alpha \in I} f^{-1}(U_{\alpha}) \supseteq K$ (since for any $U_\alpha$ we can take  $x_0 \in f^{-1}(U_{\alpha}) \subseteq X$ and this will be true for all $x_0$). Take $x_0 \in K$,  $f(x_0) \in f(K)$ and since  $\{U_{\alpha}\}_{\alpha \in I} $ is a cover of $f(K) \exists \beta \in I$ such that 
 \[
 f(x_0) \in U_{\beta} \Rightarrow x_0 \in f^{-1}(U_{\beta})
 .\] 
 Since $K \subseteq X$ compact,  $\exists \alpha_1, \ldots, \alpha_N \in I$ such that 
 \[
 \cup_{j=1}^{N} f^{-1} (U_{\alpha_j}) \supseteq K
 .\]
 Then $\{U_{\alpha_j}\}_{j=1}^{N}$ is a finite subcover of $f(K)$,
  \[
 \cup_{j=1}^{N} U_{\alpha_j} \supseteq f(K)
 .\] Take $y_0 \in f(K)$ then  $y_0 = f(x_0), x_0 \in K$ then $\exists x_0 \in f^{-1}(U_{\alpha_{j_0}}) \Rightarrow f(x_0) \in U_{\alpha_{j_0}}$  
    \end{proof}
    \end{theorem}

 \begin{remark}
   (However, the preimage of a compact set is not necessarily compact). If $f: X \to Y$ is continuous and  $K$ is compact in $Y$ then  $f^{-1}(K)$ is not necessarily compact in $X$. Consider
    \[
      f(x) = \sin x, f: \R \to \R, K= [-1,1], f^{-1}(K) = \R \text{ not compact} 
   .\] 
 \end{remark}

 \begin{theorem}
   Let $f:X \to \R$ be a continuous map and $X$ is compact then  $\exists x_m, x_M \in X$ such that $f(x_m) = \inf_{X} f = \inf\left( \{f(x) \mid  x\in X\}  \right) $ and $f(x_M) = \sup_{X} f$. 

   \begin{proof}
     Since $X$ is compact and $f:X \to \R$ is continuous $\Rightarrow$  $im(f(X)) \subseteq \R$ is compact from the previous theorem (theorem 4), so $f(X)$ is bounded and closed. Let $\alpha = \sup_{X} f$ (this $\alpha \in \R$ exists since $\{f(x) | x \in X\} $ is bounded in $\R$). If $\alpha \in f(X)$ then  $\alpha = f(x_M)$ for some $x_M \in X$. If $\alpha \not\in f(X)$, then  $\alpha$ must be a limit point of  $f(X)$. Since  $f(X)$ is closed, $\Rightarrow d \in f(X)$, a contradiction. Hence, $\alpha \in f(X)$ and there must exist some  $f(x_m) = \inf_X f$. A similar argument can be made for the  $\sup_X f$.\\
   \end{proof}
 \end{theorem}
\end{document}
