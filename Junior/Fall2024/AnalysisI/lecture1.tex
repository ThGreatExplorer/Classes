\documentclass[a4paper]{article}
% Some basic packages
\usepackage[utf8]{inputenc}
\usepackage[T1]{fontenc}
\usepackage{textcomp}
\usepackage[dutch]{babel}
\usepackage{url}
\usepackage{graphicx}
\usepackage{float}
\usepackage{booktabs}
\usepackage{enumitem}

\pdfminorversion=7

% Don't indent paragraphs, leave some space between them
\usepackage{parskip}

% Hide page number when page is empty
\usepackage{emptypage}
\usepackage{subcaption}
\usepackage{multicol}
\usepackage{xcolor}

% Other font I sometimes use.
% \usepackage{cmbright}

% Math stuff
\usepackage{amsmath, amsfonts, mathtools, amsthm, amssymb}
% Fancy script capitals
\usepackage{mathrsfs}
\usepackage{cancel}
% Bold math
\usepackage{bm}
% Some shortcuts
\newcommand\N{\ensuremath{\mathbb{N}}}
\newcommand\R{\ensuremath{\mathbb{R}}}
\newcommand\Z{\ensuremath{\mathbb{Z}}}
\renewcommand\O{\ensuremath{\emptyset}}
\newcommand\Q{\ensuremath{\mathbb{Q}}}
\newcommand\C{\ensuremath{\mathbb{C}}}

% Easily typeset systems of equations (French package)
\usepackage{systeme}

% Put x \to \infty below \lim
\let\svlim\lim\def\lim{\svlim\limits}

%Make implies and impliedby shorter
\let\implies\Rightarrow
\let\impliedby\Leftarrow
\let\iff\Leftrightarrow
\let\epsilon\varepsilon

% Add \contra symbol to denote contradiction
\usepackage{stmaryrd} % for \lightning
\newcommand\contra{\scalebox{1.5}{$\lightning$}}

% \let\phi\varphi

% Command for short corrections
% Usage: 1+1=\correct{3}{2}

\definecolor{correct}{HTML}{009900}
\newcommand\correct[2]{\ensuremath{\:}{\color{red}{#1}}\ensuremath{\to }{\color{correct}{#2}}\ensuremath{\:}}
\newcommand\green[1]{{\color{correct}{#1}}}

% horizontal rule
\newcommand\hr{
    \noindent\rule[0.5ex]{\linewidth}{0.5pt}
}

% hide parts
\newcommand\hide[1]{}

% si unitx
\usepackage{siunitx}
\sisetup{locale = FR}

% Environments
\makeatother
% For box around Definition, Theorem, \ldots
\usepackage{mdframed}
\mdfsetup{skipabove=1em,skipbelow=0em}
\theoremstyle{definition}
\newmdtheoremenv[nobreak=true]{definitie}{Definitie}
\newmdtheoremenv[nobreak=true]{eigenschap}{Eigenschap}
\newmdtheoremenv[nobreak=true]{gevolg}{Gevolg}
\newmdtheoremenv[nobreak=true]{lemma}{Lemma}
\newmdtheoremenv[nobreak=true]{propositie}{Propositie}
\newmdtheoremenv[nobreak=true]{stelling}{Stelling}
\newmdtheoremenv[nobreak=true]{wet}{Wet}
\newmdtheoremenv[nobreak=true]{postulaat}{Postulaat}
\newmdtheoremenv{conclusie}{Conclusie}
\newmdtheoremenv{toemaatje}{Toemaatje}
\newmdtheoremenv{vermoeden}{Vermoeden}
\newtheorem*{herhaling}{Herhaling}
\newtheorem*{intermezzo}{Intermezzo}
\newtheorem*{notatie}{Notatie}
\newtheorem*{observatie}{Observatie}
\newtheorem*{oef}{Oefening}
\newtheorem*{opmerking}{Opmerking}
\newtheorem*{praktisch}{Praktisch}
\newtheorem*{probleem}{Probleem}
\newtheorem*{terminologie}{Terminologie}
\newtheorem*{toepassing}{Toepassing}
\newtheorem*{uovt}{UOVT}
\newtheorem*{vb}{Voorbeeld}
\newtheorem*{vraag}{Vraag}

\newmdtheoremenv[nobreak=true]{definition}{Definition}
\newtheorem*{eg}{Example}
\newtheorem*{notation}{Notation}
\newtheorem*{previouslyseen}{As previously seen}
\newtheorem*{remark}{Remark}
\newtheorem*{note}{Note}
\newtheorem*{problem}{Problem}
\newtheorem*{observe}{Observe}
\newtheorem*{property}{Property}
\newtheorem*{intuition}{Intuition}
\newmdtheoremenv[nobreak=true]{prop}{Proposition}
\newmdtheoremenv[nobreak=true]{theorem}{Theorem}
\newmdtheoremenv[nobreak=true]{corollary}{Corollary}

% End example and intermezzo environments with a small diamond (just like proof
% environments end with a small square)
\usepackage{etoolbox}
\AtEndEnvironment{vb}{\null\hfill$\diamond$}%
\AtEndEnvironment{intermezzo}{\null\hfill$\diamond$}%
% \AtEndEnvironment{opmerking}{\null\hfill$\diamond$}%

% Fix some spacing
% http://tex.stackexchange.com/questions/22119/how-can-i-change-the-spacing-before-theorems-with-amsthm
\makeatletter
\def\thm@space@setup{%
  \thm@preskip=\parskip \thm@postskip=0pt
}


% Exercise 
% Usage:
% \oefening{5}
% \suboefening{1}
% \suboefening{2}
% \suboefening{3}
% gives
% Oefening 5
%   Oefening 5.1
%   Oefening 5.2
%   Oefening 5.3
\newcommand{\oefening}[1]{%
    \def\@oefening{#1}%
    \subsection*{Oefening #1}
}

\newcommand{\suboefening}[1]{%
    \subsubsection*{Oefening \@oefening.#1}
}


% \lecture starts a new lecture (les in dutch)
%
% Usage:
% \lecture{1}{di 12 feb 2019 16:00}{Inleiding}
%
% This adds a section heading with the number / title of the lecture and a
% margin paragraph with the date.

% I use \dateparts here to hide the year (2019). This way, I can easily parse
% the date of each lecture unambiguously while still having a human-friendly
% short format printed to the pdf.

\usepackage{xifthen}
\def\testdateparts#1{\dateparts#1\relax}
\def\dateparts#1 #2 #3 #4 #5\relax{
    \marginpar{\small\textsf{\mbox{#1 #2 #3 #5}}}
}

\def\@lecture{}%
\newcommand{\lecture}[3]{
    \ifthenelse{\isempty{#3}}{%
        \def\@lecture{Lecture #1}%
    }{%
        \def\@lecture{Lecture #1: #3}%
    }%
    \subsection*{\@lecture}
    \marginpar{\small\textsf{\mbox{#2}}}
}



% These are the fancy headers
\usepackage{fancyhdr}
\pagestyle{fancy}

% LE: left even
% RO: right odd
% CE, CO: center even, center odd
% My name for when I print my lecture notes to use for an open book exam.
% \fancyhead[LE,RO]{Gilles Castel}

\fancyhead[RO,LE]{\@lecture} % Right odd,  Left even
\fancyhead[RE,LO]{}          % Right even, Left odd

\fancyfoot[RO,LE]{\thepage}  % Right odd,  Left even
\fancyfoot[RE,LO]{}          % Right even, Left odd
\fancyfoot[C]{\leftmark}     % Center

\makeatother




% Todonotes and inline notes in fancy boxes
\usepackage{todonotes}
\usepackage{tcolorbox}

% Make boxes breakable
\tcbuselibrary{breakable}

% Verbetering is correction in Dutch
% Usage: 
% \begin{verbetering}
%     Lorem ipsum dolor sit amet, consetetur sadipscing elitr, sed diam nonumy eirmod
%     tempor invidunt ut labore et dolore magna aliquyam erat, sed diam voluptua. At
%     vero eos et accusam et justo duo dolores et ea rebum. Stet clita kasd gubergren,
%     no sea takimata sanctus est Lorem ipsum dolor sit amet.
% \end{verbetering}
\newenvironment{verbetering}{\begin{tcolorbox}[
    arc=0mm,
    colback=white,
    colframe=green!60!black,
    title=Opmerking,
    fonttitle=\sffamily,
    breakable
]}{\end{tcolorbox}}

% Noot is note in Dutch. Same as 'verbetering' but color of box is different
\newenvironment{noot}[1]{\begin{tcolorbox}[
    arc=0mm,
    colback=white,
    colframe=white!60!black,
    title=#1,
    fonttitle=\sffamily,
    breakable
]}{\end{tcolorbox}}




% Figure support as explained in my blog post.
\usepackage{import}
\usepackage{xifthen}
\usepackage{pdfpages}
\usepackage{transparent}
\newcommand{\incfig}[1]{%
    \def\svgwidth{\columnwidth}
    \import{./figures/}{#1.pdf_tex}
}

% Fix some stuff
% %http://tex.stackexchange.com/questions/76273/multiple-pdfs-with-page-group-included-in-a-single-page-warning
\pdfsuppresswarningpagegroup=1

\title{\Huge{Analysis I}}
\author{\huge{Daniel Yu}}
\date{\today}

\begin{document}
\maketitle
\newpage% or \cleardoublepage
% \pdfbookmark[<level>]{<title>}{<dest>}
\tableofcontents
\pagebreak

\section{Sets}
Given two sets $A,B \subseteq X$:
\begin{itemize}
  \item The \textbf{union} is $A \cup B = \{x \in X | x \in A \text{ or } x \in B\}$
  \item The \textbf{intersection} is $A \cap B = \{ x \in X | x \in A \text{and } x \in B\}$ 
  \item The \textbf{difference}, $A \setminus B = \{x \in X | x \in A \text{ but } x \not\in B\}$ 
  \item The \textbf{cartesian (or "direct") product} of the two sets is $A \times B = \{(a,b) | a \in A, b \in B\}$. For example, the plane $\mathbb{R}^2 = \mathbb{R} \times \R$.
\end{itemize}

\begin{definition}
  A \textbf{countable} set is a set where each element can be mapped to a unique element of $\N$.
  A \textbf{countably infinite} set is a set that is isomorphic to $\N$.
  A \textbf{uncountable} set is a set that is not isomorphic to $\N$.
\end{definition}

\begin{theorem}
  Let $A_{\alpha} \subseteq X, \alpha \in I$ where I is a set of indicies. Then an element belongs to the intersection of all elements common to $X \ A_{\alpha}$ only when it is not in the union 
  of all $A_{\alpha}$: 
   \[
     X \setminus (\cup_{\alpha \in I} A_{\alpha}) = \cap_{\alpha \in I} (X \setminus A_{\alpha})
  .\]
  Similarly, an element is NOT in all the subsets $A_{\alpha}$ but is in $X$ when it is in any of set difference $X \setminus A_{\alpha}$
  \[
    X \setminus (\cap_{\alpha \in I} A_{\alpha}) = \cup_{\alpha \in I} (X \setminus A_{\alpha})
  .\] 

\end{theorem}

\begin{remark}
  This is similar to the idea of a center of a group in group theory: $Z(G)$ in the sense that elements must have a universal property within the set.
\end{remark}

\section{Vector Spaces}
\begin{definition}
  A (\textbf{real}) vector space $(X, "+", "\cdot")$ is a set $X$ with two operations.
\begin{enumerate}
  \item Addition: $+: X \times X \to X$, $(x,y) \implies x + y$.
  \item Scalar Multiplication: $\cdot: \R \times X \to X$, $(\alpha, x) \to \alpha x$. (generalizes to $F$ a field) 
\end{enumerate}

that satisfy the following axioms:
\begin{enumerate}
  \item $x+y = y+x$
  \item $(x+y) + z = x + (y + z)$
  \item $\exists 0, \text{ s.t.} \forall x \in X, x + 0 = x$.
  \item For any $x \in X, \exists -x, \text{ s.t. } x + (-x) = 0$.
  \item $\forall \alpha \in \R, \forall x,y, \in X, \text{ then } \alpha(x+y) = \alpha x + \alpha y$
  \item $\forall \alpha, \beta \in \R, \forall x \in X, (\alpha + \beta)x = \alpha x + \beta x$
  \item $\forall \alpha, \beta \in \R, \forall x \in X, (\alpha \beta) x = \alpha (\beta x)$
  \item $1 \cdot x = x (\forall x \in X)$
\end{enumerate}
\end{definition}

\begin{remark}
  The above definition of a vector space is only for \textbf{REAL vector spaces} where the scalar $\alpha \in \R$, but \textbf{in
  general a vector space could have the scalar $\alpha \in F$ where $F$ is any arbitrary field}, such as the complex numbers
  or otherwise.
\end{remark}


\begin{remark}
  In this course we will be dealing with real and more generally continuous vector spaces.

\end{remark}

\section{Vector Space $\R^n$}
$\R^{n} = \{\begin{pmatrix} x_1\\ \vdots\\ x_n \end{pmatrix} | x_1, \ldots, x_n \in \R \}$

Addition: $\vec{x}, \vec{y} \in \R^n, \vec{x} + \vec{y} = \begin{pmatrix} x_1 + y_1 \\ \vdots\\ x_n + y_n \end{pmatrix}$ \\
Multiplication: $\vec{x} \in \R^n, \alpha \in \R: \alpha x = \begin{pmatrix} \alpha x_1\\ \vdots\\ \alpha x_n \end{pmatrix}$

$R^{n}$ is a vector space (prove axioms easily).

\begin{remark}
  I may forget the $\vec{x}$ and only write $x$ but I mean the same.
\end{remark}

\begin{definition}
  The \textbf{dot} product (aka \textbf{euclidean scalar product} of x,y in $\R^n$ is a map, $\left< , \right>: \R^n \times \R^n \to \R$, defined as:
  \[
    \left< x,y \right> = x_1 y_1 + x_2 y_2 + \ldots + x_n y_n = \sum_{k=1}^n x_k y_k 
  .\] with the following properties:
  \begin{enumerate}
    \item \textbf{Linearity} $\forall \alpha, \beta \in \R, \left< \alpha x + \beta y, z \right> = \alpha \left< x,z \right> + \beta \left< y,z \right>$ 
    \item \textbf{Symmetry} $\left< x,y  \right> = \left< y ,x \right>$
    \item \textbf{Positive Semi-Definiteness} $\forall x \in \R^{n}, \left< x,x \right> \geq 0$ and $\left<x,x \right> = 0 \iff \vec{x} = 0$.
  \end{enumerate}

\end{definition}

\begin{remark}
  This is a special case of the inner product. 
\end{remark}

\begin{definition}
  For a general vector space, if 
  \[
  \left< , \right>: X \times X \to \R
  .\] 
  satisifies the properties of the euclidean scalar product, then it is called a euclidean scalar product in X.
\end{definition}

\begin{definition}
  The above vector space with a Euclidean Scalar Product is a vector space with a defined inner product over the real numbers,
  and is known as a \textbf{Euclidean Vector Space}.
\end{definition}

\begin{remark}
  This is different from a \textbf{Euclidean Domain} which is a concept in Number Theory of an Integral Domain equipped with a Euclidean Algorithm.
\end{remark}

\begin{definition}{Cauchy Shwartz Inequality}
  $\forall x,y \in \R^n$:
  \[
  \left< x,y  \right> \leq |x| |y|
  .\] 
  where $|x| = \sqrt{\left< x,x \right>}$. Note this is a special case of a \textbf{norm}.
\end{definition}

\begin{proof}
  Take $x,y \in \R^n$. \\

  $\forall \alpha \in \R$,
  \[
  \left< \alpha x + y, \alpha x + y \right> \geq 0.
  .\] 
  \[
  = \alpha \left< x , \alpha x + y \right> + \left< y, \alpha x + y \right>
  .\]
  \[
  = \alpha (\alpha \left< x, x \right> + \left< y, x \right>) + \alpha \left<x , y \right> + \left< y, y \right> 
  .\]
  \[
   = \alpha^2 \left< x, x \right> + \alpha \left< y, x \right> + \alpha \left<x , y \right> + \left< y, y \right> 
  .\] 
  \[
  = \alpha^2 |x|^2 + 2 \alpha \left< x, y \right> + |y|^2 \geq 0
  .\]
  This is a quadratic, so we can take the discriminant (or use the quadratic formula) to get the inequality. We know the
  discriminant is non-positive because there is at most 1 root, which is 0 by the quadratic being $\geq 0$. \\
  Recall that the discrimanant is: $b^2 - 4ac$:

  \[
    (2 \left< x,y \right>)^2- 4 |x|^2 |y|^2 \leq 0
  .\] 
  \[
  2 \left< x,y \right> \leq 2 |x| |y|
  .\] 
  \[
  \left< x,y \right> \leq |x| |y|
  .\] 
\end{proof}

\begin{definition}
  The \textbf{norm} in general is defined as follows. Let $X$ be a real-vector space and let us have a map:
  \begin{align*}
       \mid \mid  \cdot  \mid \mid  : X &\longrightarrow \R \\
    x &\longmapsto  \mid \mid  x  \mid \mid  
  .\end{align*}
  which satisfies the following properties. Properties: 
  \begin{itemize}
    \item $\forall x \in \R^n$, $|x| \geq 0$ and  $|x| = 0 \iff x = 0$
    \item $\forall \alpha \in \R, \forall x \in \R^n$, $|\alpha x|= |\alpha| |x|$
    \item \textbf{Triangle inequality}:  $\forall x,y \in \R^n$, $|x + y| \leq |x| + |y|$. 
  \end{itemize}

  Correspondingly, $\left( X,  \mid  \mid \cdot  \mid  \mid  \right)$ is called a \textbf{normed space}.
\end{definition}

\begin{definition}
  The \textbf{euclidean norm} in $\R^n$ is defined as $$|x| = \sqrt{\left<x,x \right>} = \sqrt{\sum_{k=1}^n x_k^2}$$ \\
  The euclidean norm satisfies the cauchy swhartz inequality and all 3  properties above!
 \end{definition}

\begin{proof}{Proof of Triangle Inequality}
  \begin{align*}
    |x+y|^2 &= \left<x + y, x+y \right> \\ 
            &= |x|^2 + |y|^2 + 2\left< x , y\right> \\  
            &\leq |x|^2 + |y|^{2} + 2|x| |y|  \text{  by the cauchy shwartz inequality} \\
            &\leq (|x| + |y|)^2
  .\end{align*}
  Thus, $|x+y| \leq |x| + |y|$
\end{proof}


\begin{remark}{Example}
\[
  \text{Let} X = \R^n
.\] 
Other Examples of defined norms include:
\[
  \mid  \mid x \mid  \mid _p = \left(\sum_{k=1}^n x_k^p  \right) ^{\frac{n}{p}}
.\] 
\[
\[
  \mid  \mid x  \mid  \mid_\infty = \max_{1 \leq k \leq n} |x_k| \text{  where x = ($x_1,\ldots,x_n$)}
.\] 
\[
  \mid  \mid x \mid  \mid_1 = \sum_{k=1}^n |x_k|
.\]  

\end{remark}
\end{document}


