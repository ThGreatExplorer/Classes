\documentclass[a4paper]{article}
\usepackage[a4paper, margin=1in]{geometry}
\input{preamble}
\title{\Huge{Analysis I - Hw 1}}
\author{\huge{Daniel Yu}}
\date{September 23, 2024}

\pdfsuppresswarningpagegroup=1

\begin{document}
\maketitle
\newpage% or \cleardoublepage
% \pdfbookmark[<level>]{<title>}{<dest>}
\pagebreak

\section{Problem 1}
Let $(X,\rho)$ be a metric space and  $E$, a non-empty subset in $X$. Consider the new metric space,
$(E,\rho)$. Prove that a set $U \subseteq E$ is open in $E$ $\iff$ $\exists $ an open set $\tilde{U}$ in $X$ 
such that  $U = \tilde{U} \cap E$. Similarly, prove this for closed sets.

\begin{proof}
  $\to$ Assume that there is a set $U \subseteq E$ that is open in  $(E, \rho)$ and  $E \subseteq X$  where  $X$ forms a metric space  $(X, \rho)$. If the open set  $U$ is also open in  $(X, \rho)$ then $\tilde{U} = U$ and we are done, so assume that  $U$ is not open in  $(X, \rho)$. We can construct an open set $\tilde{U}$ in $(X, \rho)$ by taking the ball  $B_r(x)' \subseteq X$ that contains  $B_r(x) \subseteq U$. $B_r(x)' \supseteq B_r(x)$ since $U \subseteq X$. We can do this $\forall x \in U$ with some $r > 0$. By this construction,  $\tilde{U}$ is an open set because for all  $x \in U$,  $\exists r > 0$ such that $B_r(x)' \subseteq X$. Then, $B_r(x)' \cap E = B_r(x)$ since $B_r(x)'$ contains $B_r(x) \subseteq U \subseteq E$ and points in $B_r(x)' \setminus E$ i.e. $B_r(x)' = B_r(x) \cup B_r(x)' \setminus E$. So, 
   \[
     (\cup_{x \in U} B_r(x)') \cap E = \cup_{x \in U} B_r(x)  
   .\] 
 and every $x \in U$ has an open set  $B_r(x)\subseteq E$, which is precisely the definition of the open set $U$ in $E$! 

$\leftarrow$ Assume that  $\exists $ an open set $\tilde{U} \subseteq X$ such that $U = \tilde{U} \cap E$. 
By set theory, $U \subseteq E$. We know that $\tilde{U}$ is open, so 
\[
  \forall x \in \tilde{U}, \exists r > 0, \text{ such that } B_r(x) \subseteq \tilde{U} 
.\] 
Consider the ball  $B_r(x) \cap E$, the intersection of the ball  $B_r(x) \subseteq X$ with $E$. We know that 
this intersection must be open in $E$ because we are taking an open set in $X$ and intersecting it with
$E$ which is open with respect to itself. Since $B_r(x) \subseteq \tilde{U}$, then 
$B_r(x) \cap E \subseteq \tilde{U} \cap E = U$.  So for every $x \in U$, $\exists  r > 0$
such that an open set is formed in $E$:
 \[
   B_r(x) \cap E \subseteq U
.\] 
$U \subseteq E$ is open in $E$. \\


Now to prove the same for closed sets. We just proved that a set $U \subseteq E$ is open in $E$ $\iff $ $\exists $ an open set $\tilde{U}$ in  $X$ such that $U = \tilde{U} \cap E$. We know that the complement of an open set is a closed set. Thus, taking the complement, we know that there is a set  $C = E \setminus U$ where $C \subseteq E$ that is closed in  $E$. It follows that the open set $\tilde{U}$ in  $X$ has a complement $\tilde{C} = X \setminus \tilde{U}$ that is closed in $X$. Then 
 \[
   \tilde{C} \cap E = (X \setminus \tilde{U}) \cap E = (X \cap E) \setminus (\tilde{U} \cap E) = X \setminus U = C
.\] 
Thus, there exists a closed set $C \subseteq E$ $\iff$ $\exists $ a closed set $\tilde{C}$ in  $X$ such that  $C = \tilde{C} \cap E$.  
\end{proof}

\begin{note}{Intuition}\\
\textit{Consider $U \subseteq E$ an open set. This means that $\forall x \in U$  $\exists r > 0$ such that 
\[
B_r(x) \subseteq U
.\] 
If we consider $U$ in  $X \supseteq E$, then  $U$ may not be necessarily be open in  $X$ because there may
 $x' \in X$ but  $x' \not\in E$ such that $\exists B_r(x)$ $\forall r > 0$ such that $x' \in B_r(x)$. Since $B_r(x) \subseteq U \subseteq E$,
 if $x' \in B_r(x)$, then:
 \[
  B_r(x) \not\subseteq U
 .\] 
 For example, let $X = \R$, the closed interval $[0,1)$ would not be open because there is no ball centered at
 $B_r(0)$ of any radius greater than 0 that is a subset of $[0,1)$. However, if we restrict $X = [0,1)$, then
$[0,1)$ becomes open since $B_r(0)=[0,r) \subseteq [0,1)$ when  $r < 1$ (now the  $x < 0$ don't exist). 
The idea is that we can find an open set $\tilde{U}$ in X that is an analogue of  $U$ in  $E$.}
\end{note}

\section{Problem 2}
Given $K \subseteq E$, then prove $K$ is compact in $E$ $\iff$ $K$ is compact in $X$. 

\begin{proof}
  $\to$ If $K$ is compact in  $E \subseteq X$, then for any open cover $\{U_{\alpha}\}_{\alpha \in I}$ in $E$ such that
  $K \subseteq \cup_{\alpha \in I} U_{\alpha}$ 
  that covers $K$ and there is some finite subcover $\{U_{\alpha_1}, U_{\alpha_2}, \ldots, U_{\alpha_n}\} \subseteq E$ that
   covers $K$. Since $E \subseteq X$, we can use the statement from problem 1. For each open subset $U_{\alpha_i} \subseteq E$, there exists open subset $U_{\beta_i} \subseteq X$ such that $U_{\alpha_i} = U_{\beta_i} \cap E \to U_{\alpha_i} \subseteq U_{\beta_i}$. So,
\[
K \subseteq \cup_{\alpha_i \in I_1} U_{\alpha_i} \subseteq \cup_{\beta_i \in I_1} U_{\beta_i}
.\] 
Thus, we can construct a finite subcover $\{U_{\beta_1}, U_{\beta_2}, \ldots, U_{\beta_n}\}$ of $K$ in $X$. As any open cover of $K$ in  $X$ can be restricted $\{U_{\beta} \cap E\}_{\beta \in I}$ to be an open cover of $K$ in  $E$ and any open cover of  $K$ in  $E$ can be augmented to be an open  cover of  $K$ in  $X$,  $\{\{U_{\alpha}\}_{\alpha \in I}, X \setminus E\}$, then any open cover in $X$ can be mapped to some open cover in $E$ such that we can follow the construction above to create a finite subcover of $K$ in  $X$. So, $K$ is compact in  $X$. \\
  
  $\leftarrow$ Assume that $K$ is compact in $X$ and  $K \subseteq E$. This means for any open cover of $K$ in $X$:
  \[
  \{U_{\alpha}\}_{\alpha \in I} 
  .\]
  there exists a finite subcover $\{U_{\alpha_1}, \ldots, U_{\alpha_n}\} $ such that 
  \[
  K \subseteq \cup_{\alpha \in I_1} U_{\alpha} \subseteq X
  .\] 
  Then since $E \subseteq X$, we can use the statement from problem 1. For each open subset $U_{\alpha_i} \subseteq X$, there exists open subset $U_{\beta_i} \subseteq E$ such that $U_{\beta_i} = U_{\alpha_i} \cap E \to U_{\beta_i} \subseteq E$. And since, $K \subseteq E$, 
  \[
  U_{\alpha_i} \cap K \subseteq U_{\alpha_i} \cap E = U_{\beta_i}
  .\] 
  so,
  \[
    (\cup_{\alpha \in I_1} U_{\alpha_i}) \cap K = K \subseteq (\cup_{\alpha \in I_1} U_{\beta_i}) \cap E = \cup_{\beta \in I_1} U_{\beta_i}
  .\]
  and $\{U_{\beta_1}, \ldots, U_{\beta_n}\}$ is an finite subcover of $K$ in $E$.
  Then the open cover of $K$ in $E$ would just be $\{U_{\beta}\}_{\beta \in I}$. As any open cover of $K$ in  $X$ can be restricted $\{U_{\alpha} \cap E\}_{\alpha \in I}$ to be an open cover of $K$ in  $E$ and any open cover of  $K$ in  $E$ can be augmented to be an open  cover of  $K$ in  $X$,  $\{\{U_{\beta}\}_{\beta \in I}, X \setminus E\}$, then for any open cover in $E$ which can be mapped to some open cover in $X$, we can follow the construction above to create a finite subcover of $K$ in $E$ and $K$ is compact. 
\end{proof}

\Huge{Due October 2nd}

\end{document}
