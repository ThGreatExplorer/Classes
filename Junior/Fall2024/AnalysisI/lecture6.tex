\documentclass[a4paper]{article}
\usepackage[a4paper, margin=1in]{geometry}
% Some basic packages
\usepackage[utf8]{inputenc}
\usepackage[T1]{fontenc}
\usepackage{textcomp}
\usepackage[dutch]{babel}
\usepackage{url}
\usepackage{graphicx}
\usepackage{float}
\usepackage{booktabs}
\usepackage{enumitem}

\pdfminorversion=7

% Don't indent paragraphs, leave some space between them
\usepackage{parskip}

% Hide page number when page is empty
\usepackage{emptypage}
\usepackage{subcaption}
\usepackage{multicol}
\usepackage{xcolor}

% Other font I sometimes use.
% \usepackage{cmbright}

% Math stuff
\usepackage{amsmath, amsfonts, mathtools, amsthm, amssymb}
% Fancy script capitals
\usepackage{mathrsfs}
\usepackage{cancel}
% Bold math
\usepackage{bm}
% Some shortcuts
\newcommand\N{\ensuremath{\mathbb{N}}}
\newcommand\R{\ensuremath{\mathbb{R}}}
\newcommand\Z{\ensuremath{\mathbb{Z}}}
\renewcommand\O{\ensuremath{\emptyset}}
\newcommand\Q{\ensuremath{\mathbb{Q}}}
\newcommand\C{\ensuremath{\mathbb{C}}}

% Easily typeset systems of equations (French package)
\usepackage{systeme}

% Put x \to \infty below \lim
\let\svlim\lim\def\lim{\svlim\limits}

%Make implies and impliedby shorter
\let\implies\Rightarrow
\let\impliedby\Leftarrow
\let\iff\Leftrightarrow
\let\epsilon\varepsilon

% Add \contra symbol to denote contradiction
\usepackage{stmaryrd} % for \lightning
\newcommand\contra{\scalebox{1.5}{$\lightning$}}

% \let\phi\varphi

% Command for short corrections
% Usage: 1+1=\correct{3}{2}

\definecolor{correct}{HTML}{009900}
\newcommand\correct[2]{\ensuremath{\:}{\color{red}{#1}}\ensuremath{\to }{\color{correct}{#2}}\ensuremath{\:}}
\newcommand\green[1]{{\color{correct}{#1}}}

% horizontal rule
\newcommand\hr{
    \noindent\rule[0.5ex]{\linewidth}{0.5pt}
}

% hide parts
\newcommand\hide[1]{}

% si unitx
\usepackage{siunitx}
\sisetup{locale = FR}

% Environments
\makeatother
% For box around Definition, Theorem, \ldots
\usepackage{mdframed}
\mdfsetup{skipabove=1em,skipbelow=0em}
\theoremstyle{definition}
\newmdtheoremenv[nobreak=true]{definitie}{Definitie}
\newmdtheoremenv[nobreak=true]{eigenschap}{Eigenschap}
\newmdtheoremenv[nobreak=true]{gevolg}{Gevolg}
\newmdtheoremenv[nobreak=true]{lemma}{Lemma}
\newmdtheoremenv[nobreak=true]{propositie}{Propositie}
\newmdtheoremenv[nobreak=true]{stelling}{Stelling}
\newmdtheoremenv[nobreak=true]{wet}{Wet}
\newmdtheoremenv[nobreak=true]{postulaat}{Postulaat}
\newmdtheoremenv{conclusie}{Conclusie}
\newmdtheoremenv{toemaatje}{Toemaatje}
\newmdtheoremenv{vermoeden}{Vermoeden}
\newtheorem*{herhaling}{Herhaling}
\newtheorem*{intermezzo}{Intermezzo}
\newtheorem*{notatie}{Notatie}
\newtheorem*{observatie}{Observatie}
\newtheorem*{oef}{Oefening}
\newtheorem*{opmerking}{Opmerking}
\newtheorem*{praktisch}{Praktisch}
\newtheorem*{probleem}{Probleem}
\newtheorem*{terminologie}{Terminologie}
\newtheorem*{toepassing}{Toepassing}
\newtheorem*{uovt}{UOVT}
\newtheorem*{vb}{Voorbeeld}
\newtheorem*{vraag}{Vraag}

\newmdtheoremenv[nobreak=true]{definition}{Definition}
\newtheorem*{eg}{Example}
\newtheorem*{notation}{Notation}
\newtheorem*{previouslyseen}{As previously seen}
\newtheorem*{remark}{Remark}
\newtheorem*{note}{Note}
\newtheorem*{problem}{Problem}
\newtheorem*{observe}{Observe}
\newtheorem*{property}{Property}
\newtheorem*{intuition}{Intuition}
\newmdtheoremenv[nobreak=true]{prop}{Proposition}
\newmdtheoremenv[nobreak=true]{theorem}{Theorem}
\newmdtheoremenv[nobreak=true]{corollary}{Corollary}

% End example and intermezzo environments with a small diamond (just like proof
% environments end with a small square)
\usepackage{etoolbox}
\AtEndEnvironment{vb}{\null\hfill$\diamond$}%
\AtEndEnvironment{intermezzo}{\null\hfill$\diamond$}%
% \AtEndEnvironment{opmerking}{\null\hfill$\diamond$}%

% Fix some spacing
% http://tex.stackexchange.com/questions/22119/how-can-i-change-the-spacing-before-theorems-with-amsthm
\makeatletter
\def\thm@space@setup{%
  \thm@preskip=\parskip \thm@postskip=0pt
}


% Exercise 
% Usage:
% \oefening{5}
% \suboefening{1}
% \suboefening{2}
% \suboefening{3}
% gives
% Oefening 5
%   Oefening 5.1
%   Oefening 5.2
%   Oefening 5.3
\newcommand{\oefening}[1]{%
    \def\@oefening{#1}%
    \subsection*{Oefening #1}
}

\newcommand{\suboefening}[1]{%
    \subsubsection*{Oefening \@oefening.#1}
}


% \lecture starts a new lecture (les in dutch)
%
% Usage:
% \lecture{1}{di 12 feb 2019 16:00}{Inleiding}
%
% This adds a section heading with the number / title of the lecture and a
% margin paragraph with the date.

% I use \dateparts here to hide the year (2019). This way, I can easily parse
% the date of each lecture unambiguously while still having a human-friendly
% short format printed to the pdf.

\usepackage{xifthen}
\def\testdateparts#1{\dateparts#1\relax}
\def\dateparts#1 #2 #3 #4 #5\relax{
    \marginpar{\small\textsf{\mbox{#1 #2 #3 #5}}}
}

\def\@lecture{}%
\newcommand{\lecture}[3]{
    \ifthenelse{\isempty{#3}}{%
        \def\@lecture{Lecture #1}%
    }{%
        \def\@lecture{Lecture #1: #3}%
    }%
    \subsection*{\@lecture}
    \marginpar{\small\textsf{\mbox{#2}}}
}



% These are the fancy headers
\usepackage{fancyhdr}
\pagestyle{fancy}

% LE: left even
% RO: right odd
% CE, CO: center even, center odd
% My name for when I print my lecture notes to use for an open book exam.
% \fancyhead[LE,RO]{Gilles Castel}

\fancyhead[RO,LE]{\@lecture} % Right odd,  Left even
\fancyhead[RE,LO]{}          % Right even, Left odd

\fancyfoot[RO,LE]{\thepage}  % Right odd,  Left even
\fancyfoot[RE,LO]{}          % Right even, Left odd
\fancyfoot[C]{\leftmark}     % Center

\makeatother




% Todonotes and inline notes in fancy boxes
\usepackage{todonotes}
\usepackage{tcolorbox}

% Make boxes breakable
\tcbuselibrary{breakable}

% Verbetering is correction in Dutch
% Usage: 
% \begin{verbetering}
%     Lorem ipsum dolor sit amet, consetetur sadipscing elitr, sed diam nonumy eirmod
%     tempor invidunt ut labore et dolore magna aliquyam erat, sed diam voluptua. At
%     vero eos et accusam et justo duo dolores et ea rebum. Stet clita kasd gubergren,
%     no sea takimata sanctus est Lorem ipsum dolor sit amet.
% \end{verbetering}
\newenvironment{verbetering}{\begin{tcolorbox}[
    arc=0mm,
    colback=white,
    colframe=green!60!black,
    title=Opmerking,
    fonttitle=\sffamily,
    breakable
]}{\end{tcolorbox}}

% Noot is note in Dutch. Same as 'verbetering' but color of box is different
\newenvironment{noot}[1]{\begin{tcolorbox}[
    arc=0mm,
    colback=white,
    colframe=white!60!black,
    title=#1,
    fonttitle=\sffamily,
    breakable
]}{\end{tcolorbox}}




% Figure support as explained in my blog post.
\usepackage{import}
\usepackage{xifthen}
\usepackage{pdfpages}
\usepackage{transparent}
\newcommand{\incfig}[1]{%
    \def\svgwidth{\columnwidth}
    \import{./figures/}{#1.pdf_tex}
}

% Fix some stuff
% %http://tex.stackexchange.com/questions/76273/multiple-pdfs-with-page-group-included-in-a-single-page-warning
\pdfsuppresswarningpagegroup=1

\title{\Huge{Analysis I}\\Lecture 6}
\author{\huge{Daniel Yu}}
\date{October 16, 2024}

\pdfsuppresswarningpagegroup=1

\begin{document}
\maketitle
\newpage% or \cleardoublepage
% \pdfbookmark[<level>]{<title>}{<dest>}
\tableofcontents
\pagebreak

\section{Uniformly Continuous Maps}

\begin{definition}
  Let $(X, \rho_X), (Y, \rho_Y)$ metric spaces and  $f:X \to Y$ is \textbf{uniformly continous} if $\forall \epsilon > 0, \exists \delta > 0$ such that $\forall x,y \in X, \rho_X(x,y) < \delta$ then  $\rho_{Y}(f(x), f(y)) < \epsilon$.  

  \begin{enumerate}
    \item imposing a stronger constraint than continuous because this is saying some $\delta$ works across the entire domain for  $x,y \in X$ 
    \item stating that  $f$ behaves "similarly" across the whole domain
    \item a function that is continuous at every point $\not\iff$ the function is uniformly continuous
    \item consider behavior of functions with asymptotes, what can you say about them?
  \end{enumerate}
\end{definition}

\begin{remark}
  Recall that the definition of a continuous map is as follows. $f \in C(X,Y) \iff \forall x \in X, \forall \epsilon > 0$ $\exists \delta > 0$ such that $\forall y \in X$, $\rho(y,x) < \delta$ we have $\rho (f(x),f(y)) < \epsilon$
\end{remark}

\begin{lemma}
  If $f:X \to Y$ is uniformly continuous then  $f \in C(X,Y)$ i.e. it is continuous.
\end{lemma}

\begin{note}
  This is not true the other way around. A map that is continuous but not uniformly continuous. $f : [0,1]  
  \to \R$. $x_n = \frac{1}{n}, n \geq 1$. Consider
  \[
    \rho(x_n, x_{n+1}) = \mid x_n + x_{n+1} \mid  = \frac{1}{n} - \frac{1}{n+1} = \frac{1}{n(n+1)} \to 0 \text{ as $n \to \infty$}
  .\] 
  and,
  \[
  \rho(f(x_n), f(x_{n+1})) = \mid f(x_n) - f(x_{n+1}) \mid  = \mid n-(n+1) \mid = 1
  .\] 
  Hence this map is continuous but not uniformly continuous (the same $\delta $ won't work across all $x_n, x_{n+1}$ pairs).   
\end{note}

\begin{definition}
  f is not uniformly continuous $\iff$ $\exists \epsilon > 0$ $\forall \delta > 0$  $\exists x,y \in X$ $\rho_X (x,y) < \delta$ such that  $\delta_Y (f(x), f(y)) \geq \epsilon$
\end{definition}

\begin{theorem}
  Assume that $f:X \to Y$ is continuous and  $X$ is compact. Then,  $f$ is uniformly continuous.

   \begin{proof}
    By contradition. Assume that $f:X \to Y$ is continuous and  $X$ is comapct but  $f$ is not uniformly continuous. Then, $\exists $ $\epsilon_0 > 0$ such that  $\forall \delta > 0$  $\exists x,y \in X$, $\rho_X(x,y) < \delta$ and  $\rho_{Y} (f(x), f(y)) \geq \epsilon_0$. We can choose a construction like so
    \begin{enumerate}
      \item $x_1,y_1 \in X$ such that $\rho_X (x_1,y_1) < 1$ and  $\rho_Y(f(x_1),f(y_1)) \geq \epsilon_0$
      \item  $x_2,y_2 \in X$ such that $\rho_X (x_2,y_2) < \frac{1}{2}$ and  $\rho_Y(f(x_2),f(y_2)) \geq \epsilon_0$.
      \item $\ldots$
      \item  $x_n,y_n \in X$ such that $\rho_X (x_n,y_n) < \frac{1}{n}$ and  $\rho_Y(f(x_n),f(y_n)) \geq \epsilon_0$
    \end{enumerate}
    Since $X$ is compact, then  $\exists $ a subsequence $\{x_{n_j}\} $ such that $x_{n_j} \to x^{*}$ as $j \to \infty$. Then $f \in C(x^{*}) \Rightarrow \delta_0 > 0$ such that
    \[
    f(B_{\delta_0}^{X} (x^*)) \subseteq B_{\frac{\epsilon_0}{2}}^{Y}(f(x^*))
    .\]
    Since $x_{n_j} \to x^{*}$ in $X$ as  $j \to \infty$ then  $\exists j_0 \geq 1$ such that $x_{n_j} \in B_{\frac{\delta_0}{2}}\left( x^{*} \right) , \forall j \geq j_0$. Now choose $j_0 \geq 1$ larger if necessary so that 
     \[
    \frac{1}{n_j} \leq \frac{\delta}{2}, \forall j \geq j_0
    .\] 
  Then by triangle inequality
  \begin{align*}
    \rho_X (y_{n_j}, x^{*}) &\leq \rho_X (x^{*}, x_{n_j}) + \rho_X(x_{n_j}, y_{n_j})\\
                            &< \frac{\delta_0}{2} + \frac{1}{n_j} \\
                            &< \frac{\delta_0}{2} + \frac{\delta_0}{2} \\
                            &= \delta_0
  .\end{align*}
  $\Rightarrow$ 
   \[
  x_{n_j}, y_{n_j} \in B_{\frac{\delta_0}{2}}^{X} (x^{*}), \forall j \geq j_0
  .\]
  $\Rightarrow$
   \[
  f(x_{n_j}), f(y_{n_j}) \in B_{\frac{\epsilon_0}{2}}^{Y} (y^{*}) 
  .\] 
  $\Rightarrow$
   \[
  \rho_{Y} (f(x_{n_j}), f(y_{n_j})) < \epsilon_0 \forall j \geq j_0
  .\] 
  This is a contradiction to the construction of $x_n,y_n, n \geq 1$ since  $\rho_Y (f(x_n), f(y_n)) \geq \epsilon_0 \forall n \geq 1$.
\end{proof}

\begin{remark}
  What we essentially did in the proof above is that we took two sequences $x_i$'s and  $y_i$'s that both converge to some point  $x^{*}$ and as they converge the distance $\rho_{X}(x_n,y_n) \to 0$ as $n \to \infty$. So in $Y$, bothe sequences in  $Y$ must converge to  $f(x^{*})$ and similarily the distances in $Y$ of  $\rho_Y (f(x_n), f(y_n)) \to 0$.  
\end{remark}
\end{theorem}

\end{document}
