\documentclass[a4paper]{article}
\usepackage[a4paper, margin=1in]{geometry}
\input{preamble}
\title{\Huge{Analysis}\\ Taylor's Theorem}
\author{\huge{Daniel Yu}}
\date{October 6, 2024}

\pdfsuppresswarningpagegroup=1

\begin{document}
\maketitle
\newpage% or \cleardoublepage
% \pdfbookmark[<level>]{<title>}{<dest>}
\tableofcontents
\pagebreak

\section{Taylor's Theorem}
\begin{lemma}
  Let
  \[
  Q(x) = a_{m} x^{m} + a_{m-1}x^{m-1} + \ldots + a_1 x + a_0
  .\] 
  a polynomial with real coefficients. Then $Q \in D(\R)$ and $Q'(x) = a_m m x^{m-1} + \ldots + a_1$ ($\Rightarrow Q \in C(\R)$).  

  \noindent\hrulefill

  \begin{proof}.
    Take $x_0 \in \R$ and $f(x) = x^{m}$. Then, 
   \begin{align*}
     (x_0 + h)^{m} &= x_0^{m} + \begin{pmatrix} m\\ 1 \end{pmatrix} x_0^{m-1}h +  \{ \begin{pmatrix} m\\ 2 \end{pmatrix}x_0^{m-2} h^{2} + \ldots + h^{m} \}  \\
                   &=  x_0^{m} + \begin{pmatrix} m\\ 1 \end{pmatrix} x_0^{m-1}h +  h \{ \begin{pmatrix} m\\ 2 \end{pmatrix}x_0^{m-2} h^{1} + \ldots + h^{m-1} \}   \\
                   &= x_{0}^{m}  + m x_0^{m-1}h +  h  \cdot r(h) 
   .\end{align*}
    Notice that $f'(x_0) = m x_0^{m-1}$, so $x^{m} \in D(x_0)$ and $(x^{m}) \mid_{x = x_0} = m x_0^{m-1}$.  
      
  \end{proof}
\end{lemma}

\begin{definition}
  Define the Taylor's polynomial of degree $n$ at point  $x_0$ is a polynomial approximation of the function $f(x)$ at the point $x_0$ which matches the functions value and first  $n$ derivatives at that point. It describes the \textit{neighborhood around  $f(x_0)$}
    \[
   T_{n}(x) = f(x_0) + \frac{f'(x_0)}{1!} (x-x_0) + \ldots + \frac{f^{n}( x_0)}{(n)!} (x-x_0)^{n}
   .\] 
\end{definition}

\begin{definition}
  \[
  f \in D^{m}((a,b)) \text{ if $f, f', \ldots, f^{m-1} \in D((a,b))$}
  .\] 
  \[
  f \in C^{m} ((a,b)) \text{ if $f, f', \ldots, f^{m=1}, f^{m} \in C((a,b))$} 
  .\] 
\end{definition}

\begin{note}
  $f \in D((a,b)) \Rightarrow f \in C^{m-1} ((a,b))$
\end{note}

\begin{theorem}{Taylor's Theorem (Lagrange)}\\
  Assume that $f:(a,b) \to \R$ and $f \in D^{n} ((a,b))$ where $n \geq 1$,  $x \in (a,b)$. Then,  $\forall x \in (a,b)$ $\exists x_{*} \in (x_0,x)$ [or $x_{*} \in (x, x_0)$ if $x \leq x_0$] such that:


\[
f(x) = f(x_0) + f'(x_0)(x - x_0) + \frac{f''(x_0)}{2!}(x - x_0)^2 + \dots + \frac{f^{(n)}(x_0)}{n!}(x - x_0)^n + \frac{f^{(n+1)}(x_{*})}{(n+1)!}(x - x_0)^{n+1}.
\]
 
Note that $\frac{f^{(n+1)}(x_{*})}{(n+1)!}(x - x_0)^{n+1}$ is known as the \textit{Langrangian Error Term} which represents the difference between the taylor polynomial $T_n$ and  $f(x)$

  
  \noindent\hrulefill

  \begin{proof}
   Consider the taylor polynomial,  
   \[
   T_{n-1}(x) = f(x_0) + \frac{f'(x_0)}{1!} (x-x_0) + \ldots + \frac{f^{n-1}( x_0)}{(n-1)!} (x-x_0)^{n-1}
   .\] 
   where $T_{n-1}(x_0) = f(x_0), T^{1}_{n-1} (x_0)= f'(x_0), \ldots, T^{n-1}_{n-1} (x_0) = f^{n-1} (x_0)$
    We will prove the taylors theorem with  $x$ replaced by  $x_1$. Then, take $x_0,x_1 \in (a,b)$, $x_1 \neq x_0$. First, we find $M \in \R$ such that:
    \[
    f(x_1) = T_{n-1} (x_1) + \frac{M}{n!} (x_1 - x_0)^{n}
    .\]
    Since $M$ obviously exists since we can solve th elinear equation. Our task now is to show that  $M = f^{n}(x_{*}) $ for some $x_{*} \in (x_0,x_1)$. Consider the function,
    \[
    g(x) = f(x) - T_{n-1} (x) - \frac{M}{n!} (x-x_0)^{n}
    .\] 
    Then,
    \begin{enumerate}
      \item Take $g(x_0) = 0$, $g'(x_0) = 0, \ldots ,g^{n-1} (x_0) = 0$.
      \item Take $g(x_1) = 0$.
    \end{enumerate}
    Then, $g \in D((x_0,x_1)) \cap C([x_0,x_1])$. By the mean-value theorem, $\exists  x_1 \in (a,b)$ such that 
    \[
    g'(x_2) = 0
    .\] 
    Since by (1),  $g'(x_{0}) = 0$, and $g' \in D((x_0,x_1)) \cap C([x_0, x_1])$ we find $x_3 \in (x_0,x_2)$ such that $g''(x_3) = 0$. (apply the mean value theorem again, this time with endpoints $\{x_0,x_2\} $). \\

  Continuing this process, we find: $\exists  x_{n+1} \in (x_0, x_{n}) $ such that, 
  \[
  g^{n} (x_{n+1}) =0
  .\] 
  Then we set $x_{*} = x_{n+1}$. Then $g^{n} (x_{*}) = 0$. Clearly, $x_{*} \in (x_0,x_1)$. But $g^{n} (x) = f^{n}(x) - M \Rightarrow 0 = g^{n} (x_{*}) = f^{n} (x_{*}) -M \Rightarrow M = f^{n} (x_{*}) $

  \end{proof}
\end{theorem}

\begin{lemma}
For any $a > 0$ we have that 
\[
\lim_{n \to \infty} \frac{a^{n}}{n!} = 0
.\] 

\begin{proof}
  Take $a > 0$. Then,  \[
  \frac{a^{n}}{n!} = \frac{a}{1} \cdot \frac{a}{2} \cdot \frac{a}{3}  \cdot \ldots \cdot \frac{a}{n}
  .\] 
  choose $n_0 \geq 1$ such that $\frac{a}{n_0} < \frac{1}{2}$. Then, for $n \geq n_0$, 
  \begin{align*}
    \frac{a}{n!} &= [\frac{a}{1}\frac{a}{2}\frac{a}{3} \ldots \frac{a}{n_0}] [\frac{a}{n_0+1} \ldots \frac{a}{n}] \\
                 &\leq [\frac{a}{1}\frac{a}{2}\frac{a}{3} \ldots \frac{a}{n_0}] (\frac{1}{2})^{n-n_0}, n \geq n_0 
  .\end{align*}
  This decays to 0 as $n \to \infty$, so $\frac{a^{n}}{n!} \to 0$
\end{proof}
\end{lemma}

\begin{lemma}{Stirling Approximation}\\
  $n! \sim \sqrt{2 \pi n} \left( \frac{n}{e} \right)^{e}  $ (rate of growth)
\end{lemma}

\section{Series}
 \begin{definition}
$(a_{n})_{n \geq 1}, a_n \in \R $ (or $\C$),  $n \geq 1$. We write 
   \[
  \sum^{\infty}_{n=1} a_n = A 
  .\] 
  and say that that the series converges to $A$ if the partial sums  $S_N = \sum_{n=1}^{N} a_n$ converges to $A$, i.e. $S_N \to A$ ( $N \to \infty$). \\

  Equivalently,  $\mid \sum_{n=1}^{N} a_n - A \mid  \to 0$ as ($N \to \infty$). 
\end{definition}

\begin{definition}
  For series, we will often consider the "upper limit",
  \[
    \overline{\lim_{n \to \infty}} x_n = \lim_{n \to \infty} (\sup \{x_n \mid n \geq N\}) 
  .\] 
  \begin{enumerate}
    \item $\overline{\lim_{n \to \infty}} \mid a_\frac{n}{a_{n+1}}\mid  < 1$ converges
    \item $\overline{\lim_{n \to \infty}} \sqrt{\mid a_n\mid }^{n} < 1$ converges
  \end{enumerate}
\end{definition}
\end{document}
