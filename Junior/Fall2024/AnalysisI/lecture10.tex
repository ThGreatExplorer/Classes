\documentclass[a4paper]{article}
\usepackage[a4paper, margin=1in]{geometry}
% Some basic packages
\usepackage[utf8]{inputenc}
\usepackage[T1]{fontenc}
\usepackage{textcomp}
\usepackage[dutch]{babel}
\usepackage{url}
\usepackage{graphicx}
\usepackage{float}
\usepackage{booktabs}
\usepackage{enumitem}

\pdfminorversion=7

% Don't indent paragraphs, leave some space between them
\usepackage{parskip}

% Hide page number when page is empty
\usepackage{emptypage}
\usepackage{subcaption}
\usepackage{multicol}
\usepackage{xcolor}

% Other font I sometimes use.
% \usepackage{cmbright}

% Math stuff
\usepackage{amsmath, amsfonts, mathtools, amsthm, amssymb}
% Fancy script capitals
\usepackage{mathrsfs}
\usepackage{cancel}
% Bold math
\usepackage{bm}
% Some shortcuts
\newcommand\N{\ensuremath{\mathbb{N}}}
\newcommand\R{\ensuremath{\mathbb{R}}}
\newcommand\Z{\ensuremath{\mathbb{Z}}}
\renewcommand\O{\ensuremath{\emptyset}}
\newcommand\Q{\ensuremath{\mathbb{Q}}}
\newcommand\C{\ensuremath{\mathbb{C}}}

% Easily typeset systems of equations (French package)
\usepackage{systeme}

% Put x \to \infty below \lim
\let\svlim\lim\def\lim{\svlim\limits}

%Make implies and impliedby shorter
\let\implies\Rightarrow
\let\impliedby\Leftarrow
\let\iff\Leftrightarrow
\let\epsilon\varepsilon

% Add \contra symbol to denote contradiction
\usepackage{stmaryrd} % for \lightning
\newcommand\contra{\scalebox{1.5}{$\lightning$}}

% \let\phi\varphi

% Command for short corrections
% Usage: 1+1=\correct{3}{2}

\definecolor{correct}{HTML}{009900}
\newcommand\correct[2]{\ensuremath{\:}{\color{red}{#1}}\ensuremath{\to }{\color{correct}{#2}}\ensuremath{\:}}
\newcommand\green[1]{{\color{correct}{#1}}}

% horizontal rule
\newcommand\hr{
    \noindent\rule[0.5ex]{\linewidth}{0.5pt}
}

% hide parts
\newcommand\hide[1]{}

% si unitx
\usepackage{siunitx}
\sisetup{locale = FR}

% Environments
\makeatother
% For box around Definition, Theorem, \ldots
\usepackage{mdframed}
\mdfsetup{skipabove=1em,skipbelow=0em}
\theoremstyle{definition}
\newmdtheoremenv[nobreak=true]{definitie}{Definitie}
\newmdtheoremenv[nobreak=true]{eigenschap}{Eigenschap}
\newmdtheoremenv[nobreak=true]{gevolg}{Gevolg}
\newmdtheoremenv[nobreak=true]{lemma}{Lemma}
\newmdtheoremenv[nobreak=true]{propositie}{Propositie}
\newmdtheoremenv[nobreak=true]{stelling}{Stelling}
\newmdtheoremenv[nobreak=true]{wet}{Wet}
\newmdtheoremenv[nobreak=true]{postulaat}{Postulaat}
\newmdtheoremenv{conclusie}{Conclusie}
\newmdtheoremenv{toemaatje}{Toemaatje}
\newmdtheoremenv{vermoeden}{Vermoeden}
\newtheorem*{herhaling}{Herhaling}
\newtheorem*{intermezzo}{Intermezzo}
\newtheorem*{notatie}{Notatie}
\newtheorem*{observatie}{Observatie}
\newtheorem*{oef}{Oefening}
\newtheorem*{opmerking}{Opmerking}
\newtheorem*{praktisch}{Praktisch}
\newtheorem*{probleem}{Probleem}
\newtheorem*{terminologie}{Terminologie}
\newtheorem*{toepassing}{Toepassing}
\newtheorem*{uovt}{UOVT}
\newtheorem*{vb}{Voorbeeld}
\newtheorem*{vraag}{Vraag}

\newmdtheoremenv[nobreak=true]{definition}{Definition}
\newtheorem*{eg}{Example}
\newtheorem*{notation}{Notation}
\newtheorem*{previouslyseen}{As previously seen}
\newtheorem*{remark}{Remark}
\newtheorem*{note}{Note}
\newtheorem*{problem}{Problem}
\newtheorem*{observe}{Observe}
\newtheorem*{property}{Property}
\newtheorem*{intuition}{Intuition}
\newmdtheoremenv[nobreak=true]{prop}{Proposition}
\newmdtheoremenv[nobreak=true]{theorem}{Theorem}
\newmdtheoremenv[nobreak=true]{corollary}{Corollary}

% End example and intermezzo environments with a small diamond (just like proof
% environments end with a small square)
\usepackage{etoolbox}
\AtEndEnvironment{vb}{\null\hfill$\diamond$}%
\AtEndEnvironment{intermezzo}{\null\hfill$\diamond$}%
% \AtEndEnvironment{opmerking}{\null\hfill$\diamond$}%

% Fix some spacing
% http://tex.stackexchange.com/questions/22119/how-can-i-change-the-spacing-before-theorems-with-amsthm
\makeatletter
\def\thm@space@setup{%
  \thm@preskip=\parskip \thm@postskip=0pt
}


% Exercise 
% Usage:
% \oefening{5}
% \suboefening{1}
% \suboefening{2}
% \suboefening{3}
% gives
% Oefening 5
%   Oefening 5.1
%   Oefening 5.2
%   Oefening 5.3
\newcommand{\oefening}[1]{%
    \def\@oefening{#1}%
    \subsection*{Oefening #1}
}

\newcommand{\suboefening}[1]{%
    \subsubsection*{Oefening \@oefening.#1}
}


% \lecture starts a new lecture (les in dutch)
%
% Usage:
% \lecture{1}{di 12 feb 2019 16:00}{Inleiding}
%
% This adds a section heading with the number / title of the lecture and a
% margin paragraph with the date.

% I use \dateparts here to hide the year (2019). This way, I can easily parse
% the date of each lecture unambiguously while still having a human-friendly
% short format printed to the pdf.

\usepackage{xifthen}
\def\testdateparts#1{\dateparts#1\relax}
\def\dateparts#1 #2 #3 #4 #5\relax{
    \marginpar{\small\textsf{\mbox{#1 #2 #3 #5}}}
}

\def\@lecture{}%
\newcommand{\lecture}[3]{
    \ifthenelse{\isempty{#3}}{%
        \def\@lecture{Lecture #1}%
    }{%
        \def\@lecture{Lecture #1: #3}%
    }%
    \subsection*{\@lecture}
    \marginpar{\small\textsf{\mbox{#2}}}
}



% These are the fancy headers
\usepackage{fancyhdr}
\pagestyle{fancy}

% LE: left even
% RO: right odd
% CE, CO: center even, center odd
% My name for when I print my lecture notes to use for an open book exam.
% \fancyhead[LE,RO]{Gilles Castel}

\fancyhead[RO,LE]{\@lecture} % Right odd,  Left even
\fancyhead[RE,LO]{}          % Right even, Left odd

\fancyfoot[RO,LE]{\thepage}  % Right odd,  Left even
\fancyfoot[RE,LO]{}          % Right even, Left odd
\fancyfoot[C]{\leftmark}     % Center

\makeatother




% Todonotes and inline notes in fancy boxes
\usepackage{todonotes}
\usepackage{tcolorbox}

% Make boxes breakable
\tcbuselibrary{breakable}

% Verbetering is correction in Dutch
% Usage: 
% \begin{verbetering}
%     Lorem ipsum dolor sit amet, consetetur sadipscing elitr, sed diam nonumy eirmod
%     tempor invidunt ut labore et dolore magna aliquyam erat, sed diam voluptua. At
%     vero eos et accusam et justo duo dolores et ea rebum. Stet clita kasd gubergren,
%     no sea takimata sanctus est Lorem ipsum dolor sit amet.
% \end{verbetering}
\newenvironment{verbetering}{\begin{tcolorbox}[
    arc=0mm,
    colback=white,
    colframe=green!60!black,
    title=Opmerking,
    fonttitle=\sffamily,
    breakable
]}{\end{tcolorbox}}

% Noot is note in Dutch. Same as 'verbetering' but color of box is different
\newenvironment{noot}[1]{\begin{tcolorbox}[
    arc=0mm,
    colback=white,
    colframe=white!60!black,
    title=#1,
    fonttitle=\sffamily,
    breakable
]}{\end{tcolorbox}}




% Figure support as explained in my blog post.
\usepackage{import}
\usepackage{xifthen}
\usepackage{pdfpages}
\usepackage{transparent}
\newcommand{\incfig}[1]{%
    \def\svgwidth{\columnwidth}
    \import{./figures/}{#1.pdf_tex}
}

% Fix some stuff
% %http://tex.stackexchange.com/questions/76273/multiple-pdfs-with-page-group-included-in-a-single-page-warning
\pdfsuppresswarningpagegroup=1

\title{\Huge{Analysis}\\ Taylor's Theorem}
\author{\huge{Daniel Yu}}
\date{October 6, 2024}

\pdfsuppresswarningpagegroup=1

\begin{document}
\maketitle
\newpage% or \cleardoublepage
% \pdfbookmark[<level>]{<title>}{<dest>}
\tableofcontents
\pagebreak

\section{Taylor's Theorem}
\begin{lemma}
  Let
  \[
  Q(x) = a_{m} x^{m} + a_{m-1}x^{m-1} + \ldots + a_1 x + a_0
  .\] 
  a polynomial with real coefficients. Then $Q \in D(\R)$ and $Q'(x) = a_m m x^{m-1} + \ldots + a_1$ ($\Rightarrow Q \in C(\R)$).  

  \noindent\hrulefill

  \begin{proof}.
    Take $x_0 \in \R$ and $f(x) = x^{m}$. Then, 
   \begin{align*}
     (x_0 + h)^{m} &= x_0^{m} + \begin{pmatrix} m\\ 1 \end{pmatrix} x_0^{m-1}h +  \{ \begin{pmatrix} m\\ 2 \end{pmatrix}x_0^{m-2} h^{2} + \ldots + h^{m} \}  \\
                   &=  x_0^{m} + \begin{pmatrix} m\\ 1 \end{pmatrix} x_0^{m-1}h +  h \{ \begin{pmatrix} m\\ 2 \end{pmatrix}x_0^{m-2} h^{1} + \ldots + h^{m-1} \}   \\
                   &= x_{0}^{m}  + m x_0^{m-1}h +  h  \cdot r(h) 
   .\end{align*}
    Notice that $f'(x_0) = m x_0^{m-1}$, so $x^{m} \in D(x_0)$ and $(x^{m}) \mid_{x = x_0} = m x_0^{m-1}$.  
      
  \end{proof}
\end{lemma}

\begin{definition}
  Define the Taylor's polynomial of degree $n$ at point  $x_0$ is a polynomial approximation of the function $f(x)$ at the point $x_0$ which matches the functions value and first  $n$ derivatives at that point. It describes the \textit{neighborhood around  $f(x_0)$}
    \[
   T_{n}(x) = f(x_0) + \frac{f'(x_0)}{1!} (x-x_0) + \ldots + \frac{f^{n}( x_0)}{(n)!} (x-x_0)^{n}
   .\] 
\end{definition}

\begin{definition}
  \[
  f \in D^{m}((a,b)) \text{ if $f, f', \ldots, f^{m-1} \in D((a,b))$}
  .\] 
  \[
  f \in C^{m} ((a,b)) \text{ if $f, f', \ldots, f^{m=1}, f^{m} \in C((a,b))$} 
  .\] 
\end{definition}

\begin{note}
  $f \in D((a,b)) \Rightarrow f \in C^{m-1} ((a,b))$
\end{note}

\begin{theorem}{Taylor's Theorem (Lagrange)}\\
  Assume that $f:(a,b) \to \R$ and $f \in D^{n} ((a,b))$ where $n \geq 1$,  $x \in (a,b)$. Then,  $\forall x \in (a,b)$ $\exists x_{*} \in (x_0,x)$ [or $x_{*} \in (x, x_0)$ if $x \leq x_0$] such that:


\[
f(x) = f(x_0) + f'(x_0)(x - x_0) + \frac{f''(x_0)}{2!}(x - x_0)^2 + \dots + \frac{f^{(n)}(x_0)}{n!}(x - x_0)^n + \frac{f^{(n+1)}(x_{*})}{(n+1)!}(x - x_0)^{n+1}.
\]
 
Note that $\frac{f^{(n+1)}(x_{*})}{(n+1)!}(x - x_0)^{n+1}$ is known as the \textit{Langrangian Error Term} which represents the difference between the taylor polynomial $T_n$ and  $f(x)$

  
  \noindent\hrulefill

  \begin{proof}
   Consider the taylor polynomial,  
   \[
   T_{n-1}(x) = f(x_0) + \frac{f'(x_0)}{1!} (x-x_0) + \ldots + \frac{f^{n-1}( x_0)}{(n-1)!} (x-x_0)^{n-1}
   .\] 
   where $T_{n-1}(x_0) = f(x_0), T^{1}_{n-1} (x_0)= f'(x_0), \ldots, T^{n-1}_{n-1} (x_0) = f^{n-1} (x_0)$
    We will prove the taylors theorem with  $x$ replaced by  $x_1$. Then, take $x_0,x_1 \in (a,b)$, $x_1 \neq x_0$. First, we find $M \in \R$ such that:
    \[
    f(x_1) = T_{n-1} (x_1) + \frac{M}{n!} (x_1 - x_0)^{n}
    .\]
    Since $M$ obviously exists since we can solve th elinear equation. Our task now is to show that  $M = f^{n}(x_{*}) $ for some $x_{*} \in (x_0,x_1)$. Consider the function,
    \[
    g(x) = f(x) - T_{n-1} (x) - \frac{M}{n!} (x-x_0)^{n}
    .\] 
    Then,
    \begin{enumerate}
      \item Take $g(x_0) = 0$, $g'(x_0) = 0, \ldots ,g^{n-1} (x_0) = 0$.
      \item Take $g(x_1) = 0$.
    \end{enumerate}
    Then, $g \in D((x_0,x_1)) \cap C([x_0,x_1])$. By the mean-value theorem, $\exists  x_1 \in (a,b)$ such that 
    \[
    g'(x_2) = 0
    .\] 
    Since by (1),  $g'(x_{0}) = 0$, and $g' \in D((x_0,x_1)) \cap C([x_0, x_1])$ we find $x_3 \in (x_0,x_2)$ such that $g''(x_3) = 0$. (apply the mean value theorem again, this time with endpoints $\{x_0,x_2\} $). \\

  Continuing this process, we find: $\exists  x_{n+1} \in (x_0, x_{n}) $ such that, 
  \[
  g^{n} (x_{n+1}) =0
  .\] 
  Then we set $x_{*} = x_{n+1}$. Then $g^{n} (x_{*}) = 0$. Clearly, $x_{*} \in (x_0,x_1)$. But $g^{n} (x) = f^{n}(x) - M \Rightarrow 0 = g^{n} (x_{*}) = f^{n} (x_{*}) -M \Rightarrow M = f^{n} (x_{*}) $

  \end{proof}
\end{theorem}

\begin{lemma}
For any $a > 0$ we have that 
\[
\lim_{n \to \infty} \frac{a^{n}}{n!} = 0
.\] 

\begin{proof}
  Take $a > 0$. Then,  \[
  \frac{a^{n}}{n!} = \frac{a}{1} \cdot \frac{a}{2} \cdot \frac{a}{3}  \cdot \ldots \cdot \frac{a}{n}
  .\] 
  choose $n_0 \geq 1$ such that $\frac{a}{n_0} < \frac{1}{2}$. Then, for $n \geq n_0$, 
  \begin{align*}
    \frac{a}{n!} &= [\frac{a}{1}\frac{a}{2}\frac{a}{3} \ldots \frac{a}{n_0}] [\frac{a}{n_0+1} \ldots \frac{a}{n}] \\
                 &\leq [\frac{a}{1}\frac{a}{2}\frac{a}{3} \ldots \frac{a}{n_0}] (\frac{1}{2})^{n-n_0}, n \geq n_0 
  .\end{align*}
  This decays to 0 as $n \to \infty$, so $\frac{a^{n}}{n!} \to 0$
\end{proof}
\end{lemma}

\begin{lemma}{Stirling Approximation}\\
  $n! \sim \sqrt{2 \pi n} \left( \frac{n}{e} \right)^{e}  $ (rate of growth)
\end{lemma}

\section{Series}
 \begin{definition}
$(a_{n})_{n \geq 1}, a_n \in \R $ (or $\C$),  $n \geq 1$. We write 
   \[
  \sum^{\infty}_{n=1} a_n = A 
  .\] 
  and say that that the series converges to $A$ if the partial sums  $S_N = \sum_{n=1}^{N} a_n$ converges to $A$, i.e. $S_N \to A$ ( $N \to \infty$). \\

  Equivalently,  $\mid \sum_{n=1}^{N} a_n - A \mid  \to 0$ as ($N \to \infty$). 
\end{definition}

\begin{definition}
  For series, we will often consider the "upper limit",
  \[
    \overline{\lim_{n \to \infty}} x_n = \lim_{n \to \infty} (\sup \{x_n \mid n \geq N\}) 
  .\] 
  \begin{enumerate}
    \item $\overline{\lim_{n \to \infty}} \mid a_\frac{n}{a_{n+1}}\mid  < 1$ converges
    \item $\overline{\lim_{n \to \infty}} \sqrt{\mid a_n\mid }^{n} < 1$ converges
  \end{enumerate}
\end{definition}
\end{document}
