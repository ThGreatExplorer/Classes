\documentclass[a4paper]{article}
\usepackage[a4paper, margin=1in]{geometry}
\input{preamble}
\title{\Huge{Lie Algebra Intuition}}
\author{\huge{Daniel Yu}}
\date{October 1, 2024}

\pdfsuppresswarningpagegroup=1

\begin{document}
\maketitle
\newpage% or \cleardoublepage
% \pdfbookmark[<level>]{<title>}{<dest>}
\pagebreak

Lie Algebra: a vector space with multiplication (function from product of vector space to vector space itself). Take $V$ a vector space.
\begin{align*}
  V \times V &\longrightarrow  V\\
  (a,b) &\longmapsto [a,b]   \text{ read as bracket a,b}
.\end{align*}
Note, we can use $[ab]$ as shorthand for  $[a,b]$
\begin{enumerate}
  \item 
  \item Multiplication is not associative:

    \begin{note}
      Why is this natural? Take matrices w det  $\neq$ 0. Take $V=R^{n \times n}$. Consider inverible matrices $[A,B] = AB - BA$. It's a measure of commutativity, if  $AB = BA$ then $[A,B]=0$. This has the property of \textbf{lie bracket} 
    \end{note}
\end{enumerate}


\begin{remark}
  Notice that $V$ is a vector space, so it has all the vector space axioms. We are simply imposing additional structure on top of $V$.
\end{remark}

\begin{note}{Example}\\
  Consider the $2 \times 2$ matrices with  $tr(A) = 0$ aka  $SL_2 (\C)$. The basis is: 
  \[
     h = \begin{pmatrix} 1 & 0 \\ 0 & -1 \end{pmatrix}, e = \begin{pmatrix} 0 & 1 \\ 0 & 0 \end{pmatrix},
     f = \begin{pmatrix} 0 & 0 \\ 0 &  1 \end{pmatrix} 
  .\]  
\end{note}


\end{document}
